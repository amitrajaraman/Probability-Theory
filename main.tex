\documentclass{article}

\title{Probability Theory}
\author{Amit Rajaraman}
\date{Summer 2020}

\usepackage[utf8]{inputenc}
\usepackage{amsmath,amssymb,amsthm,amsfonts}
\usepackage{bbm}
\usepackage{enumerate}
\usepackage[tmargin=1in, bmargin=1in, lmargin=0.75in, rmargin=0.75in]{geometry}
\usepackage[colorlinks]{hyperref}
\usepackage{tikz}
\usepackage{cleveref}
\usetikzlibrary{arrows}

\usepackage{titlesec}
\titleformat{\section}[block]{\sffamily\Large\filcenter\bfseries}{\S\thesection.}{0.25cm}{\Large}
\titleformat{\subsection}[block]{\large\bfseries\sffamily}{\thesubsection.}{0.2cm}{\large}

\usepackage{fancyhdr}
\lhead{\sffamily{Probability Theory}}
\chead{\sffamily{\thepage}}
\rhead{\sffamily{-Amit Rajaraman}}
\cfoot{}
\pagestyle{fancy}

\setlength\parindent{0pt}

\newcommand{\Mod}[1]{\ (\mathrm{mod}\ #1)}
\newcommand{\indic}{\mathbbm{1}}
\newcommand{\iddistrib}{\stackrel{\mathcal{D}}{=}}

\renewcommand{\qedsymbol}{$\blacksquare$}
\renewcommand{\emptyset}{\varnothing}

\newcommand{\Ber}{\operatorname{Ber}}
\newcommand{\Rad}{\operatorname{Rad}}
\newcommand{\Poi}{\operatorname{Poi}}
\renewcommand{\d}[1]{\ensuremath{\operatorname{d}\!{#1}}}
\mathchardef\mhyphen="2D

\numberwithin{equation}{section}
\theoremstyle{definition}
\newtheorem{theorem}{Theorem}
\newtheorem{lemma}[theorem]{Lemma}
\newtheorem{corollary}[theorem]{Corollary}
\newtheorem{definition}{Definition}
\numberwithin{definition}{section}
\numberwithin{theorem}{section}
\newtheorem{exercise}{Exercise}
\newtheorem*{example}{Example}

\theoremstyle{remark}
\numberwithin{exercise}{section}
\newtheorem*{solution}{Solution}

\begin{document}

\maketitle
\thispagestyle{empty}

\setcounter {section}{-1}

\tableofcontents
\clearpage

\section{Notation}

$\mathbb{N}$ represents the set $\{1,2,\ldots\}$.

\vspace{1mm}
$\mathbb{N}_0$ represents the set $\{0,1,2,\ldots\}$.

\vspace{1mm}
For $x\in\mathbb{R}$ and $n\in\mathbb{N}$,
$$\binom{x}{r}=\frac{x(x-1)\cdots(x-r+1)}{r!}$$
is the generalised binomial coefficient.

\vspace{2mm}
For $n\in\mathbb{N}$, we denote $\{1,2,\ldots,n\}$ as $[n]$ and $\{0,1,2,\ldots,n\}$ as $[n]_0$

\vspace{2mm}
For $a\in\mathbb{R}^n$, we denote the $i$th coordinate of $a$ by $a_i$ for each $i=1,2,\ldots,n$.

\vspace{1mm}
For $a,b\in\mathbb{R}^n$, we write $a<b$ if $a_i<b_i$ for each $i=1,2,\ldots,n$.

\vspace{2mm}
Let $(a_n)_{n\in\mathbb{N}}$ be a sequence of reals. Then
\begin{align*}
    \limsup_{n\to\infty}a_n &= \lim_{n\to\infty}\left(\sup_{m\geq n}a_n\right) =    \inf_{n\to\infty}\left(\sup_{m\geq n}a_n\right) \\
    \liminf_{n\to\infty}a_n &= \lim_{n\to\infty}\left(\inf_{m\geq n}a_n\right) =    \sup_{n\to\infty}\left(\inf_{m\geq n}a_n\right)
\end{align*}

\vspace{2mm}
Let $A$ and $B$ be two sets. We denote by
$$A\triangle B=(A\setminus B)\cup(B\setminus A)$$
the \textit{symmetric difference} of $A$ and $B$.

\clearpage
\section{Measure Theory}

Before beginning a rigorous study of probability theory, it is necessary to understand some parts of basic measure theory.

\subsection{Classes of Sets}

\vspace{2mm}
Let $\Omega$ be a non-empty set and $\mathcal{A}\subseteq2^\Omega$, where $2^\Omega$ is the power set of $\Omega$. Then

\begin{definition}
    $\mathcal{A}$ is called
    \begin{itemize}
        \item $\cap$-closed (closed under intersections) or a $\pi$-system if $A\cap B\in \mathcal{A}$ for all $A,B\in\mathcal{A}$.
        \item $\sigma$-$\cap$-closed (closed under countable intersections) if $\bigcap_{i=1}^\infty A_i\in\mathcal{A}$ for any choice of countably many sets\\ $A_1, A_2, \ldots\in\mathcal{A}$.
        \item $\cup$-closed (closed under unions) if $A\cup B\in \mathcal{A}$ for all $A,B\in\mathcal{A}$.
        \item $\sigma$-$\cup$-closed (closed under countable unions) if $\bigcup_{i=1}^\infty A_i\in\mathcal{A}$ for any choice of countably many sets $A_1, A_2, \ldots\in\mathcal{A}$.
        \item $\setminus$-closed (closed under differences) if $A\setminus B\in\mathcal{A}$ for all $A,B\in\mathcal{A}$.
        \item closed under complements if $A^c=\Omega\setminus A\in\mathcal{A}$ for all $A\in\mathcal{A}$.
    \end{itemize}
\end{definition}

\begin{theorem}
\label{cupclosediffcapclosed}
    Let $\mathcal{A}$ be closed under complements. Then $\mathcal{A}$ is $\cup$-closed ($\sigma$-$\cup$-closed) if and only if $\mathcal{A}$ is closed $\cap$-closed ($\sigma$-$\cap$-closed).
\end{theorem}

The above is relatively straightforward to prove using De Morgan's Laws.

\begin{theorem}
\label{if setminus closed}
    Let $\mathcal{A}$ be $\setminus$-closed. Then
    \begin{enumerate}[(a)]
        \item $\mathcal{A}$ is $\cap$-closed,
        \item if $\mathcal{A}$ is $\sigma$-$\cup$-closed, then $\mathcal{A}$ is $\sigma$-$\cap$-closed.
        \item Any countable union of sets in $\mathcal{A}$ can be expressed as a countable union of pairwise disjoint sets in $\mathcal{A}$.
    \end{enumerate}
\end{theorem}
\begin{proof}
~
\begin{enumerate}[(a)]
    \item For $A,B\in\mathcal{A}$, $A\cap B=A\setminus(A\setminus B)\in\mathcal{A}$.
    \item Let $A_1,A_2,\ldots\in\mathcal{A}$. Then
    \begin{align*}
        \bigcap_{i=1}^\infty A_i &= \bigcap_{i=1}^\infty (A_1\cap A_i) \\
        &= \bigcap_{i=1}^\infty A_1\setminus (A_1\setminus A_i) \\
        &= A_1\setminus\bigcup_{i=1}^\infty (A_1\setminus A_i).
    \end{align*}
    \item Let $A_1,A_2,\ldots\in\mathcal{A}$. We then have
    $$\bigcup_{i=1}^\infty A_i = A_1 \uplus (A_2\setminus A_1)\uplus ((A_3\setminus A_2)\setminus A_1)\uplus\cdots$$
    The result follows.
    \end{enumerate}
\end{proof}

This equivalence between $\cap$ and $\cup$ if the class is $\setminus$-closed is apparent from De Morgan's laws.

\begin{definition}[Algebra]
\label{defAlgebra}
    A class of sets $\mathcal{A}\subseteq2^\Omega$ is called an \textit{algebra} if
    \begin{enumerate}[(i)]
        \item $\Omega\in\mathcal{A}$,
        \item $\mathcal{A}$ is $\setminus$-closed, and
        \item $\mathcal{A}$ is $\cup$-closed.
    \end{enumerate}
\end{definition}

\begin{definition}[$\sigma$-algebra]
\label{defSigAlgebra}
    A class of sets $\mathcal{A}\subseteq 2^\Omega$ is called a \textit{$\sigma$-algebra} if
    \begin{enumerate}[(i)]
        \item $\Omega\in\mathcal{A}$,
        \item $\mathcal{A}$ is closed under complements, and
        \item $\mathcal{A}$ is $\sigma$-$\cup$-closed.
    \end{enumerate}
\end{definition}

$\sigma$-algebras are also known as \textit{$\sigma$-fields}.

Note that any $\sigma$-algebra is an algebra (but the converse is not true).

\begin{theorem}
\label{algebra iff conditions}
    A class of sets $\mathcal{A}\subseteq2^\Omega$ is an algebra if and only if
    \begin{enumerate}[(a)]
        \item $\Omega\in\mathcal{A}$,
        \item $\mathcal{A}$ is closed under complements, and
        \item $\mathcal{A}$ is $\cap$-closed.
    \end{enumerate}
\end{theorem}

The proof of the above is left as an exercise to the reader.

\begin{definition}[Ring]
\label{defRing}
    A class of sets $\mathcal{A}\subseteq2^\Omega$ is called a \textit{ring} if
    \begin{enumerate}[(i)]
        \item $\emptyset\in\mathcal{A}$,
        \item $\mathcal{A}$ is $\setminus$-closed, and
        \item $\mathcal{A}$ is $\cup$-closed.
    \end{enumerate}
\end{definition}

Further, a ring is a \textit{$\sigma$-ring} if it is $\sigma$-$\cup$-closed.

\begin{definition}[Semiring]
\label{defSemiring}
    A class of sets $\mathcal{A}\subseteq2^\Omega$ is called a \textit{semiring} if
    \begin{enumerate}[(i)]
        \item $\emptyset\in\mathcal{A}$, 
        \item for any $A,B\in\mathcal{A}$, $A\setminus B$ is a finite union of mutually disjoint sets in $\mathcal{A}$, and
        \item $\mathcal{A}$ is $\cap$-closed.
    \end{enumerate}
\end{definition}

\begin{definition}[$\lambda$-system]
\label{defLamSystem}
    A class of sets $\mathcal{A}\subseteq2^\Omega$ is called a \textit{$\lambda$-system} (or \textit{Dynkin's $\lambda$-system}) if
    \begin{enumerate}[(i)]
        \item $\Omega\in\mathcal{A}$,
        \item for any $A,B\in\mathcal{A}$ with $B\subseteq A$, $A\setminus B\in\mathcal{A}$, and
        \item $\displaystyle\biguplus_{i=1}^\infty A_i\in\mathcal{A}$ for any choice of countably many pairwise disjoint sets $A_1,A_2,\ldots\in\mathcal{A}$.
    \end{enumerate}
\end{definition}

Among the above classes of sets, $\sigma$-algebras in particular are extremely important as we shall use them when defining probability.

\begin{theorem}
~
    \begin{enumerate}[(a)]
        \item Every $\sigma$-algebra is also a $\lambda$-system, an algebra and a $\sigma$-ring.
        \item Every $\sigma$-ring is a ring, and every ring is a semiring.
        \item Every algebra is a ring. An algebra on a finite set $\Omega$ is a $\sigma$-algebra.
    \end{enumerate}
\end{theorem}
\begin{proof}
~
    \begin{enumerate}[(a)]
        \item Let $\mathcal{A}$ be a $\sigma$-algebra. Then for any $A,B\in\mathcal{A}$, $A\setminus B = (A^c\cup B)^c \in \mathcal{A}$ and $A\cap B=(A^c\cup B^c)^c\in\mathcal{A}$, that is, $\mathcal{A}$ is $\setminus$-closed and $\cup$-closed. The result follows.
        
        \item Let $\mathcal{A}$ be a ring. Then \cref{cupclosediffcapclosed} implies that $\mathcal{A}$ is $\cap$-closed. The result follows.
        
        \item Let $\mathcal{A}$ be an algebra. With proof similar to the first part of this theorem, it is seen that $\mathcal{A}$ is $\setminus$-closed. We have $\emptyset=\Omega\setminus\Omega\in\mathcal{A}$ and thus, it is a ring. If $\Omega$ is finite, then $\mathcal{A}$ is finite. Thus any countable union of sets is a finite union of sets and the result follows.
        \end{enumerate}
\end{proof}

\begin{definition}
\label{defLimes}
    Let $A_1,A_2,\ldots$ be subsets of $\Omega$. Then
    $$\liminf_{n\to\infty}A_n:=\bigcup_{i=1}^\infty\bigcap_{j=i}^\infty A_j\text{ and }\limsup_{n\to\infty}A_n:=\bigcap_{i=1}^\infty\bigcup_{j=i}^\infty A_j$$
    are respectively called the \textit{limit inferior} and \textit{limit superior}, of the sequence $(A_n)_{n\in\mathbb{N}}$.
\end{definition}

The above may be rewritten as
\begin{align*}
    A_*:=\liminf_{n\to\infty}A_n &= \{\omega\in\Omega : |n\in\mathbb{N}:\omega\not\in A_n|<\infty\} \\
    A^*:=\limsup_{n\to\infty}A_n &= \{\omega\in\Omega : |n\in\mathbb{N}:\omega\in A_n|=\infty\}
\end{align*}

That is, $A_*$ represents the set of elements that do not appear in a finite number of sets and $A^*$ represents the set of elements that appear in an infinite number of sets. This implies that $A_*\subseteq A^*$. (Why is the opposite not necessarily true?)

\begin{definition}[Indicator function]
    Let $A$ be a subset of $\Omega$. The \textit{indicator function on $A$} is defined by
    $$\indic_A(x)
    =
    \begin{cases}
    1, & x\in A \\
    0, & x\not\in A
    \end{cases}
    $$
\end{definition}

With the above notation, it may be shown that
$$\indic_{A_*}=\liminf_{n\to\infty}\indic_{A_n}\text{ and }\indic_{A^*}=\limsup_{n\to\infty}\indic_{A_n}.$$

If $\mathcal{A}\subseteq2^\Omega$ is a $\sigma$-algebra and if $A_n\in\mathcal{A}$ for every $n\in\mathbb{N}$, then $A_*\in\mathcal{A}$ and $A^*\in\mathcal{A}$. This is clear from the fact that $\sigma$-algebras are closed under countable unions and intersections.

Proving the above statements is left as an exercise to the reader.

\begin{theorem}
\label{capofSigmaAisSigmaA}
    Let $I$ be some index set and $\mathcal{A}_i$ be a $\sigma$-algebra for each $i\in I$. Then the intersection $\mathcal{A}_I=\bigcap_{i\in I}\mathcal{A}_i$ is also a $\sigma$-algebra.
\end{theorem}
\begin{proof}
    We can prove this by using the three conditions in the definition of a $\sigma$-algebra.
    \begin{enumerate}[(i)]
        \item Since $\Omega\in \mathcal{A}_i$ for every $i\in I$, $\Omega\in \mathcal{A}_I$.
        \item Let $A\in \mathcal{A}_I$. Then $A\in \mathcal{A}_i$ for each $i\in I$ and thus $A^c\in \mathcal{A}_i$ for each $i\in I$. Therefore, $A^c\in \mathcal{A}_I$.
        \item Let $A_1,A_2,\ldots\in \mathcal{A}_I$. Then $A_n\in\mathcal{A}_i$ for each $n\in\mathbb{N}$ and $i\in I$. Thus $A=\bigcup_{n=1}^\infty A_n\in\mathcal{A}_i$ for each $i$ as well. The result follows.
    \end{enumerate}
\end{proof}

A similar statement holds for $\lambda$-systems.

\begin{theorem}
    Let $\mathcal{E}\subseteq2^\Omega$. Then there exists a smallest $\sigma$-algebra $\sigma(\mathcal{E})$ with $\mathcal{E}\subseteq\sigma(\mathcal{E})$:
    $$\sigma(\mathcal{E})=\bigcap_{\substack{\mathcal{A}\subseteq2^\Omega\text{ is a $\sigma$-algebra} \\ \mathcal{E}\subseteq\mathcal{A}}}\mathcal{A}.$$
    $\sigma(\mathcal{E})$ is called the \textit{$\sigma$-algebra generated by $\mathcal{E}$} and $\mathcal{E}$ is called a \textit{generator of $\sigma(\mathcal{E})$}. 
\end{theorem}
\begin{proof}
    $2^\Omega$ is a $\sigma$-algebra that contains $\mathcal{E}$ so the intersection is non-empty. By \cref{capofSigmaAisSigmaA}, $\sigma(\mathcal{E})$ is a $\sigma$-algebra.
\end{proof}

Similar to the above, $\delta(\mathcal{E})$ is defined as the $\lambda$-system generated by $\mathcal{E}$.

\vspace{2mm}
We always have the following:
\begin{enumerate}
    \item $\mathcal{E}\subseteq \sigma(\mathcal{E})$.
    \item If $\mathcal{E}_1\subseteq\mathcal{E}_2$, then $\sigma(\mathcal{E}_1)\subseteq\sigma(\mathcal{E}_2)$.
    \item $\mathcal{A}$ is a $\sigma$-algebra if and only if $\sigma(\mathcal{A})=\mathcal{A}$.
\end{enumerate}
Similar statements hold for $\lambda$-systems. Further, $\delta(\mathcal{E})\subseteq\sigma(\mathcal{E})$. This is to be expected as $\sigma$-algebras have more ``structure" than $\lambda$-systems.

\begin{theorem}[$\cap$-closed $\lambda$-system]
\label{cap closed lam sys}
    Let $\mathcal{D}\subseteq2^\Omega$ be a $\lambda$-system. Then $\mathcal{D}$ is a $\pi$-system if and only if $\mathcal{D}$ is a $\sigma$-algebra.
\end{theorem}
\begin{proof}
    If $\mathcal{D}$ is a $\sigma$-algebra, then it is obviously a $\pi$-system. Let $\mathcal{D}$ be a $\pi$-system. Then
    \begin{enumerate}[(a)]
        \item As $\mathcal{D}$ is a $\lambda$-system, $\Omega\in\mathcal{D}$.
        \item Let $A\in\mathbb{D}$. Since $\Omega\in\mathcal{D}$ and $\mathcal{D}$ is a $\lambda$-system, $A^c=\Omega\setminus A\in\mathcal{D}$.
        \item Let $A,B\in\mathcal{D}$. We have $A\cap B\in\mathcal{D}$. We now have $A\setminus B = A\setminus (A\cap B)\in\mathcal{D}$, that is, $\mathcal{D}$ is $\setminus$-closed.
        
        Let $A_1,A_2,\ldots\in\mathcal{D}$. Then by \cref{if setminus closed}, there exist $B_1,B_2,\ldots\in\mathcal{D}$ such that
        $$\bigcup_{i=1}^\infty A_i=\biguplus_{i=1}^\infty B_i\in\mathcal{D}.$$
    \end{enumerate}
    This completes the proof.
\end{proof}

\begin{theorem}[Dynkin's $\pi$-$\lambda$ theorem]
\label{dynkins pi lam theorem}
    If $\mathcal{E}\subseteq2^\Omega$ is a $\pi$-system, then $\delta(\mathcal{E})=\sigma(\mathcal{E})$.
\end{theorem}
\begin{proof}
    We already have $\delta(\mathcal{E})\subseteq\sigma(\mathcal{E})$. We must now prove the reverse inclusion. We shall show that $\delta(\mathcal{E})$ is a $\pi$-system.
    
    For each $E\in\delta(\mathcal{E})$, let
    $$\mathcal{D}_E=\{A\in\delta(\mathcal{E}) : A\cap E\in\delta(\mathcal{E})\}.$$
    
    To show that $\delta(\mathcal{E})$ is a $\pi$-system, it suffices to show that $\delta(\mathcal{E})\subseteq\mathcal{D}_E$ for all $E\in\delta(\mathcal{E})$. We shall first show that $\mathcal{D}_E$ is a $\lambda$-system for each $E\in\mathcal{E}$ by checking each of the conditions in \cref{defLamSystem}.
    \begin{enumerate}[(a)]
        \item We clearly have $\Omega\in\mathcal{D}_E$ as $\Omega\cap E=E$.
        \item For any $A,B\in\mathcal{D}_E$ with $A\subseteq B$, 
        $$(B\setminus A)\cap E = (B\cap E)\setminus(A\cap E)\in\delta(\mathcal{E}).$$
        \item Let $A_1,A_2,\ldots\in\mathcal{D}_E$ be mutually disjoint sets. Then
        $$\left(\biguplus_{i=1}^\infty A_i\right)\cap E = \biguplus_{i=1}^\infty\left(A_i\cap E\right)\in\delta(\mathcal{E}).$$
    \end{enumerate}
    Now since $\mathcal{D}_E$ is a $\lambda$-system and $\mathcal{E}\subseteq\mathcal{D}_E$ (Why?), $\delta(\mathcal{E})\subseteq\mathcal{D}_E$.
    
    Now that we have shown that $\delta(\mathcal{E})$ is a $\pi$-system, the result follows by \cref{cap closed lam sys}.
\end{proof}

\begin{definition}[Topology]
    Let $\Omega\neq\emptyset$ be an arbitrary set. A class of sets $\tau\subseteq2^\Omega$ is called a \textit{topology} on $2^\Omega$ if
    \begin{enumerate}[(i)]
        \item $\emptyset,\Omega\in\tau$,
        \item $\tau$ is $\cap$-closed, and
        \item for any $\mathcal{F}\subseteq\tau$, $\bigcup_{A\in\mathcal{F}}A\in\tau.$
    \end{enumerate}
\end{definition}

In the above case, the pair $(\Omega,\tau)$ is called a \textit{topological space}. The sets $A\in\tau$ are called \textit{open} and the sets $A\subseteq\Omega$ with $A^c\in\tau$ are called \textit{closed}.

\vspace{1mm}
Note that in contrast with $\sigma$-algebras, topologies are closed under only finite intersections but are also closed under arbitrary unions.

\vspace{2mm}
For example, consider the natural topology on $\mathbb{R}$ which consists of all open intervals in $\mathbb{R}$ and any arbitrary union of them.

\begin{definition}[Borel $\sigma$-algebra]
    Let $(\Omega,\tau)$ be a topological space. The $\sigma$-algebra
    $$\mathcal{B}(\Omega)=\mathcal{B}(\Omega,\tau)=\sigma(\tau)$$
    that is generated by the open sets is called the \textit{Borel $\sigma$-algebra on $\Omega$}. The elements $A\in\mathcal{B}(\Omega,\tau)$ are called \textit{Borel sets} or \textit{Borel measurable sets}.
\end{definition}


A Borel $\sigma$-algebra that we shall often encounter is $\mathcal{B}(\mathbb{R}^n)$ for $n\in\mathbb{N}$. Consider the following classes of sets:
\begin{align*}
    \mathcal{A}_1 &= \{A\subseteq\mathbb{R}^n:A\text{ is open}\} \\
    \mathcal{A}_2 &= \{A\subseteq\mathbb{R}^n:A\text{ is closed}\} \\
    \mathcal{A}_3 &= \{A\subseteq\mathbb{R}^n:A\text{ is compact}\} \\
    \mathcal{A}_4 &= \{(a,b):a,b\in\mathbb{Q}^n\text{ and }a<b\} \\
    \mathcal{A}_5 &= \{(a,b]:a,b\in\mathbb{Q}^n\text{ and }a<b\} \\
    \mathcal{A}_6 &= \{[a,b):a,b\in\mathbb{Q}^n\text{ and }a<b\} \\
    \mathcal{A}_7 &= \{[a,b]:a,b\in\mathbb{Q}^n\text{ and }a<b\} \\
    \mathcal{A}_8 &= \{(-\infty,b):b\in\mathbb{Q}^n\} \\
    \mathcal{A}_9 &= \{(-\infty,b]:b\in\mathbb{Q}^n\} \\
    \mathcal{A}_{10} &= \{(a,\infty):a\in\mathbb{Q}^n\} \\
    \mathcal{A}_{11} &= \{[a,\infty):a\in\mathbb{Q}^n\} \\
\end{align*}
It may be proved that $\mathcal{B}(\mathbb{R}^n)$ is generated by any of the classes of sets $\mathcal{A}_1,\mathcal{A}_2,\ldots,\mathcal{A}_{11}$.

\vspace{2mm}
For $A\in\mathcal{B}(\mathbb{R})$, we represent by $\left.\mathcal{B}(\mathbb{R})\right|_{A}$ the restriction of $\mathcal{B}(\mathbb{R})$ to $A$. It may be proved that this is equal to $\mathcal{B}(A)$, the $\sigma$-algebra generated by the open subsets of $A$.

\subsection{Measure}

\begin{definition}
    Let $\mathcal{A}\subseteq2^\Omega$ and let $\mu:\mathcal{A}\to[0,\infty]$ be a set function. We say that $\mu$ is
    \begin{enumerate}[(i)]
        \item \textit{monotone} if for any $A,B\in\mathcal{A}$, $A\subseteq B$ implies that $\mu(A)\leq\mu(B)$,
        
        \item \textit{additive} if for any choice of finitely many mutually disjoint sets $A_1,\ldots,A_n\in\mathcal{A}$ with $\biguplus_{i=1}^n A_i\in\mathcal{A}$,
        $$\mu\left(\biguplus_{i=1}^nA_i\right)=\sum_{i=1}^n\mu(A_i),$$
        
        \item $\sigma$-additive if for any choice of countably many mutually disjoint sets $A_1,A_2,\ldots\in\mathcal{A}$ with $\biguplus_{i=1}^\infty A_i\in\mathcal{A}$,
        $$\mu\left(\biguplus_{i=1}^\infty A_i\right)=\sum_{i=1}^\infty\mu(A_i),$$
        
        \item subadditive if for any choice of finitely many sets $A,A_1,A_2,\ldots,A_n\in\mathcal{A}$ with $A\subseteq\bigcup_{i=1}^nA_i$, we have $$\mu(A)\leq\sum_{i=1}^n\mu(A_i),\text{ and}$$
        
        \item $\sigma$-subadditive if for any choice of countably many sets $A,A_1,A_2,\ldots\in\mathcal{A}$ with $A\subseteq\bigcup_{i=1}^\infty A_i$, we have
        $$\mu(A)\leq\sum_{i=1}^\infty \mu(A_i).$$
    \end{enumerate}
\end{definition}

\begin{definition}
\label{measureDef}
    Let $\mathcal{A}$ be a semiring and $\mu:\mathcal{A}\to[0,\infty]$ be a set function with $\mu(\emptyset)=0$. $\mu$ is called a
    \begin{enumerate}[(i)]
        \item \textit{content} if $\mu$ is additive,
        \item \textit{premeasure} if $\mu$ is $\sigma$-additive, and
        \item \textit{measure} if $\mu$ is $\sigma$-additive and $\mathcal{A}$ is a $\sigma$-algebra.
    \end{enumerate}
\end{definition}

\begin{theorem}[Properties of contents]
\label{properties of content}
    Let $\mathcal{A}$ be a semiring and $\mu$ be a content on $\mathcal{A}$. Then
    \begin{enumerate}[(a)]
        \item If $\mathcal{A}$ is a ring, then $\mu(A\cup B)+\mu(A\cap B)=\mu(A)+\mu(B)$ for any $A,B\in\mathcal{A}$.
        
        \item $\mu$ is monotone. If $\mathcal{A}$ is a ring, then $\mu(B)=\mu(A)+\mu(B\setminus A)$ for any $A,B\in\mathcal{A}$ with $A\subseteq B$.
        
        \item $\mu$ is subadditive. If $\mu$ is $\sigma$-additive, then it is also $\sigma$-subadditive.
        
        \item If $\mathcal{A}$ is a ring, then $$\sum_{n=1}^\infty \mu(A_n)\leq\mu\left(\bigcup_{n=1}^\infty A_n\right)$$ for any choice of countably many mutually disjoint sets $A_1,A_2,\ldots\in\mathcal{A}$ with $\bigcup_{i=1}^\infty A_i\in\mathcal{A}$.
    \end{enumerate}
\end{theorem}
\begin{proof}
~
    \begin{enumerate}[(a)]
        \item Note that $A\cup B=A\uplus (B\setminus A)$ and $B=(A\cap B)\uplus (B\setminus A)$. As $\mu$ is additive,
        $$\mu(A\cup B)=\mu(A)+\mu(B\setminus A)\text{ and }\mu(B)=\mu(A\cap B)+\mu(B\setminus A).$$
        The result follows.
        
        \item Let $A\subseteq B$. If $B\setminus A\in\mathcal{A}$ (which is true in the case of a ring), we have $B=A\uplus (B\setminus A)$ and thus
        $$\mu(B)=\mu(A)+\mu(B\setminus A).$$
        If $\mathcal{A}$ is just a semiring, then there exist $n\in\mathbb{N}$ and mutually disjoint sets $C_1,C_2,\ldots,C_n\in\mathcal{A}$ such that $$B\setminus A=\biguplus_{i=1}^n C_i.$$
        In either case, we have $\mu(A)\leq \mu(B)$.
        
        \item Let $A, A_1,A_2,\ldots,A_n\in\mathcal{A}$ such that $A\subseteq\bigcup_{i=1}^n A_i$. Let $B_1=A_1$ and for each $k=2,3,\ldots,n$, let
        $$B_k = A_k\setminus\left(\bigcup_{i=1}^{k-1}A_i\right).$$
        Note that any two $B_i$s are disjoint. As $\mu$ is additive and monotone, we have
        \begin{align*}
            \mu(A) &\leq \mu\left(\bigcup_{i=1}^n A_i\right) \\
            &= \mu\left(\biguplus_{i=1}^n B_i\right) \\
            &= \sum_{i=1}^n \mu(B_i) \leq \sum_{i=1}^n \mu(A_i).
        \end{align*}
        We can similarly prove that if $\mu$ is $\sigma$-additive, then it is $\sigma$-subadditive.
        
        \item Let $A=\bigcup_{i=1}^\infty A_i\in\mathcal{A}$. Since $\mu$ is additive and monotone,
        $$\sum_{i=1}^m\mu(A_i)=\mu\left(\biguplus_{i=1}^m A_i\right)\leq \mu(A)\text{ for any $m\in\mathbb{N}$.}$$
        The result follows.
    \end{enumerate}
\end{proof}

Note that if equality holds in the fourth part of the above theorem, $\mu$ is a premeasure.

\begin{definition}[Finite content]
    Let $\mathcal{A}$ be a semiring. A content $\mu$ on $A$ is called
    \begin{enumerate}[(i)]
        \item \textit{finite} if $\mu(A)<\infty$ for all $A\in\mathcal{A}$ and
        \item \textit{$\sigma$-finite} if there exists a sequence of sets $\Omega_1,\Omega_2,\ldots\in\mathcal{A}$ such that $\Omega=\bigcup_{i=1}^\infty\Omega_i$ and $\mu(\Omega_i)<\infty$ for every $i\in\mathbb{N}$.
    \end{enumerate}
\end{definition}

\begin{definition}
    Let $A,A_1,A_2,\ldots$ be sets. We write
    \begin{enumerate}[(i)]
        \item $A_n\uparrow A$ if $A_1\subseteq A_2\subseteq A_3\subseteq\cdots$ and $\bigcup_{i=1}^\infty A_i=A$. In this case, we say that $A_n$ increases to $A$.
        \item $A_n\downarrow A$ if $A_1\supseteq A_2\supseteq A_3\supseteq\cdots$ and $\bigcap_{i=1}^\infty A_i=A$. In this case, we say that $A_n$ decreases to $A$.
    \end{enumerate}
\end{definition}

For example, if $A_n=\left(-\frac{1}{n},\frac{1}{n}\right)$ for $n\in\mathbb{N}$, then $A_n\downarrow \{0\}$.

\begin{definition}[Continuity of contents]
    Let $\mu$ be a content on the ring $\mathcal{A}$. $\mu$ is called
    \begin{enumerate}[(i)]
        \item \textit{lower semicontinuous} if $\lim_{n\to\infty}\mu(A_n)=\mu(A)$ for any $A\in\mathcal{A}$ and sequence $(A_n)_{n\in\mathbb{N}}$ such that $A_n\uparrow A$,
        \item \textit{upper semicontinuous} if $\lim_{n\to\infty}\mu(A_n)=\mu(A)$ for any $A\in\mathcal{A}$ and sequence $(A_n)_{n\in\mathbb{N}}$ such that $\mu(A_n)<\infty$ for some $n$ (this implies that it holds for all $n\in\mathbb{N}$) and $A_n\downarrow A$,
        \item \textit{$\emptyset$-continuous} if (ii) holds for $A=\emptyset$.
    \end{enumerate}
\end{definition}

\begin{theorem}
\label{tripledoubleEquivalence}
    Let $\mu$ be a content on the ring $\mathcal{A}$. The following properties are equivalent:
    \begin{enumerate}[(a)]
        \item $\mu$ is $\sigma$-additive (and hence a premeasure).
        \item $\mu$ is $\sigma$-subadditive.
        \item $\mu$ is lower semicontinuous.
        \item $\mu$ is $\emptyset$-continuous.
        \item $\mu$ is upper semicontinuous.
    \end{enumerate}
    Then (a)$\iff$(b)$\iff$(c)$\implies$(d)$\iff$(e).
    If $\mu$ is finite, then all five statements are equivalent.
\end{theorem}
\begin{proof}
~
    \begin{itemize}
        \item (a)$\implies$(b) ($\sigma$-additivity implies $\sigma$-subadditivity).
        
        This follows from \cref{properties of content}(c).
        
        \item (b)$\implies$(a) ($\sigma$-subadditivity implies $\sigma$-additivity).
        
        This follows from \cref{properties of content}(d).
        
        \item (a)$\implies$(c) ($\sigma$-additivity implies lower semicontinuity).
        
        Let $\mu$ be a premeasure and $A\in\mathcal{A}$. Let $A_1,A_2,\ldots\in\mathcal{A}$ such that $A_n\uparrow A$ and let $A_0=\emptyset$. Then
        $$\mu(A)=\sum_{i=1}^\infty \mu(A_i\setminus A_{i-1})=\lim_{n\to\infty}\sum_{i=1}^n\mu(A_i\setminus A_{i-1})=\lim_{n\to\infty}\mu(A_n).$$
        
        \item (c)$\implies$(a) (lower semicontinuity implies $\sigma$-additivity).
        
        Let $B_1,B_2,\ldots\in\mathcal{A}$ be mutually disjoint and let $B=\biguplus_{n=1}^\infty B_n\in\mathcal{A}$. Let $A_n=\biguplus_{i=1}^n B_i$ for each $n\in\mathbb{N}$. Then
        $$\mu(B)=\lim_{n\to\infty}\mu(A_n)=\sum_{i=1}^\infty\mu(B_i).$$
        Thus $\mu$ is $\sigma$-additive.
        
        \item (d)$\implies$(e) ($\emptyset$-continuity implies upper semicontinuity).
        
        Let $A,A_1,A_2,\ldots\in\mathcal{A}$ with $A_n\downarrow A$ and $\mu(A_1)<\infty$. Define $B_n=A_n\setminus A\in\mathcal{A}$ for valid $n$. Then $B_n\downarrow\emptyset$. Thus
        $$\lim_{n\to\infty}\mu(A_n)-\mu(A)=\lim_{n\to\infty}\mu(B_n)=0$$
        and the result is proved.
        
        \item (e)$\implies$(d) (upper semicontinuity implies $\emptyset$-continuity).
        
        This is obvious.
        
        \item (c)$\implies$(d) (lower semicontinuity implies $\emptyset$-continuity).
        
        Let $A_1,A_2,\ldots\in\mathcal{A}$ with $A_n\downarrow\emptyset$ and $\mu(A_1)<\infty$. Then $A_1\setminus A_n\in\mathcal{A}$ for all $n\in\mathbb{N}$ and $A_1\setminus A_n\uparrow A_1$. Thus
        $$\mu(A_1)=\lim_{n\to\infty}\mu(A_1)-\mu(A_n).$$
        Since $\mu(A_1)<\infty$, $\lim_{n\to\infty}A_n=0$ and the result is proved.
        
        \item (d)$\implies$(c) ($\emptyset$-continuity implies lower semicontinuity) if $\mu$ is finite.
        
        Let $A,A_1,A_2,\ldots\in\mathcal{A}$ with $A_n\uparrow A$. Then $A\setminus A_n\downarrow\emptyset$ and
        $$\lim_{n\to\infty}\mu(A)-\mu(A_n)=\lim_{n\to\infty}\mu(A\setminus A_n)=0.$$
        The result follows.
    \end{itemize}
\end{proof}

\begin{definition}[Measurable spaces]
~
\begin{enumerate}[(i)]
    \item A pair $(\Omega,\mathcal{A})$ consisting of a nonempty set $\Omega$ and a $\sigma$-algebra $\mathcal{A}\subseteq2^\Omega$ is called a \textit{measurable space}. The sets $A\in\mathcal{A}$ are called \textit{measurable sets}. If $\Omega$ is countable and $\mathcal{A}=2^\Omega$, then the space $(\Omega,2^\Omega)$ is called \textit{discrete}.
    
    \item A triple $(\Omega,\mathcal{A},\mu)$ is called a \textit{measure space} if $(\Omega,\mathcal{A})$ is a measurable space and $\mu$ is a measure on $\mathcal{A}$.
\end{enumerate}
\end{definition}


\subsection{Measurable Maps}

In measure theory, the homomorphisms (structure-preserving maps between objects) are studied as measurable maps.

\begin{definition}[Measurable map]
    Let $(\Omega,\mathcal{A})$ and $(\Omega',\mathcal{A}')$ be measurable spaces. A map $X:\Omega\to\Omega'$ is called \textit{$\mathcal{A}-\mathcal{A}'$-measurable} (or just measurable) if
    $$X^{-1}(A')\in\mathcal{A}\text{ for any }A'\in\mathcal{A}'.$$
\end{definition}

In this case, we write $X:(\Omega,\mathcal{A})\to(\Omega',\mathcal{A}')$.

\begin{theorem}[Generated $\sigma$-algebra]
    Let $(\Omega',\mathcal{A}')$ be a measurable space and $\Omega$ be a nonempty set. Let $X:\Omega\to\Omega'$ be a map. Then
    $$X^{-1}(\mathcal{A}')=\{X^{-1}(A'):A'\in\mathcal{A}'\}$$
    is the smallest $\sigma$-algebra with respect to which $X$ is measurable. We call $X^{-1}(\mathcal{A}')$ the \textit{$\sigma$-algebra generated by $X$} and denote it as $\sigma(X)$.
\end{theorem}
\begin{proof}
    Let $X$ be measurable with respect to some $\sigma$-algebra $\mathcal{A}$. Then $X^{-1}(A')\in\mathcal{A}$ for any $A'\in\mathcal{A}'$, that is, $\sigma(X)\subseteq\mathcal{A}$. Let us now prove that $\sigma(X)$ is a $\sigma$-algebra by checking each of the axioms in \cref{defSigAlgebra}.
    \begin{enumerate}
        \item As $\Omega'\in\mathcal{A}'$ and $X^{-1}(\Omega')=\Omega$, $\Omega\in\sigma(X)$.
        
        \item Let $A\in\sigma(X)$ and $A'\in\mathcal{A}'$ such that $X^{-1}(A')=A$. Then as $\mathcal{A}'$ is closed under complements,
        $$\Omega\setminus A = X^{-1}(\Omega')\setminus X^{-1}(A') = X^{-1}(\Omega'\setminus A')\in\sigma(X).$$
        Therefore, $\sigma(X)$ is closed under complements.
        
        \item Let $A_1,A_2\ldots\in\sigma(X)$ and $A_1',A_2',\ldots\in\mathcal{A}'$ such that $A_i=X^{-1}(A_i')$ for each $i\in\mathbb{N}$. Then as $\mathcal{A}'$ is $\sigma$-$\cup$-closed,
        $$\bigcup_{i\in\mathbb{N}} A_i = \bigcup_{i\in\mathbb{N}} X^{-1} (A_i') = X^{-1}\left(\bigcup_{i\in\mathbb{N}} A_i'\right) \in\sigma(X)$$
    \end{enumerate}
    Therefore, $\sigma(X)$ is a $\sigma$-algebra.
\end{proof}

\begin{theorem}
\label{generating pi system fixed under preimage}
    Let $(\Omega,\mathcal{A})$ and $(\Omega',\mathcal{A}')$ be measurable spaces and $X:\Omega\to\Omega'$ be a map.
    Let $\mathcal{E}'\subseteq\mathcal{A}'$ be a class of sets. Then $\sigma(X^{-1}(\mathcal{E}'))=X^{-1}(\sigma(\mathcal{E}'))$.
\end{theorem}
\begin{proof}
    We have that $X^{-1}(\mathcal{E})\subseteq X^{-1}(\sigma(\mathcal{E}))=\sigma(X^{-1}(\sigma(\mathcal{E})))$. This implies that
    $$\sigma(X^{-1}(\mathcal{E}))\subseteq X^{-1}(\sigma(\mathcal{E})).$$
    To establish the reverse inclusion, consider
    $$\mathcal{A}_0'=\{A'\in\sigma(\mathcal{E}'):X^{-1}(A')\in\sigma(X^{-1}(\mathcal{E}'))\}$$
    We shall show that $\mathcal{A}_0'$ is a $\sigma$-algebra.
    \begin{enumerate}[(a)]
        \item Clearly, $\Omega'\in\mathcal{A}_0'$ as $\Omega\in\sigma(X^{-1}(\mathcal{E}'))$ and $\Omega'\in\sigma(\mathcal{E}')$.
        
        \item Let $A_0'\in\mathcal{A}_0'$. Then
        $$X^{-1}((A_0')^c) = (X^{-1}(A_0'))^c \in \sigma(X^{-1}(\mathcal{E}'))$$
        and thus $\mathcal{A_0}'$ is closed under complements.
        
        \item Let $A_1',A_2',\ldots\in\mathcal{A}_0'$. Then
        $$X^{-1}\left(\bigcup_{i=1}^\infty A_i'\right) = \bigcup_{i=1}^\infty X^{-1}\left(A_i'\right)\in \sigma(X^{-1}(\mathcal{E}')).$$
        Thus, $\mathcal{A}_0'$ is $\sigma$-$\cup$-closed.
    \end{enumerate}
    
    Now, note that $\mathcal{E}'\subseteq\mathcal{A}_0'$ and $\mathcal{A}_0'\subseteq\sigma(\mathcal{E}')$. This implies that $\mathcal{A}_0'=\sigma(\mathcal{E}')$, and thus $X^{-1}(\sigma(\mathcal{E}'))\subseteq\sigma(X^{-1}(\mathcal{E}'))$.
    
    This proves the result.
\end{proof}

\begin{corollary}
\label{measurable if generator inv}
    Let $(\Omega,\mathcal{A})$ and $(\Omega',\mathcal{A}')$ be measurable spaces and $X:\Omega\to\Omega'$ be a map. Let $\mathcal{E}'\subseteq\mathcal{A}'$ be a class of sets. Then $X$ is $\mathcal{A}$-$\sigma(\mathcal{E}')$ measurable if and only if $X^{-1}(\mathcal{E}')\in\mathcal{A}$. If in particular $\sigma(\mathcal{E}')=\mathcal{A}'$, then $X$ is $\mathcal{A}-\mathcal{A}'$-measurable if and only if $X^{-1}(\mathcal{E}')\subseteq\mathcal{A}'$.
\end{corollary}

\begin{theorem}
        Let $(\Omega,\mathcal{A})$, $(\Omega',\mathcal{A}')$, and $(\Omega'',\mathcal{A}'')$ be measurable spaces and let $X:\Omega\to\Omega'$ and $X':\Omega'\to\Omega''$ be measurable. Then $Y=X'\circ X:\Omega\to\Omega''$ is $\mathcal{A}-\mathcal{A}''$-measurable.
\end{theorem}
\begin{proof}
    This is due to the fact that
    $$Y^{-1}(\mathcal{A}'')\subseteq X^{-1}((X^{-1})(\mathcal{A}''))\subseteq X^{-1}(\mathcal{A}')\subseteq \mathcal{A}.$$
\end{proof}

The above theorem just states that the composition of two measurable maps is measurable.

\begin{theorem}[Measurability of Continuous Maps]
    Let $(\Omega,\tau)$ and $(\Omega',\tau')$ be topological spaces and let $f:\Omega\to\Omega'$ be a continuous map. Then $f$ is $\mathcal{B}(\Omega)-\mathcal{B}(\Omega')$-measurable.
\end{theorem}
\begin{proof}
    As $\mathcal{B}(\Omega')=\sigma(\tau')$, by \cref{measurable if generator inv} it is enough to show that $f^{-1}(A')\in\sigma(\tau)$ for all $A'\in\tau'$. However, since $f$ is continuous, $f^{-1}(A')\in\tau$ for all $A'\in\tau'$ so the result follows.
\end{proof}

\begin{theorem}
    Let $X_1,X_2,\ldots$ be measurable maps $(\Omega,\mathcal{A})\to(\overline{\mathbb{R}}, \mathcal{B}(\overline{\mathbb{R}}))$. Then $\inf_{n\in\mathbb{N}}X_n$, $\sup_{n\in\mathbb{N}}X_n$, $\liminf_{n\in\mathbb{N}}X_n$, and $\limsup_{n\in\mathbb{N}}X_n$ are also measurable.
\end{theorem}
\begin{proof}
    For any $x\in\overline{\mathbb{R}}$,
    $$\left(\inf_{n\in\mathbb{N}}X_n\right)^{-1}([-\infty,x))=\bigcup_{n=1}^\infty (X_n)^{-1}([-\infty,x))\in\mathcal{A}.$$
    The first part of the result follows by \cref{measurable if generator inv}. The proof for $\sup_{n\in\mathbb{N}}X_n$ is similar.
    
    For $\limsup_{n\in\mathbb{N}}$, consider the sequence $(Y_n)_{n\in\mathbb{N}}$ where $Y_n=\sup_{m\geq n}X_m$. Each $Y_m$ is measurable. Then since $\inf_{n\in\mathbb{N}}Y_m$ is measurable, the result follows.
\end{proof}

\begin{definition}[Simple Function]
    Let $(\Omega,\mathcal{A})$ be a measurable space. A map $f:\Omega\to\mathbb{R}$ is called \textit{simple} if there exists some $n\in\mathbb{N}$, mutually disjoint sets $A_1,\ldots,A_n\in\mathcal{A}$, and $\alpha_1,\ldots,\alpha_n\in\mathbb{R}$ such that
    $$f=\sum_{i=1}^n \alpha_i\indic_{A_i}.$$
\end{definition}

\begin{definition}
    Let $f,f_1,f_2,\ldots$ be maps $\Omega\to\overline{\mathbb{R}}$ such that
    $$f_1(\omega)\leq f_2(\omega)\leq\cdots\text{ and }\lim_{n\to\infty}f_n(\omega)=\omega\text{ for all }\omega\in\Omega.$$
    We then write $f_n\uparrow f$. Similarly, we write $f_n\downarrow f$ if $(-f_n)\uparrow(-f)$.
\end{definition}

\begin{theorem}
\label{measurable function simple sequence}
    Let $(\Omega,\mathcal{A})$ be a measurable space and let $f:\Omega\to[0,\infty]$ be measurable. Then
    \begin{enumerate}[(a)]
        \item There exists a sequence $(f_n)_{n\in\mathbb{N}}$ of non-negative simple functions such that $f_n\uparrow f$.
        \item There are measurable sets $A_1,A_2,\ldots\in\mathcal{A}$ and $\alpha_1,\alpha_2,\ldots\in[0,\infty)$ such that $f=\sum_{i=1}^n \alpha_i\indic_{A_i}$.
    \end{enumerate}
\end{theorem}
\begin{proof}
    ~
    \begin{enumerate}[(a)]
        \item For $n\in\mathbb{N}_0$, define
        $$f_n=\min\{n, 2^{-n}\lfloor2^nf\rfloor\}.$$
        Each $f_n$ is measurable. (Why?) Since it can take at most $n2^n+1$ distinct values, each $f_n$ is simple. Clearly, $f_n\uparrow f$.
        
        \item Let $f_n$ be the same as above. For $n\in\mathbb{N}$ and $i\in[2^n]$, define
        $$B_{n,i}=\{\omega:f_n(\omega)-f_{n-1}(\omega)=i2^{-n}\}\text{ and }\beta_{n,i}=i2^{-n}.$$
        Then $f_n-f_{n-1}=\sum_{i=1}^{2^n} \beta_{n,i}\indic_{B_{n,i}}$. Changing the enumeration from $(n,i)$ to $m$, we get some $(\alpha_m)_{m\in\mathbb{N}}$ and $(A_m)_{m\in\mathbb{N}}$ such that
        $$f=f_0 + \sum_{n=1}^\infty(f_n-f_{n-1})=\sum_{m=1}^\infty \alpha_m\indic_{A_m}.$$
    \end{enumerate}
\end{proof}

\subsection{Outer Measure}

\begin{lemma}
\label{uniquely defined by base pi sys}
    Let $(\Omega,\mathcal{A},\mu)$ be a $\sigma$-finite measure space and $\mathcal{E}\subseteq\mathcal{A}$ be a $\pi$-system that generates $\mathcal{A}$. Assume there exists sequence $\Omega_1,\Omega_2\ldots\in\mathcal{E}$ such that $\bigcup_{i=1}^\infty\Omega_i=\Omega$ and $\mu(\Omega_i)<\infty$ for all $i\in\mathbb{N}$. Then $\mu$ is uniquely determined by the values $\mu(E)$, $E\in\mathcal{E}$.
    
    If $\Omega\in\mathcal{A}$ and $\mu(\Omega)=1$, then the existence of the sequence $(\Omega_n)_{n\in\mathbb{N}}$ is not required.
\end{lemma}
\begin{proof}
    Let $\nu$ be a $\sigma$-finite measure on $(\Omega,\mathcal{A})$ such that $\mu(E)=\nu(E)$ for all $E\in\mathcal{E}$.
    
    \vspace{1mm}
    Let $E\in\mathcal{E}$ with $\mu(E)<\infty$. Consider
    $$\mathcal{D}_E=\{A\in\mathcal{A}: \mu(A\cap E)=\nu(A\cap E)\}.$$
    
    We claim that $\mathcal{D}_E$ is a $\lambda$-system. We shall prove this by checking each of the conditions of \cref{defLamSystem}.
    
    \begin{enumerate}[(a)]
        \item Clearly, $\Omega\in\mathcal{D}_E$.
        \item Let $A,B\in\mathcal{D}_E$ with $B\subseteq A$. Then
        \begin{align*}
            \mu((A\setminus B)\cap E) &= \mu(A\cap E) - \mu(B\cap E)\quad\text{(using \cref{properties of content})} \\
            &= \nu(A\cap E) - \nu(B\cap E) \\
            &= \nu((A\setminus B)\cap E).
        \end{align*}
        That is, $(A\setminus B)\in\mathcal{D}_E$.
        \item Let $A_1,A_2,\ldots\in\mathcal{D}_E$ be mutually disjoint sets. Then
        \begin{align*}
            \mu\left(\left(\biguplus_{i=1}^\infty A_i\right)\cap E\right) &= \sum_{i=1}^\infty \mu(A_i\cap E) \\
            &= \sum_{i=1}^\infty\nu(A_i\cap E) \\
            &= \nu\left(\left(\biguplus_{i=1}^\infty A_i\right)\cap E\right).
        \end{align*}
        Therefore, $\biguplus_{i=1}^\infty A_i\in\mathcal{D}_E$ and $\mathcal{D}_E$ is a $\lambda$-system.
    \end{enumerate}
    As $\mathcal{E}\subseteq\mathcal{D}_E$ (Why?), $\delta(\mathcal{E})\subseteq\mathcal{D}_E$. Since $\mathcal{E}$ is a $\pi$-system, \cref{dynkins pi lam theorem} implies that
    $$\mathcal{A}\supseteq\mathcal{D}_E\supseteq\delta(\mathcal{E})=\sigma(\mathcal{E})=\mathcal{A}.$$
    Hence $\mathcal{D}_E=\mathcal{A}$.
    
    Therefore, $\mu(A\cap E)=\nu(A\cap E)$ for any $A\in\mathcal{A}$ and $E\in\mathcal{E}$ with $\mu(E)<\infty$.
    
    Now, let $\Omega_1,\Omega_2,\ldots\in\mathcal{E}$ be a sequence such that $\bigcup_{i=1}^\infty\Omega_i=\Omega$ and $\mu(\Omega_i)<\infty$ for all $i\in\mathbb{N}$. Let $E_0=\emptyset$ and $E_n=\bigcup_{i=1}^n\Omega_i$ for each $n\in\mathbb{N}$. Note that
    $$E_n=\biguplus_{i=1}^n(E_{i-1}^c\cap \Omega_i).$$
    Therefore for any $A\in\mathcal{A}$ and $n\in\mathbb{N}$,
    \begin{align*}
        \mu(A\cap E_n) &= \sum_{i=1}^n\mu((A\cap E_{i-1}^c)\cap\Omega_i) \\
        &= \sum_{i=1}^n\nu((A\cap E_{i-1}^c)\cap\Omega_i) = \nu(A\cap E_n).
    \end{align*}
    
    Now, since $E_n\uparrow\Omega$ and $\mu,\nu$ are lower semicontinuous (by \cref{tripledoubleEquivalence}),
    \begin{align*}
        \mu(A) &= \lim_{n\to\infty}\mu(A\cap E_n) \\
        &= \lim_{n\to\infty}\nu(A\cap E_n) = \nu(A)
    \end{align*}
    
    This proves the result.
    
    \vspace{2mm}
    The second part of the theorem is trivial as $\mathcal{E}\cup\{\Omega\}$ is a $\pi$-system that generates $\mathcal{A}$. Hence one can choose the constant sequence $E_n=\Omega, n\in\mathbb{N}$.
\end{proof}

\begin{definition}[Outer Measure]
    A function $\mu^*:2^\Omega\to[0,\infty]$ is called an \textit{outer measure} if
    \begin{enumerate}[(i)]
        \item $\mu^*(\emptyset)=0$,
        \item $\mu^*$ is monotone, and
        \item $\mu^*$ is $\sigma$-subadditive.
    \end{enumerate}
\end{definition}

\begin{lemma}
\label{set of countable coverings outer measure}
    Let $\mathcal{A}\subseteq 2^\Omega$ be an arbitrary class of sets with $\emptyset\in\mathcal{A}$ and let $\mu$ be a nonnegative set function on $\mathcal{A}$ with $\mu(\emptyset)=0$. For $A\subseteq\Omega$, define the set of countable coverings $\mathcal{F}$ with sets $F\in\mathcal{A}$
    $$\mathcal{U}(A)=\left\{\mathcal{F}\subseteq\mathcal{A} : \mathcal{F}\text{ is countable and }A\subseteq\bigcup_{F\in\mathcal{F}}F\right\}.$$
    Define
    $$\mu^*(A)=\inf\left\{\sum_{F\in\mathcal{F}}\mu(F) : \mathcal{F}\in\mathcal{U}(A) \right\}$$
    where $\inf\emptyset=\infty$. Then $\mu^*$ is an outer measure. If $\mu$ is $\sigma$-subadditive then $\mu^*(A)=\mu(A)$ for all $A\in\mathcal{A}$.
\end{lemma}
\begin{proof}
    Let us check each of the three conditions in the definition of an outer measure.
    \begin{enumerate}[(a)]
        \item Since $\emptyset\in\mathcal{A}$, we have $\{\emptyset\}\in\mathcal{U}(\emptyset)$ and hence $\mu(\emptyset)=0$.
        
        \item If $A\subseteq B$, then $\mathcal{U}(A)\subseteq\mathcal{U}(B)$, and hence $\mu^*(A)\leq\mu^*(B)$.
        
        \item Let $A,A_1,A_2,\ldots\subseteq\Omega$ such that $A\subseteq\bigcup_{i=1}^\infty A_i$. We claim that $\mu^*(A)\leq\sum_{i=1}^\infty \mu^*(A_i)$.
        
        Without loss of generality, assume that $\mu^*(A_i)<\infty$ and hence $\mathcal{U}(A_i)\neq\emptyset$ for all $i\in\mathbb{N}$. Fix some $\varepsilon>0$. Now, for every $n\in\mathbb{N}$, we may choose a covering $\mathcal{F}_n\in\mathcal{U}(A_n)$ such that
        $$\sum_{F\in\mathcal{F}_n}\mu(F)\leq\mu^*(A_n)+\varepsilon2^{-n}.$$
        Then let $\mathcal{F}=\bigcup_{n=1}^\infty \mathcal{F}_n\in\mathcal{U}(A)$.
        $$\mu^*(A)\leq\sum_{F\in\mathcal{F}}\mu(F)\leq\sum_{n=1}^\infty\sum_{F\in\mathcal{F}_n}\mu(F)\leq\sum_{n=1}^\infty \mu^*(A_n) + \varepsilon.$$
        
        This proves the first part of the result.
    \end{enumerate}
    
    To prove the next part of the result, first note that since $\{A\}\in\mathcal{U}(A)$, we have $\mu^*(A)\leq\mu(A)$. If $\mu$ is $\sigma$-subadditive, then for any $\mathcal{F}\in\mathcal{U}(A)$,
    $$\sum_{F\in\mathcal{F}}\mu(F)\geq\mu(A).$$
    It follows that $\mu^*(A)\geq\mu(A).$
\end{proof}

\begin{definition}[$\mu^*$-measurable sets]
    Let $\mu^*$ be an outer measure. A set $A\in 2^\Omega$ is called \textit{$\mu^*$-measurable} if
    $$\mu^*(A\cap E) + \mu^*(A^c\cap E) = \mu^*(E)\text{ for any }E\in 2^\Omega.$$
    We write $\mathcal{M}(\mu^*)=\{A\subseteq\Omega:A\text{ is }\mu^*\text{-measurable}\}.$
\end{definition}

\begin{lemma}
\label{mu measurable iff leq}
    $A\in\mathcal{M}(\mu^*)$ if and only if
    $$\mu^*(A\cap E) + \mu^*(A^c\cap E) \leq \mu^*(E)\text{ for any }E\in 2^\Omega.$$
\end{lemma}
\begin{proof}
    As $\mu^*$ is subadditive, we trivially have
    $$\mu^*(A\cap E) + \mu^*(A^c\cap E) \geq \mu^*(E)\text{ for any }E\in 2^\Omega.$$
    The result follows.
\end{proof}

\begin{lemma}
    $\mathcal{M}(\mu^*)$ is an algebra.
\end{lemma}
\begin{proof}
    We shall check the conditions given in the definition of an algebra \cref{defAlgebra}.
    \begin{enumerate}[(a)]
        \item We clearly have $\Omega\in\mathcal{M}(\mu^*)$.
        \item By definition, $\mathcal{M}(\mu^*)$ is closed under complements.
        \item We must check that $\mathcal{M}(\mu^*)$ is closed under intersections. Let $A,B\in\mathcal{M}(\mu^*)$ and $E\subseteq\Omega$. Then
        \begin{align*}
            \mu^*((A\cap B)\cap E)+\mu^*((A\cap B)^c\cap E) &= \mu^*((A\cap B)\cap E) \\ &\hspace{4mm}+\mu^*((A\cap B^c\cap E)\cup(A^c\cap B\cap E)\cup(A^c\cap B^c\cap E)) \\
            &\leq \mu^*(A\cap (B\cap E))+\mu^*(A\cap (B^c\cap E))\\ &\hspace{4mm}+\mu^*(A^c\cap (B\cap E)) + \mu^*(A^c\cap (B^c\cap E)) \\
            &= \mu^*(B\cap E) + \mu^*(B^c\cap E)\quad\text{(since $A\in\mathcal{M}(\mu^*)$)} \\
            &= \mu^*(E). \quad\text{(since $B\in\mathcal{M}(\mu^*)$)}
        \end{align*}
    \end{enumerate}
    This proves the result.
\end{proof}

\begin{lemma}
\label{outer measure is sig additive on M}
    An outer measure $\mu^*$ is $\sigma$-additive on $\mathcal{M}(\mu^*)$.
\end{lemma}
\begin{proof}
    Let $A,B\in\mathcal{M}(\mu^*)$ with $A\cap B=\emptyset$. Then
    \begin{align*}
        \mu^*(A\cup B) &= \mu^*(A\cap(A\cup B)) + \mu^*(A^c\cap(A\cup B)) \\
        &= \mu^*(A) + \mu^*(B).
    \end{align*}
    That is, $\mu^*$ is additive (and thus a content) on $\mathcal{M}(\mu^*)$. Since $\mu^*$ is $\sigma$-subadditive, \cref{tripledoubleEquivalence} gives the required result.
\end{proof}

\begin{lemma}
    If $\mu^*$ is an outer measure, $\mathcal{M}(\mu^*)$ is a $\sigma$-algebra.
\end{lemma}
\begin{proof}
    We have already shown that $\mathcal{M}(\mu^*)$ is an algebra (and thus a $\pi$-system). Using \cref{cap closed lam sys}, it is sufficient to show that $\mathcal{M}(\mu^*)$ is a $\lambda$-system.
    
    Let $A_1,A_2,\ldots\in\mathcal{M}(\mu^*)$ be mutually disjoint sets and let $A=\biguplus_{i=1}^\infty A_i$. Further, for each $n\in\mathbb{N}$, let $B_n=\biguplus_{i=1}^n A_i$. We must show that $M\in\mathcal{M}(\mu^*)$.
    
    For any $E$ and valid $n\in\mathbb{N}$, we have
    \begin{align*}
        \mu^*(E\cap B_{n+1}) &= \mu^*((E\cap B_{n+1})\cap B_n) + \mu^*((E\cap B_{n+1})\cap B_n^c) \\
        &= \mu^*(E\cap B_n) + \mu^*(E\cap A_{n+1}).
    \end{align*}
    By a simple induction, it follows that
    $$\mu(E\cap B_{n})=\sum_{i=1}^n \mu^*(E\cap A_i).$$
    Since $\mu^*$ is monotonic, we have
    \begin{align*}
        \mu^*(E) &= \mu^*(E\cap B_n) + \mu^*(E\cap B_n^c) \\
        &\geq \mu^*(E\cap B_n) + \mu^*(E\cap A^c) \\
        &= \sum_{i=1}^n\mu^*(E\cap A_i) + \mu^*(E\cap A^c).
    \end{align*}
    Letting $n\to\infty$ and using the fact that $\mu^*$ is $\sigma$-subadditive, we have
    \begin{align*}
        \mu^*(E) &\geq \sum_{i=1}^\infty\mu^*(E\cap A_i) + \mu^*(E\cap A^c) \\
        &\geq \mu^*(E\cap A) + \mu^*(E\cap A^c)
    \end{align*}
    Therefore, $A\in\mathcal{M}(\mu^*)$ and this completes the proof.
\end{proof}

\subsection{The Approximation and Extension Theorems}

\begin{theorem}[Approximation Theorem for Measures]
\label{Approximation Thm for Measures}
    Let $\mathcal{A}\subseteq 2^\Omega$ be a semiring and let $\mu$ be a measure on $\sigma(\mathcal{A})$ that is $\sigma$-finite on $\mathcal{A}$.
    For any $A\in\sigma(\mathcal{A})$ with $\mu(\mathcal{A})<\infty$ and any $\varepsilon>0$, there exists $n\in\mathbb{N}$ and mutually disjoint sets $A_1,A_2,\ldots,A_n\in \mathcal{A}$ such that $\mu\left(A\triangle\bigcup_{i=1}^n A_n\right)<\varepsilon$.
\end{theorem}
\begin{proof}
    Consider the outer measure $\mu^*$ as defined in \cref{set of countable coverings outer measure}. Note that by \cref{set of countable coverings outer measure} and \cref{uniquely defined by base pi sys}, $\mu$ and $\mu^*$ are equal on $\sigma(\mathcal{A})$. By the definition of $\mu^*$, for any $A\in\mathcal{A}$, there exists a covering $B_1,B_2,\ldots\in\mathcal{A}$ of $A$ such that
    $$\mu(A)\geq\sum_{i=1}^\infty\mu(B_i) - \varepsilon/2.$$
    Since $\mu(A)<\infty$, there exists some $n\in\mathbb{N}$ such that $\sum_{i=n+1}^\infty \mu(B_i) < \varepsilon/2$. Now, let $D=\bigcup_{i=1}^n B_i$ and $E=\bigcup_{i=n+1}^\infty B_i$. We have
    \begin{align*}
        A\triangle D &= (D\setminus A)\cup(A\setminus D) \\
        &\subseteq (D\setminus A)\cup (A\setminus (D\cup E))\cup E \\
        &\subseteq (A\triangle (D\cup E))\cup E.
    \end{align*}
    This together with the fact that $A\subseteq\bigcup_{i=1}^\infty B_i$ implies that
    \begin{align*}
        \mu(A\triangle D) &\leq \mu(A\triangle (D\cup E)) + \mu(E) \\
        &\leq  \mu\left(\bigcup_{i=1}^\infty B_i\right) - \mu(A) + \frac{\varepsilon}{2} \\
        &\leq \varepsilon.
    \end{align*}
    Now define $A_1=B_1$ and for each $i\geq 2$, $A_i = B_i\setminus \bigcup_{j=1}^{i=1} B_j$. By definition, $A_1,A_2\ldots$ are mutually disjoint. This proves the result.
    
\end{proof}

The following theorem allows us to ``extend" measures from a semiring to the $\sigma$-algebra generated by it. This allows us to define measures over an entire $\sigma$-algebra by defining its values over just a semiring that generates it.

\begin{ftheo}[Measure Extension Theorem]
\label{MeasureExtensionTh}
    Let $\mathcal{A}$ be a semiring and let $\mu:\mathcal{A}\to[0,\infty]$ be an additive, $\sigma$-subadditive and $\sigma$-finite set function with $\mu(\emptyset)=0$. Then there is a unique $\sigma$-finite measure $\tilde\mu:\sigma(\mathcal{A})\to[0,\infty]$ such that $\tilde\mu(A)=\mu(A)$ for all $A\in\mathcal{A}$.
\end{ftheo}
\begin{proof}
    Since $\mathcal{A}$ is a $\pi$-system, if such a $\tilde\mu$ exists, it is uniquely defined due to \cref{uniquely defined by base pi sys}.
    
    We shall explicitly construct a function that satisfies the given conditions. In order to do so, define as in \cref{set of countable coverings outer measure}
    $$\mu^*(A)=\inf\left\{\sum_{F\in\mathcal{F}}\mu(F) : \mathcal{F}\in\mathcal{U}(A) \right\}\text{ for any $A\subseteq\Omega$.}$$
    
    By \cref{set of countable coverings outer measure}, $\mu^*$ is an outer measure and $\mu^*(A)=\mu(A)$ for any $A\in\mathcal{A}$.
    
    \vspace{2mm}
    We first claim that $\mathcal{A}\subseteq\mathcal{M}(\mu^*)$.
    
    To prove this, let $A\in\mathcal{A}$ and $E\subseteq\Omega$ with $\mu^*(E)<\infty$. Fix some $\varepsilon>0$. Then by the definition of $\mu^*$, there exists a sequence $E_1,E_2,\ldots\in\mathcal{A}$ such that
    $$E\subseteq\bigcup_{i=1}^\infty E_i\text{ and }\sum_{i=1}^\infty \mu(E_i)\leq \mu^*(E)+\varepsilon.$$
    
    For each $n$, define $B_n=E_n\cap A$. Since $\mathcal{A}$ is a semiring, there exists for each $n$ some $m_n\in\mathbb{N}$ and mutually disjoint sets $C_{n, 1},C_{n, 2},\ldots,C_{n, m_n}$ such that
    $$E_n\setminus A = E_n\setminus B_n = \biguplus_{i=1}^{m_n}C_{n, i}$$
    Then we have that
    \begin{align*}
        E\cap A &\subseteq \bigcup_{n=1}^\infty B_n, \\
        E\cap A^c &\subseteq \bigcup_{n=1}^\infty\biguplus_{i=1}^{m_n}C_{n,i},\text{ and } \\
        E_n &= B_n\uplus\biguplus_{i=1}^{m_n}C_{n,i}.
    \end{align*}
    
    This implies that
    \begin{align*}
        \mu^*(E\cap A) + \mu^*(E\cap A^c) &\leq \sum_{n=1}^\infty\mu(B_n) + \sum_{n=1}^\infty\sum_{i=1}^{m_n}\mu(C_{n,i}) \quad\text{(since $\mu$ is $\sigma$-subadditive)} \\
        &= \sum_{n=1}^\infty\left(\mu(B_n) + \sum_{i=1}^{m_n}\mu(C_{n,i})\right) \\
        &= \sum_{n=1}^\infty \mu(E_n) \quad\text{(since $\mu$ is additive)} \\
        &   \leq \mu^*(E) + \varepsilon.
    \end{align*}
    
    \cref{mu measurable iff leq} implies that $A\in\mathcal{M}(\mu^*)$, that is, $\mathcal{A}\subseteq\mathcal{M}(\mu^*)$. This in turn in implies that $\sigma(\mathcal{A})\subseteq\mathcal{M}(\mu^*)$. Define the required function by $\tilde\mu:\sigma(\mathcal{A})\to[0,\infty]$, $A\mapsto\mu^*(A)$. By \cref{outer measure is sig additive on M}, $\tilde\mu$ is $\sigma$-additive. Since $\mu$ is $\sigma$-finite, $\tilde\mu$ is $\sigma$-finite as well. This proves the result.
\end{proof}

\subsection{Important Examples of Measures}

Now that we have the Measure Extension Theorem, we may introduce the Lebesgue-Stieltjes measure, a very useful measure on $(\mathbb{R},\mathcal{B}(\mathbb{R}))$, which is given as follows.

\begin{definition}[Lebesgue-Stieltjes Measure]
\label{defLebStielMeasure}
    Let $F:\mathbb{R}\to\mathbb{R}$ be monotone increasing and right continuous. The measure $\mu_F$ on $(\mathbb{R},\mathcal{B}(\mathbb{R}))$ defined by 
    $$\mu_F((a,b])= F(b)-F(a)\text{ for all $a,b\in\mathbb{R}$ such that $a<b$}$$
    is called the \textit{Lebesgue-Stieltjes measure} with distribution function $F$.
\end{definition}

The Lebesgue-Stieltjes measure is well-defined due to the Measure Extension Theorem \cref{MeasureExtensionTh}.

\vspace{1mm}
To see this more clearly, let $\mathcal{A}=\{(a,b\,]:a,b\in\mathbb{R}\text{ and }a\leq b\}$. It may be checked that $\mathcal{A}$ is a semiring. Further, $\sigma(\mathcal{A})=\mathcal{B}(\mathbb{R})$. Now, define the function $\tilde\mu_F:\mathcal{A}\to[0,\infty)$ by $(a,b]\mapsto F(b)-F(a)$. Clearly $\tilde\mu_F(\emptyset)=0$ and the function is additive. It remains to check that $\tilde\mu_F$ is $\sigma$-subadditive.

\vspace{1mm}
Let $(a,b],(a_1,b_1],(a_2,b_2],\ldots\in\mathcal{A}$ such that $(a,b\,]\subseteq\bigcup_{i=1}^\infty (a_i, b_i]$. Fix some $\varepsilon>0$ and choose $a_\varepsilon\in(a,b)$ such that
$$F(a_\varepsilon)-F(a)< \varepsilon/2\implies \tilde\mu_F((a,b]) - \tilde\mu_F((a_\varepsilon,b\,]) < \varepsilon/2.$$
It is possible to choose such an $\varepsilon$ due to the right continuity of $F$. Also, for any $k\in\mathbb{N}$, choose $b_{k, \varepsilon}$ such that $$F(b_{k,\varepsilon})-F(b_k)<\varepsilon 2^{-k-1}\implies \tilde\mu_F((a_k,b_{k,\varepsilon}]) - \tilde\mu_F((a_k,b_k]) < \varepsilon 2^{-k-1}.$$
We now have
$$[a_\varepsilon, b\,]\subseteq (a,b]\subseteq \bigcup_{i=1}^\infty (a_k,b_k]\subseteq \bigcup_{k=1}^\infty (a_k,b_{k,\varepsilon}]$$
Due to the compactness of $[a_\varepsilon,b]$, there then exists some $k_0\in\mathbb{N}$ such that
$$(a_{\varepsilon},b\,]\subseteq\bigcup_{k=1}^{k_0}(a_k,b_{k,\varepsilon}].$$
This implies that
\begin{align*}
    \tilde\mu_F((a,b\,]) &\leq \frac{\varepsilon}{2} + \tilde\mu_F((a,b]) \\
    &\leq \frac{\varepsilon}{2} + \sum_{k=1}^{k_0} \tilde\mu_F((a_k,b_{k,\varepsilon}]) \\
    &\leq \frac{\varepsilon}{2} + \sum_{k=1}^{k_0} \left(\tilde\mu_F((a_k,b_k]) + \varepsilon2^{-k-1}\right) \\
    &\leq \varepsilon + \sum_{k=1}^{\infty} \tilde\mu_F((a_k,b_k])
\end{align*}
As this is true for any choice of $\varepsilon$, $\tilde\mu_F$ is $\sigma$-subadditive.

\vspace{2mm}
Then the extension of $\tilde\mu_F$ uniquely to a $\sigma$-finite measure is guaranteed by \cref{MeasureExtensionTh}. This measure is known as the Lebesgue-Stieltjes measure.

\vspace{2mm}
The measure that results when the function $F$ is equal to the identity function is referred to the \textit{Lebesgue measure} on $\mathbb{R}^1$. Similar to this, we can define the Lebesgue measure in general as follows.

\begin{definition}[Lebesgue Measure]
    There exists a unique measure $\lambda^n$ on $(\mathbb{R}^n,\mathcal{B}(\mathbb{R}^n))$ such that for all $a,b\in\mathbb{R}^n$ with $a<b$,
    $$\lambda^n((a,b])=\prod_{i=1}^n (b_i-a_i).$$
    $\lambda^n$ is called the \textit{Lebesgue measure} on $(\mathbb{R}^n, \mathcal{B}(\mathbb{R}^n))$ or the \textit{Lebesgue-Borel measure}.
\end{definition}

\vspace{2mm}
Let $E$ be a finite nonempty set and $\Omega = E^\mathbb{N}$. If $\omega_1,\omega_2,\ldots,\omega_n\in E$, we define the following.
$$[\omega_1,\omega_2,\ldots,\omega_n]=\{\omega'\in\Omega : \omega'_i=\omega_i\text{ for }i\in[n]\}.$$
This represents the set of all sequences whose first $n$ elements are $\omega_1,\omega_2,\ldots,\omega_n$.

\begin{theorem}[Finite Products of Measures]
        Let $n\in\mathbb{N}$ and $\mu_1,\mu_2,\ldots,\mu_n$ be Lebesgue-Stieltjes measures on $(\mathbb{R},\mathcal{B}(\mathbb{R}))$. Then there exists a unique $\sigma$-finite measure $\mu$ on $(\mathbb{R}^n, \mathcal{B}(\mathbb{R}^n))$ such that for all $a,b\in\mathbb{R}^n$ with $a<b$,
        $$\mu((a,b])=\prod_{i=1}^n \mu_i((a_i,b_i])$$
        We call $\mu$ the \textit{product measure of $\mu_1,\mu_2,\ldots,\mu_n$} and denote it by $\bigotimes_{i=1}^n\mu_i$.
\end{theorem}

The proof of the above is similar to that of \cref{MeasureExtensionTh}. We choose intervals $(a,b_\varepsilon]$ and so on such that $\mu((a,b_\varepsilon])<\mu((a,b])+\varepsilon$. Such $b_\varepsilon$ exists due to the right continuity of each of the $F_i$s corresponding to each of the $\mu_i$s.

\clearpage
\section{Introduction to Probability}

\subsection{Basic Definitions}

\begin{definition}[Probability Measure]
    Let $\mathcal{A}$ be a $\sigma$-algebra and $\mu$ be a measure on $\mathcal{A}$. $\mu$ is called a \textit{probability measure} if $\mu(\Omega)=1$.
\end{definition}

Let $\Omega$ be a countable non-empty set and $\mathcal{A}=2^\Omega$. Further let $(p_\omega)_{\omega\in\Omega}$ be non-negative numbers. The map given by $A\mapsto\mu(A)=\sum_{\omega\in A}p_\omega$ defines a $\sigma$-finite measure on $2^\Omega$. $p=(p_\omega)_{\omega\in\Omega}$ is called the \textit{weight function} of $\mu$. $p_\omega$ is called the \textit{weight of $\mu$ at point $\omega$}.

In the case where $\sum_{\omega\in\Omega}p_\omega=1$, $\mu$ is a probability measure. Then the vector $(p_\omega)_{\omega\in\Omega}$ is called a \textit{probability vector}. 

\begin{definition}[Probability Distribution Function]
    A right continuous monotonically increasing function $F:\mathbb{R}\to[0,1]$ such that $\lim_{x\to-\infty}F(x)=0$ and $\lim_{x\to\infty}F(x)=1$ is called a \textit{(proper) probability distribution function}, often abbreviated as \textit{p.d.f.} If we instead have $\lim_{x\to\infty}F(x)\leq 1$, $F$ is called a \textit{(possibly) defective p.d.f.} If $\mu$ is a probability measure on $(\mathbb{R},\mathcal{B}(\mathbb{R}))$, then the function $F_\mu$ given by $x\mapsto \mu((\infty,x])$ is called the \textit{distribution function} of $\mu$.
\end{definition}

Note that a probability measure is uniquely determined by its distribution function.

\begin{definition}[Probability Space]
    Let $(\Omega,\mathcal{A},\mu)$ be a measure space. If in addition $\mu(\Omega)=1$, then $(\Omega,\mathcal{A},\mu)$ is called a \textit{probability space}.
\end{definition}

In the above definition, $\Omega$ is called the \textit{sample space}, $\mathcal{A}$ is called the \textit{event space} (and its elements are called \textit{events}, and $\mu$ is called the \textit{probability function}.

\vspace{2mm}
Let $\Omega$ be a finite nonempty set. Let $\mathcal{A}=2^\Omega$ and consider the function $\mu:\mathcal{A}\to[0,1]$ given by
$$\mu(A)=\frac{|A|}{|\Omega|}\text{ for each $A\subseteq\Omega$.}$$
This defines a probability measure on $\mathcal{A}$. This function $\mu$ is called the \textit{uniform distribution on $\Omega$} and is denoted $\mathcal{U}_\Omega$. The resulting probability space $(\Omega,\mathcal{A},\mathcal{U}_\Omega)$ is called a \textit{Laplace space}.

\vspace{2mm}
Another example is as follows. Let $\omega\in\Omega$ and $\delta_\omega(A)=\indic(\{\omega\})$. Then $\delta_\omega$ is a probability measure on any $\sigma$-algebra $\mathcal{A}\subseteq2^\Omega$. $\delta_\omega$ is called the \textit{Dirac measure} for the point $\omega$.

The Dirac measure is useful in constructing discrete probability distributions.

\vspace{2mm}
Consider the example of a coin toss. The sample space $\Omega$ has two elements, $\text{H}$ (for heads) and $\text{T}$ (for tails). The event space $\mathcal{A}$ then has four elements $\emptyset$, $\{H\}$, $\{T\}$, and $\{H,T\}$. Each of these events have associated probabilities $0,\frac{1}{2},\frac{1}{2}$, and $1$ respectively. Note that $\{H,T\}$ represents the event that either a heads or a tails occurs.`

\begin{definition}[Random Variable]
    Let $(\Omega,\mathcal{A},\textbf{P})$ be a probability space, $(\Omega',\mathcal{A}')$ a measurable space, and $X:\Omega\to\Omega'$ be measurable. Then
    \begin{enumerate}[(i)]
        \item $X$ is called a \textit{random variable} with values in $(\Omega',\mathcal{A}')$. If $(\Omega',\mathcal{A}')=(\mathbb{R},\mathcal{B}(\mathbb{R}))$, then $X$ is called a \textit{real random variable}.
        
        \item For $A'\in\mathcal{A}'$, we often denote
        $$\textbf{P}[X^{-1}(A')]\text{ as }\textbf{P}[X\in A']\text{ and } X^{-1}(A')\text{ as }\{X\in A'\}.$$
        In particular, we let $\{X\geq 0\}=X^{-1}([0,\infty))$ and define $\{X\leq b\}$ and other terms similarly.
    \end{enumerate}
\end{definition}

As we shall primarily deal with real random variables in our study of probability, we often drop the ``real" and refer to them as just random variables.

\begin{definition}
    Let $X$ be a random variable with underlying probability space $(\Omega,\mathcal{A},\textbf{P})$.
    \begin{enumerate}[(i)]
        \item The probability measure $\textbf{P}_X=\textbf{P}\circ X^{-1}$ is called the \textit{distribution} of $X$.
        \item For a real random variable $X$, the map $F_X$ given by $x\mapsto\textbf{P}[X\leq x]$ is called the \textit{distribution function} of $P_X$ (or $X$). If $\mu=\textbf{P}_X$, we write $X\sim\mu$ and say that $X$ has distribution $\mu$.
        \item A family $(X_i)_{i\in I}$ of random variables is called \textit{identically distributed} if $\textbf{P}_{X_i}=\textbf{P}_{X_j}$ for all $i,j\in I$. We write $X\iddistrib Y$ if $\textbf{P}_X=\textbf{P}_Y$ ($\mathcal{D}$ for \textit{distribution}).
    \end{enumerate}
\end{definition}

\begin{theorem}
    For any p.d.f. $F$, there exists a real random variable $X$ with $F_X=F$.
\end{theorem}

\begin{proof}
    We shall explicitly construct a probability space $(\Omega,\mathcal{A},\textbf{P})$ and random variable $X:\Omega\to\mathbb{R}$ such that $F_X=F$.
    
    \vspace{1mm}
    One choice that might come to mind is to take $(\Omega,\mathcal{A}) = (\mathbb{R},\mathcal{B}(\mathbb{R}))$, $X:\mathbb{R}\to\mathbb{R}$ as the identity function, and \textbf{$P$} the Lebesgue-Stieltjes measure with distribution function $F$.
    
    \vspace{1.5mm}
    While this choice of ours works, let us attempt to construct another more ``standard" choice that is perhaps more enlightening. Let $\Omega=(0,1),\mathcal{A}=\left.\mathcal{B}(\Omega)\right|_\Omega$ and $\textbf{P}$ be the Lebesgue measure on $(\Omega,\mathcal{A})$. This is standard in the sense that given any $F$, we construct a random variable over the same probability space. Define the left continuous inverse of $F$ as
    $$F^{-1}(t)=\inf\{x\in\mathbb{R}:F(x)\geq t\}\text{ for }t\in (0,1).$$
    Note that $F^{-1}(t)\leq x$ if and only if $F(x)\geq t$.
    In particular,
    $$\{t:F^{-1}(t)\leq x\}=(0,F(x)]\cap(0,1)$$
    and so $F^{-1}:(\Omega,\mathcal{A})\to(\mathbb{R},\mathcal{B}(\mathbb{R}))$ is measurable. Thus
    $$\textbf{P}\left[\{t:F^{-1}(t)\leq x\}\right]=F(x).$$
    This implies that $F^{-1}$ is the random variable we wish to construct.
\end{proof}

Note that the above implies that there is a bijection between probability distribution functions and distribution functions corresponding to random variables.

\begin{definition}
    If a distribution $F:\mathbb{R}^n\to[0,1]$ is of the form
    $$F(x)=\int_{-\infty}^{x_1}\d{t_1}\int_{-\infty}^{x_2}\d{t_2}\cdots\int_{-\infty}^{x_n}\d{t_n}\, f(t_1,t_2,\ldots,t_n)\text{ for }(x_1,x_2,\ldots,x_n)\in\mathbb{R}^n$$
    for some integrable function $f:\mathbb{R}^n\to[0,\infty)$, then $f$ is called the \textit{density of the distribution}.
\end{definition}

\subsection{Important Examples of Random Variables}

We now give several important examples of random variables that we shall encounter several times in our study of probability.

\begin{enumerate}
    \item \textit{Bernoulli Distribution.}
    
    Let $p\in[0,1]$ and $\textbf{P}[X=1]=p$, $P[X=0]=1-p$. Then $\textbf{P}_X$ is called the \textit{Bernoulli distribution with parameter $p$} and is denoted $\Ber_p$. More formally,
    $$\Ber_p=(1-p)\delta_0 + p\delta_1.$$
    Its distribution function is
    $$
    F_X(x) = 
    \begin{cases}
    0, &x<0 \\
    1-p, &x\in[0,1) \\
    1, &x\geq 1
    \end{cases}
    $$
    
    Note that the above can be likened to the outcome of a weighted coin, with heads and tails corresponding to $0$ and $1$.
    
    \vspace{2mm}
    The distribution $\textbf{P}_Y$ of $Y=2X-1$ is called the \textit{Rademacher distribution with parameter $p$}. More formally,
    $$\Rad_p = (1-p)\delta_{-1}+p\delta_1.$$
    $\Rad_{1/2}$ is simply called the Rademacher distribution.
    
    \item \textit{Binomial Distribution.}
    
    Let $p\in[0,1]$ and $n\in\mathbb{N}$. Let $X:\Omega\to\{0,1,2,\ldots,n\}$ be such that for each valid $k$,
    $$\textbf{P}[X=k]=\binom{n}{k}p^k(1-p)^{n-k}.$$
    Then $\textbf{P}_X$ is called the \textit{binomial distribution with parameters $n$ and $p$} and is denoted $b_{n,p}$. More formally,
    $$b_{n,p}=\sum_{k=0}^n \binom{n}{k}p^k(1-p)^{n-k}\delta_k.$$
    
    \item \textit{Geometric Distribution.}
    
    Let $p\in(0,1]$ and $X:\Omega\to\mathbb{N}_0$ be such that for each $n\in\mathbb{N}_0$,
    $$\textbf{P}[X=n]=p(1-p)^n.$$
    Then $\textbf{P}_X$ is called the \textit{geometric distribution with parameter $p$} and is denoted $\gamma_p$ or $b_{1,p}^{-}$. More formally,
    $$\gamma_p=\sum_{n=0}^\infty p(1-p)^n\delta_n.$$
    
    \item \textit{Negative Binomial Distribution.}
    
    Let $r>0$ and $p\in(0,1]$. We denote by
    $$b^{-}_{r,p}=\sum_{k=0}^\infty \binom{-r}{k}(-1)^kp^r(1-p)^k\delta_k$$
    the \textit{negative binomial distribution} or \textit{Pascal distribution} with parameters $r$ and $p$. Note that $r$ need not be an integer.
    
    \item \textit{Poisson Distribution.}
    
    Let $\lambda\in[0,\infty)$ and $X:\Omega\to\mathbb{N}_0$ be such that for each $n\in\mathbb{N}_0$,
    $$P[X=n]=e^{-\lambda}\frac{\lambda^n}{n!}.$$
    Then $\textbf{P}_x=\Poi_\lambda$ is called the \textit{Poisson distribution with parameter $\lambda$}.
    
    \item \textit{Hypergeoemetric Distribution.}
    
    Consider a basket with $B\in\mathbb{N}$ black balls and $W\in\mathbb{N}$ white balls. If we draw $n\in\mathbb{N}$ balls from the basket, some simple combinatorics shows that the probability of drawing (exactly) $b\in\{0,1,2,\ldots,n\}$ black balls is given by the \textit{hypergeometric distribution with parameters $B,W,n$}:
    $$\operatorname{Hyp}_{B,W;n}(\{b\})=\frac{\binom{B}{b}\binom{W}{n-b}}{\binom{B+W}{n}}.$$
    In general, if we have $k$ colors with $B_i$ balls of colour $i$ for each $i$, the probability of drawing exactly $b_i$ balls of colour $i$ for each $i$ is given by the \textit{generalised hypergeometric distribution}:
    $$\operatorname{Hyp}_{B_1,B_2,\ldots,B_k;n}(\left\{(b_1,b_2,\ldots,b_k)\right\}) = \frac{\binom{B_1}{b_1}\binom{B_2}{b_2}\cdots\binom{B_k}{b_k}}{\binom{B_1+B_2+\cdots+B_k}{n}}$$
    where $n=b_1+b_2+\cdots+b_k$.
    
    \item \textit{Gaussian Normal Distribution.}
    
    Let $\mu\in\mathbb{R}, \sigma^2>0$. Let $X$ be a real random variable such that for $x\in\mathbb{R}$,
    $$\textbf{P}[X\leq x]=\frac{1}{\sqrt{2\pi\sigma^2}}\int_{-\infty}^x \exp\left(-\frac{(t-\mu)^2}{2\sigma^2}\right)\d{t}$$
    Then $\textbf{P}_X$ is called the \textit{Gaussian normal distribution} (or just \textit{normal distribution}) \textit{with parameters $\mu$ and $\sigma^2$} and is denoted $\mathcal{N}_{\mu,\sigma^2}$. In particular, $\mathcal{N}_{0,1}$ is the standard normal distribution. 
    
    \item \textit{Exponential Distribution.}
    
    Let $\theta>0$ and $X$ be a nonnegative random variable such that for each $x\geq 0$,
    $$\textbf{P}[X\leq x]=\textbf{P}\left[X\in[0,x]\right]=\int_{0}^x\theta e^{-\theta t}\d{t}.$$
    Then $\textbf{P}_X$ is called the \textit{exponential distribution with parameter $\theta$} and is denoted $\exp_\theta$.
    
    % \item \textit{$d$-dimensional Normal Distribution.}
    
    % Let $\mu\in\mathbb{R}^d$ and $\Sigma$ be a positive definite symmetric $d\times d$ matrix. Let $X$ be an $\mathbb{R}^d$-valued random variable such that for each $x\in\mathbb{R}^d$,
    % $$\textbf{P}[X\leq x] = \det(2\pi\Sigma)^{-1/2}\int_{-\infty}^x \exp\left(-\frac{1}{2}\langle t-\mu, \Sigma^{-1}(t-\mu)\rangle\right)\lambda^d\d{t}$$
    % where $\langle\cdot,\cdot\rangle$ represents the standard inner product in $\mathbb{R}^d$. Then $\textbf{P}_X$ is called the \textit{$d$-dimensional normal distribution with parameters $\mu$ and $\Sigma$} and is denoted $\mathcal{N}_{\mu,\Sigma}$.
    
\end{enumerate}

\subsection{The Product Measure}

Let $E$ be a finite set and $\Omega=E^\mathbb{N}$. Let $(p_e)_{e\in E}$ be a probability vector. Define
$$\mathcal{A}=\{[\omega_1,\ldots,\omega_n]:\omega_1,\ldots,\omega_n\text{ and }n\in\mathbb{N}\}$$
and a content $\mu$ on $\mathcal{A}$ by
$$\mu([\omega_1,\omega_2,\ldots,\omega_n])=\prod_{i=1}^n p_{\omega_i}$$
We wish to extend $\mu$ to a measure on $\sigma(\mathcal{A})$. Similar to how we proved the existence of the Lebesgue-Stieltjes measure \ref{defLebStielMeasure}, we use a compactness argument to show that $\mu$ is $\sigma$-subadditive.

Let $A,A_1,A_2,\ldots\in\mathcal{A}$ such that $A\subseteq\bigcup_{i=1}^\infty A_i$. We claim that there exists $n\in\mathbb{N}$ such that $A\subseteq\bigcup_{i=1}^n A_i.$

For each $n\in\mathbb{N}$, let $B_n=A\setminus\bigcup_{i=1}^n A_i$. We assume that $B_n\neq\emptyset$ for all $n\in\mathbb{N}$ and prove the required by contradiction.

Due to the pigeonhole principle, there exists some $\omega_1\in E$ such that $[\omega_1]\cap B_n\neq\emptyset$ for infinitely many $n\in\mathbb{N}$. Since $B_1\supseteq B_2\supseteq\cdots$, we have that
$$[\omega_1]\cap B_n\neq\emptyset\text{ for all }n\in\mathbb{N}.$$
Similarly, there exist $\omega_2,\omega_3,\ldots\in E$ such that
$$[\omega_1,\ldots,\omega_k]\cap B_n\neq\emptyset\text{ for all }k,n\in\mathbb{N}.$$

Each $B_n$ is a disjoint union of sets $C_{n,1},\ldots,C_{n,m_n}\in\mathcal{A}$. Thus for each $n\in\mathbb{N}$, there is some $i_n\in\{1,2,\ldots,m_n\}$ such that
$$[\omega_1,\omega_2,\ldots,\omega_k]\cap C_{n,i_n}\neq\emptyset\text{ for infinitely many }k\in\mathbb{N}.$$
As $[\omega_1]\supseteq[\omega_1,\omega_2]\supseteq\cdots$, this implies that
$$[\omega_1,\omega_2,\ldots,\omega_k]\cap C_{n,i_n}\neq\emptyset\text{ for all } k\in\mathbb{N}$$

As $C_{n,i_n}\in\mathcal{A}$, for fixed $n$ and large $k$ ($k\geq m_n$), we have $$[\omega_1,\omega_2,\ldots,\omega_k]\subseteq C_{n,i_n}.$$
This implies that $\omega=(\omega_1,\omega_2,\ldots)\in C_{n,i_n}\subseteq B_n$. This in turn implies that $\bigcap_{i=0}^\infty B_i\neq\emptyset$, which yields a contradiction.

Therefore, $A\subseteq\bigcup_{i=1}^n A_n$ for some $n\in\mathbb{N}$. Since $\mu$ is known to be (finite) subadditive, we have
$$\mu(A)\leq \sum_{i=1}^n\mu(A_i)\leq \sum_{i=1}^\infty \mu(A_i),$$
which is the required result.

\begin{definition}[Product Measure]
\label{defProductMeasure}
    Let $E$ be a finite nonempty set and $\Omega=E^\mathbb{N}$. Let $(p_e)_{e\in E}$ be a probability vector. There then is a unique probability measure $\mu$ on $\sigma(\mathcal{A})=\mathcal{B}(\Omega)$ (where $\mathcal{A}$ is defined as above) such that
    $$\mu([\omega_1,\omega_2,\ldots,\omega_n])=\prod_{i=1}^n p_{\omega_i}\text{ for all $\omega_i\in E$ and $n\in\mathbb{N}$}.$$
    $\mu$ is called the \textit{product measure} or \textit{Bernoulli measure} on $\Omega$ with weights $(p_e)_{e\in E}$ and is denoted by $\left(\sum_{e\in E}p_e\delta_e\right)^{\otimes\mathbb{N}}$. The $\sigma$-algebra $\sigma(\mathcal{A})$ is called the \textit{product $\sigma$-algebra on $\Omega$} and is denoted by $(2^E)^{\otimes\mathbb{N}}$.
\end{definition}

We explain the above more intuitively in the following subsection.

\subsection{Independent Events}

In the following, let $(\Omega,\mathcal{A},\textbf{P})$ be a probability space and the sets $A\in\mathcal{A}$ be events. The $\Pr$ or $\textbf{P}$ symbol will denote the universal object of a probability measure (we used $\textbf{P}$ to denote this until now), and the probabilities $\Pr[\cdot]$ are always written in (square) brackets.
    
\begin{definition}
    Two events $A$ and $B$ are said to be \textit{independent} if
    $$\Pr[A\cap B] = \Pr[A]\Pr[B].$$
\end{definition}

An common example of independent events is that of rolling a die twice, where the outcome of the first roll is independent of the second.

Here, $\Omega=\{1,2,\ldots,6\}^2, \mathcal{A}=2^\Omega$ and the probability distribution is $\textbf{P}=\mathcal{U}_\Omega$. Our claim may be verified as follows. Let $\tilde A,\tilde B\subseteq\Omega, A=\tilde A\times\Omega$ and $B=\Omega\times\tilde B$. We must show that $\Pr[A]\Pr[B]=\Pr[A\cap B]$. This is obvious as follows:
\begin{align*}
    \Pr[A] &= \frac{|A|}{36}=\frac{|\tilde A|}{6} \\
    \Pr[B] &= \frac{|B|}{36}=\frac{|\tilde B|}{6} \\
    \Pr[A\cap B] &= \frac{|A\cap B|}{36} = \frac{|\tilde A||\tilde B|}{36} = \Pr[A]\Pr[B].
\end{align*}

While in the above example it is intuitively clear that the two events must be independent, we can have less obvious examples as well. For example, the event that the sum of the two rolls is odd and the event that the first roll gives at most a three are independent. We leave it to the reader to verify this claim.

\vspace{2mm}
We extend this definition of two independent events to any number of independent events as follows.

\begin{definition}[Independence of Events]
    Let $I$ be an index set and $(A_i)_{i\in I}$ be a family of events. The family $(A_i)_{i\in I}$ is called \textit{independent} if for any finite subset $J\subseteq I$, the following holds:
    $$\Pr\left[\bigcap_{j\in J}A_j\right]=\prod_{j\in J}\Pr[A_j].$$
\end{definition}

Let us now return to the product measure discussed in \ref{defProductMeasure}, which can be understood intuitively as follows. If $E$ is a finite set of outcomes, consider the probability space comprising $\Omega=E^\mathbb{N}$, the $\sigma$-algebra
$$\mathcal{A}=\sigma([\omega_1,\ldots,\omega_n]:\omega_1,\ldots,\omega_n\in E\text{ and }n\in\mathbb{N})$$
and the product measure $\textbf{P}=\left(\sum_{e\in E}p_e\delta_e\right)^{\otimes\mathbb{N}}$. This basically represents that we repeatedly conduct the experiment of choosing an outcome from $E$. Let $\tilde A_i\subseteq E$ for any $i\in\mathbb{N}$ and let $A_i$ be the event such that $\tilde A_i$ occurs in the $i$th experiment, given by
$$A_i = \{\omega\in\Omega:\omega_i\in\tilde A_i\}
  =\biguplus_{(\omega_1,\ldots,\omega_i)\in E^{i-1}\times \tilde A_i} [\omega_1,\ldots,\omega_i]$$

Intuitively, the family $(A_i)_{i\in\mathbb{N}}$ should be independent, since the outcome of one of the conducted experiments does not depend on the outcomes of the other experiments.

Let us check this. Let $J\subseteq\mathbb{N}$. For $j\in J$, let $B_j=A_j$ and $\tilde B_j=\tilde A_j$ and for $j\in\{1,2,\ldots,n\}\setminus J$, let $B_j=\Omega$ and $\tilde B_j=E$. Then
\begin{align*}
    \Pr\left[\bigcap_{j\in J}A_j\right] &= \Pr\left[\bigcap_{j=1}^n B_j\right] \\
    &= \Pr\left[\left\{\omega\in\Omega:\omega_j\in\tilde B_j\text{ for each }j\in\{1,2,\ldots,n\}\right\}\right] \\
    &= \sum_{e_1\in\tilde B_1} \cdots \sum_{e_n\in\tilde B_n} \prod_{j=1}^n p_{e_j} \\
    &= \prod_{j=1}^n \left(\sum_{e\in\tilde B_j}p_e\right) \\
    &= \prod_{j\in J} \left(\sum_{e\in\tilde A_j}p_e\right)
\end{align*}

As this is true in particular for $|J|=1$, we have for some fixed $i\in\{1,2,\ldots,n\}$,
$$\Pr[A_i] = \left(\sum_{e\in\tilde A_i}p_e\right).$$
Substituting the above, we have
$$
\Pr\left[\bigcap_{j\in J}A_j\right] 
= \prod_{j\in J} \left(\sum_{e\in\tilde A_j}p_e\right)
= \prod_{j\in J} \Pr[A_j].
$$
This proves the result.

\vspace{2mm}
Note that if events $A$ and $B$ are independent, then the events $A^c$ and $B$ are independent as well. This can be described more precisely as follows.

\begin{theorem}
    Let $I$ be an index set and $(A_i)_{i\in I}$ be a family of events. Define $B_i^0 = A_i$ and $B_i^1 = A_i^c$ for each $i\in I$. Then the following statements are equivalent.
    \begin{enumerate}[(a)]
        \item The family $(A_i)_{i\in I}$ is independent.
        \item There is some $\alpha\in\{0,1\}^I$ such that the family $(B_i^{\alpha_i})_{i\in I}$ is independent.
        \item For all $\alpha\in\{0,1\}^I$, the family $(B_i^{\alpha_i})_{i\in I}$ is independent.
    \end{enumerate}
\end{theorem}

We leave the proof of the above to the reader.

Now, recall the limes superior, defined in \ref{defLimes}. The limes superior represents the event that a particular event occurs an infinite amount of times. For example, if we roll a die an infinite number of times, we could consider the event that we roll a four an infinite number of times. This is formalized in the following.

\begin{theorem}[Borel-Cantelli Lemma]
\label{borelCantelliLemma}
    Let $A_1,A_2,\ldots$ be events and define $A^*=\limsup_{n\to\infty} A_n$. Then
    \begin{enumerate}[(a)]
        \item If $\sum_{n=1}^\infty \Pr[A_n]<\infty$, $\Pr[A^*]=0$.
        \item If $(A_n)_{n\in\mathbb{N}}$ is independent and $\sum_{n=1}^\infty \Pr[A_n]=\infty$, then $\Pr[A^*]=1$.
    \end{enumerate}
\end{theorem}
\begin{proof}
    By \ref{tripledoubleEquivalence}, $\textbf{P}$ is upper semicontinuous, lower semicontinuous and $\sigma$-subadditive.
    \begin{enumerate}
        \item As $\textbf{P}$ is upper semicontinuous and $\sigma$-subadditive,
        \begin{align*}
            \Pr[A^*] &= \lim_{n\to\infty}\Pr\left[\bigcup_{m=n}^\infty A_m\right] \\
            &\leq \lim_{n\to\infty} \Pr[A_m] = 0
        \end{align*}
        The result follows.
        
        \item As $\textbf{P}$ is lower semicontinuous and the family $(A_n)_{n\in\mathbb{N}}$ is independent, we have
        \begin{align*}
            \Pr[(A^*)^c] &= \Pr\left[\bigcup_{n=1}^\infty\bigcap_{m=n}^\infty A_m^c\right] \\
            &= \lim_{n\to\infty} \Pr\left[\bigcap_{m=n}^\infty A_m^c\right] \\
            &= \lim_{n\to\infty} \prod_{m=n}^\infty \left(1-\Pr[A_m]\right) \\
        \end{align*}
        Now for any $n\in\mathbb{N}$, as $\log(1-x)\leq -x$
        \begin{align*}
            \prod_{m=n}^\infty \left(1-\Pr[A_m]\right)&= \exp\left(\sum_{m=n}^\infty \log(1-\Pr[A_m])\right) \\
            &\leq \exp\left(-\sum_{m=n}^\infty \Pr[A_m]\right) = 0
        \end{align*}
        The result follows.
    \end{enumerate}
\end{proof}

The reader may verify using the Borel-Cantelli Lemma that if we roll a die an infinite number of times, the probability of rolling a four an infinite number of times is $1$ by considering the events $A_n = \{\omega\in\Omega:\omega_n=6\}$ for each $n\in\mathbb{N}$ where $\Omega=\{1,\ldots,6\}^\mathbb{N}$.

\vspace{2mm}
We now extend the definition of independence of a family of events as follows.

\begin{definition}[Independence of classes of events]
\label{independence of classes of events}
    Let $I$ be an index set and $\mathcal{E}_i\subseteq\mathcal{A}$ for all $i\in I$. The family $(\mathcal{E}_i)_{i\in I}$ is called \textit{independent} if for any finite $J\subseteq I$ and any choice of $E_j\in\mathcal{E}_j$ and $j\in J$, the family $(E_j)_{j\in J}$ is independent.
\end{definition}

For example, if we roll a die an infinite number of times, for each $i\in\mathbb{N}$, consider the class of events given by $\mathcal{E}_i=\{\{\omega\in\Omega:\omega_i\in A\}:A\subseteq\{1,\ldots,6\}\}$
where $\Omega=\{1,\ldots,6\}^\mathbb{N}$. Then the family $(\mathcal{E}_i)_{i\in I}$ is independent.

\begin{theorem}
~
    Let $I$ be an index set and for each $i\in I$, let $\mathcal{E}_i\subseteq \mathcal{A}$. Then
    \begin{enumerate}[(a)]
        \item Let $I$ be finite. If $\Omega\in\mathcal{E}_i$ for each $i$, then $(\mathcal{E}_i)_{i\in I}$ is independent if and only if $(E_i)_{i\in I}$ is independent for any choice of $E_i\in\mathcal{E}_i,i\in I$.
        
        \item If $(\mathcal{E}_i\cup\{\emptyset\})$ is $\cap$-closed for each $i$, then $(\mathcal{E}_i)_{i\in I}$ is independent if and only if $(\sigma(\mathcal{E}_i))_{i\in I}$ is independent.
        
        % \item Let $K$ be an arbitrary set and let $(I_k)_{k\in K}$ be mutually disjoint subsets of $I$. If $(\mathcal{E}_i)_{i\in I}$ is independent, then $\left(\bigcup_{i\in I_k}\mathcal{E}_i\right)_{k\in K}$ is also independent.
    \end{enumerate}
\end{theorem}

\begin{proof}
    ~
    \begin{enumerate}[(a)]
        \item The forward implication is obvious from the definition. To prove the backward implication, for $J\subseteq I$ and $j\in I\setminus J$, choose $E_j=\Omega$.
        
        \item The backward implication is obvious. Let us now prove the forward implication.
        
        First, we claim that for any $J\subseteq J'\subseteq I$ where $J$ is finite, 
        $$\Pr\left[\bigcap_{i\in J'}E_i\right] = \prod_{i\in J'}\Pr[E_i]$$
        for any choice of $E_i\in\sigma(\mathcal{E}_i)$ if $i\in J$ and
        $E_i\in\mathcal{E}_i$ if $i\in J'\setminus J$.
        
        We shall prove the above claim by induction on $|J|$. If $|J|=0$, then the claim is true as $(\mathcal{E}_i)_{i\in I}$ is independent. Now assume that the claim is true for all $J\subseteq I$ with $|J|=n$ and all finite $J'\supseteq J$. Fix such a $J$. Let $j\in I\setminus J$. Define $\tilde J=J\cup\{j\}$ and choose some $J'\supseteq\tilde J$. We shall show that the claim is true if we replace $J$ with $\tilde J$, thus proving the inductive step.
    
        Fix $E_i\in\sigma(\mathcal{E}_i)$ for each $i\in J$ and $E_i\in\mathcal{E}_i$ for each $i\in J'\setminus\tilde J$. Consider measures $\mu,\nu$ on $(\Omega,\mathcal{A})$ such that
        \begin{align*}
            \mu &: E_j\mapsto \Pr\left[\bigcap_{i\in J'}E_i\right] \\
            \nu &: E_j\mapsto \prod_{i\in J'}\Pr[E_i]
        \end{align*}
        By the induction hypothesis, $\mu(E_j)=\nu(E_j)$ for all $E_j\in\mathcal{E}_j\cup\{\emptyset,\Omega\}$. As $\mathcal{E}_j\cup\{\emptyset\}$ is $\cap$-closed, \ref{uniquely defined by base pi sys} implies that $\mu(E_j)=\nu(E_j)$ for all $E_j\in\sigma(\mathcal{E}_j)$.
        
        This proves our claim. Setting $J=J'$ yields the required result.
    
    \end{enumerate}
\end{proof}

\section{Independence}

\subsection{Independent Events}

In the following, let $(\Omega,\mathcal{A},\textbf{P})$ be a probability space and the sets $A\in\mathcal{A}$ be events.
    
\begin{definition}
    Two events $A$ and $B$ are said to be \textit{independent} if
    $$\Pr[A\cap B] = \Pr[A]\Pr[B].$$
\end{definition}

For example, if we roll a die twice, the event of rolling a $6$ the first time is independent of the event of rolling a $6$ the second time.

Here, $\Omega=[6]^2, \mathcal{A}=2^\Omega$ and the probability distribution is $\textbf{P}=\mathcal{U}_\Omega$. Our claim may be verified as follows. Towards showing that the outcome of the first roll is independent of that of the second, let $\tilde A,\tilde B\subseteq\Omega, A=\tilde A\times\Omega$ and $B=\Omega\times\tilde B$. We must show that $\Pr[A]\Pr[B]=\Pr[A\cap B]$. This is obvious as follows: 
\begin{align*}
    \Pr[A] &= \frac{|A|}{36}=\frac{|\tilde A|}{6} \\
    \Pr[B] &= \frac{|B|}{36}=\frac{|\tilde B|}{6} \\
    \Pr[A\cap B] &= \frac{|A\cap B|}{36} = \frac{|\tilde A||\tilde B|}{36} = \Pr[A]\Pr[B].
\end{align*}

While in the above example it is intuitively clear that the two events must be independent, we can have less obvious examples as well. For example, the event that the sum of the two rolls is odd and the event that the first roll gives at most a three are independent. We leave it to the reader to verify this claim.

\vspace{2mm}
We extend this definition of two independent events to any number of independent events as follows.

\begin{definition}[Independence of Events]
    Let $I$ be an index set and $(A_i)_{i\in I}$ be a family of events. The family $(A_i)_{i\in I}$ is called \textit{independent} if for any finite subset $J\subseteq I$, the following holds:
    $$\Pr\left[\bigcap_{j\in J}A_j\right]=\prod_{j\in J}\Pr[A_j].$$
\end{definition}

Note that pairwise independence does not guarantee overall independence. For example, For example, consider $(X,Y,Z)$ chosen uniformly from $\{(0,0,0),(1,1,0),(1,0,1),(0,1,1)\}$. Then $X,Y,Z$ are pairwise independent but they are not overall independent.

\vspace{2mm}
Let us now return to the product measure defined in \cref{defProductMeasure}, which can be understood intuitively as follows. If $E$ is a finite set of outcomes, consider the probability space comprising $\Omega=E^\mathbb{N}$, the $\sigma$-algebra
$$\mathcal{A}=\sigma([\omega_1,\ldots,\omega_n]:\omega_1,\ldots,\omega_n\in E\text{ and }n\in\mathbb{N})$$
and the product measure $\textbf{P}=\left(\sum_{e\in E}p_e\delta_e\right)^{\otimes\mathbb{N}}$. This basically represents that we repeatedly conduct the experiment of choosing an outcome from $E$. Let $\tilde A_i\subseteq E$ for each $i\in\mathbb{N}$ and let $A_i$ be the event such that $\tilde A_i$ occurs in the $i$th experiment, given by
$$A_i = \{\omega\in\Omega:\omega_i\in\tilde A_i\}
  =\biguplus_{(\omega_1,\ldots,\omega_i)\in E^{i-1}\times \tilde A_i} [\omega_1,\ldots,\omega_i]$$

Intuitively, the family $(A_i)_{i\in\mathbb{N}}$ should be independent, since the outcome of one of the conducted experiments does not depend on the outcomes of the other experiments.

We shall check this. Let $J\subseteq\mathbb{N}$ and For $j\in J$, let $B_j=A_j$ and $\tilde B_j=\tilde A_j$ and for $j\in[n]\setminus J$, let $B_j=\Omega$ and $\tilde B_j=E$. Then
\begin{align*}
    \Pr\left[\bigcap_{j\in J}A_j\right] &= \Pr\left[\bigcap_{j=1}^n B_j\right] \\
    &= \Pr\left[\left\{\omega\in\Omega:\omega_j\in\tilde B_j\text{ for each }j\in[n]\right\}\right] \\
    &= \sum_{e_1\in\tilde B_1} \cdots \sum_{e_n\in\tilde B_n} \prod_{j=1}^n p_{e_j} \\
    &= \prod_{j=1}^n \left(\sum_{e\in\tilde B_j}p_e\right) \\
    &= \prod_{j\in J} \left(\sum_{e\in\tilde A_j}p_e\right)
\end{align*}

In particular, as this is true for $|J|=1$, we have for some fixed $i\in[n]$,
$$\Pr[A_i] = \left(\sum_{e\in\tilde A_i}p_e\right).$$
Substituting the above, we have
$$
\Pr\left[\bigcap_{j\in J}A_j\right] 
= \prod_{j\in J} \left(\sum_{e\in\tilde A_j}p_e\right)
= \prod_{j\in J} \Pr[A_j].
$$
This proves the result. We state the result for future reference as follows.

\begin{theorem}
\label{product measure independence}
    Let $E$ be a finite set. Consider the probability space $(\Omega,\mathcal{A},\textbf{P})$ where $\Omega=E^\mathbb{N}$,
    $$\mathcal{A}=\sigma([\omega_1,\ldots,\omega_n]:\omega_1,\ldots,\omega_n\in E\text{ and }n\in\mathbb{N})$$
    and $\textbf{P}=\left(\sum_{e\in E}p_e\delta_e\right)^{\otimes\mathbb{N}}$. For each $i\in\mathbb{N}$, let $\tilde A_i\subseteq E$ and $A_i$ be the event such that $\tilde A_i$ occurs in the $i$th experiment, given by
    $$A_i = \{\omega\in\Omega:\omega_i\in\tilde A_i\}
    =\biguplus_{(\omega_1,\ldots,\omega_i)\in E^{i-1}\times \tilde A_i} [\omega_1,\ldots,\omega_i]$$
  Then the family $(A_i)_{i\in\mathbb{N}}$ is independent.
\end{theorem}


\vspace{2mm}
Note that if events $A$ and $B$ are independent, then the events $A^c$ and $B$ are independent as well. This can be described more precisely as follows.

\begin{theorem}
    Let $I$ be an index set and $(A_i)_{i\in I}$ be a family of events. Define $B_i^0 = A_i$ and $B_i^1 = A_i^c$ for each $i\in I$. Then the following statements are equivalent.
    \begin{enumerate}[(a)]
        \item The family $(A_i)_{i\in I}$ is independent.
        \item There is some $\alpha\in\{0,1\}^I$ such that the family $(B_i^{\alpha_i})_{i\in I}$ is independent.
        \item For all $\alpha\in\{0,1\}^I$, the family $(B_i^{\alpha_i})_{i\in I}$ is independent.
    \end{enumerate}
\end{theorem}

We leave the proof of the above to the reader.

Now, recall the limit superior defined in \cref{defLimes}. The limit superior represents the event that a particular event occurs an infinite amount of times (for example, the event that we roll a $4$ an infinite number of times when we roll a die a countably infinite number of times). This is formalized in the following.

\begin{ftheo}[Borel-Cantelli Lemma]
\label{borelCantelliLemma}
    Let $A_1,A_2,\ldots$ be events and let $A^*=\limsup_{n\to\infty} A_n$. Then
    \begin{enumerate}[(a)]
        \item If $\sum_{n=1}^\infty \Pr[A_n]<\infty$, $\Pr[A^*]=0$.
        \item If $(A_n)_{n\in\mathbb{N}}$ is independent and $\sum_{n=1}^\infty \Pr[A_n]=\infty$, then $\Pr[A^*]=1$.
    \end{enumerate}
\end{ftheo}
\begin{proof}
    By \cref{tripledoubleEquivalence}, $\textbf{P}$ is upper semicontinuous, lower semicontinuous and $\sigma$-subadditive.
    \begin{enumerate}[(a)]
        \item As $\textbf{P}$ is upper semicontinuous and $\sigma$-subadditive,
        \begin{align*}
            \Pr[A^*] &= \Pr\left[\bigcap_{n=1}^\infty\bigcup_{m=n}^\infty A_m\right] \\
            \lim_{n\to\infty}\Pr\left[\bigcup_{m=n}^\infty A_m\right] \\
            &\leq \lim_{n\to\infty} \sum_{m=n}^\infty \Pr[A_m] = 0
        \end{align*}
        The result follows.
        
        \item As $\textbf{P}$ is lower semicontinuous and the family $(A_n)_{n\in\mathbb{N}}$ is independent, we have
        \begin{align*}
            \Pr[(A^*)^c] &= \Pr\left[\bigcup_{n=1}^\infty\bigcap_{m=n}^\infty A_m^c\right] \\
            &= \lim_{n\to\infty} \Pr\left[\bigcap_{m=n}^\infty A_m^c\right] \\
            &= \lim_{n\to\infty} \prod_{m=n}^\infty \left(1-\Pr[A_m]\right) \\
        \end{align*}
        Now for any $n\in\mathbb{N}$, as $\log(1-x)\leq -x$
        \begin{align*}
            \prod_{m=n}^\infty \left(1-\Pr[A_m]\right)&= \exp\left(\sum_{m=n}^\infty \log(1-\Pr[A_m])\right) \\
            &\leq \exp\left(-\sum_{m=n}^\infty \Pr[A_m]\right) = 0
        \end{align*}
        The result follows.
    \end{enumerate}
\end{proof}

A saying the reader might have come across is that if a monkey is left with a typewriter for an infinite amount of time, it will eventually type the complete works of Shakespeare. This is in fact a consequence of the Borel-Cantelli Lemma, and is often referred to as the \href{https://en.wikipedia.org/wiki/Infinite_monkey_theorem}{Infinite Monkey Theorem}!

\vspace{2mm}
We now extend the definition of independence of a family of events as follows.

\begin{definition}[Independence of classes of events]
\label{independence of classes of events}
    Let $I$ be an index set and $\mathcal{E}_i\subseteq\mathcal{A}$ for all $i\in I$. The family $(\mathcal{E}_i)_{i\in I}$ is called \textit{independent} if for any finite $J\subseteq I$ and any choice of $j\in J$ and $E_j\in\mathcal{E}_j$, the family $(E_j)_{j\in J}$ is independent.
\end{definition}

For example, if we roll a die an infinite number of times, for each $i\in\mathbb{N}$, consider the class of events given by $\mathcal{E}_i=\{\{\omega\in\Omega:\omega_i\in A\}:A\subseteq[6]\}$
where $\Omega=[6]^\mathbb{N}$. Then the family $(\mathcal{E}_i)_{i\in I}$ is independent.

\begin{theorem}
\label{independent set classes subset}
~
    Let $I$ be an index set and for each $i\in I$, let $\mathcal{E}_i\subseteq \mathcal{A}$. Then
    \begin{enumerate}[(a)]
        \item Let $I$ be finite. If $\Omega\in\mathcal{E}_i$ for each $i$, then $(\mathcal{E}_i)_{i\in I}$ is independent if and only if $(E_i)_{i\in I}$ is independent for any choice of $E_i\in\mathcal{E}_i, i\in I$.
        
        \item If $(\mathcal{E}_i\cup\{\emptyset\})$ is $\cap$-closed for each $i$, then $(\mathcal{E}_i)_{i\in I}$ is independent if and only if $(\sigma(\mathcal{E}_i))_{i\in I}$ is independent.
        
        % \item Let $K$ be an arbitrary set and let $(I_k)_{k\in K}$ be mutually disjoint subsets of $I$. If $(\mathcal{E}_i)_{i\in I}$ is independent, then $\left(\bigcup_{i\in I_k}\mathcal{E}_i\right)_{k\in K}$ is also independent.
    \end{enumerate}
\end{theorem}

\begin{proof}
    ~
    \begin{enumerate}[(a)]
        \item The forward implication is obvious from the definition. To prove the backward implication, for $J\subseteq I$ and $j\in I\setminus J$, choose $E_j=\Omega$.
        
        \item The backward implication is obvious. Let us now prove the forward implication.
        
        First, we claim that for any $J\subseteq J'\subseteq I$ where $J$ is finite, 
        $$\Pr\left[\bigcap_{i\in J'}E_i\right] = \prod_{i\in J'}\Pr[E_i]$$
        for any choice of $E_i\in\sigma(\mathcal{E}_i)$ if $i\in J$ and
        $E_i\in\mathcal{E}_i$ if $i\in J'\setminus J$.
        
        We shall prove the above claim by induction on $|J|$. If $|J|=0$, then the claim is true as $(\mathcal{E}_i)_{i\in I}$ is independent. Now assume that the claim is true for all $J\subseteq I$ with $|J|=n$ and all finite $J'\supseteq J$. Fix such a $J$. Let $j\in I\setminus J$. Define $\tilde J=J\cup\{j\}$ and choose some $J'\supseteq\tilde J$. We shall show that the claim is true if we replace $J$ with $\tilde J$, thus proving the inductive step.
    
        Fix $E_i\in\sigma(\mathcal{E}_i)$ for each $i\in J$ and $E_i\in\mathcal{E}_i$ for each $i\in J'\setminus\tilde J$. Consider measures $\mu,\nu$ on $(\Omega,\mathcal{A})$ such that
        \begin{align*}
            \mu &: E_j\mapsto \Pr\left[\bigcap_{i\in J'}E_i\right] \\
            \nu &: E_j\mapsto \prod_{i\in J'}\Pr[E_i]
        \end{align*}
        By the induction hypothesis, $\mu(E_j)=\nu(E_j)$ for all $E_j\in\mathcal{E}_j\cup\{\emptyset,\Omega\}$. As $\mathcal{E}_j\cup\{\emptyset\}$ is $\cap$-closed, \cref{uniquely defined by base pi sys} implies that $\mu(E_j)=\nu(E_j)$ for all $E_j\in\sigma(\mathcal{E}_j)$.
        
        This proves our claim. Setting $J=J'$ yields the required result.
    
    \end{enumerate}
\end{proof}

\subsection{Independence of Random Variables}

Let $I$ be an index set and $(\Omega,\mathcal{A})$ be measurable space. For each $i\in I$, let $(\Omega_i,\mathcal{A}_i)$ be a measurable space and $X_i:(\Omega,\mathcal{A})\to(\Omega_i,\mathcal{A}_i)$ be a random variable with generated $\sigma$-algebra $\sigma(X_i)$.

\begin{definition}
    The family $(X_i)_{i\in I}$ of random variables is said to be independent if the family $(\sigma(X_i))_{i\in I}$ of generated $\sigma$-algebras is independent.
\end{definition}

We say that a family $(X_i)_{i\in I}$ of random variables is said to be i.i.d. (independent and identically distributed) if the family $(X_i)_{i\in I}$ is independent and $\textbf{P}_{X_i}=\textbf{P}_{X_j}$ for any $i,j\in I$.

\vspace{2mm}
The meaning of independence of random variables might be more clear from the following restructuring of the definition. A family $(X_i)_{i\in I}$ of random variables is independent if and only if for any finite set $J\subseteq I$ and any choice of $A_j\in\mathcal{A}_j,j\in J$, we have
$$\Pr\left[\bigcap_{j\in J}\{X_j\in A_j\}\right] = \prod_{j\in J}\Pr[X_j\in A_j].$$

\vspace{1mm}
For example, let us flip a coin four times. Let $X$ be the random variable be the number of heads that show up in the first two tosses and $Y$ be the random variable given by the number of tails that show up in the next two tosses. Then $X$ and $Y$ are independent.

\vspace{2mm}
For each $i\in I$, let $(\Omega_i',\mathcal{A}_i')$ be another measurable space and assume that $f_i:(\Omega,\mathcal{A})\to(\Omega_i',\mathcal{A}_i')$ is a measurable map. If $(X_i)_{i\in I}$ is independent, then $(f_i\circ X_i)_{i\in I}$ is independent as well. This is a consequence of the fact that $f_i\circ X_i$ is $\sigma(X_i)-\mathcal{A}_i'$-measurable.

\begin{theorem}
\label{independent if generators independent}
    For each $i\in I$, let $\mathcal{E}_i$ be a $\pi$-system that generates $\mathcal{A}$. If $(X^{-1}(\mathcal{E}_i))$ is independent, then $(X_i)_{i\in I}$ is independent.
\end{theorem}
\begin{proof}
    By \cref{generating pi system fixed under preimage}, $X^{-1}(\mathcal{E}_i)$ is a $\pi$-system that generates $X^{-1}(\mathcal{A}_i)=\sigma(X_i)$. The result follows from \cref{independent set classes subset}.
\end{proof}

\begin{theorem}
    Let $E$ be a finite set $(p_e)_{e\in E}$ be a probability vector on $E$. There then exists a probability space $(\Omega,\mathcal{A},\textbf{P})$ and an independent family $(X_n)_{n\in\mathbb{N}}$ of $E$-valued random variables on $(\Omega,\mathcal{A},\textbf{P})$ such that $\Pr[X_n = e] = p_e$ for each $e\in E$.
\end{theorem}
\begin{proof}
    We shall prove this by constructing the required probability space $(\Omega,\mathcal{A},\textbf{P})$. Let $\Omega=E^\mathbb{N}$ and $\mathcal{A}=\sigma\left(\{[\omega_1,\omega_2,\ldots,\omega_k]:\omega_i\in E\text{ for each }i\in[k]\text{ and }k\in\mathbb{N}\}\right)$. Let $\textbf{P}=\left(\sum_{e\in E}p_e\delta_e\right)^{\otimes\mathbb{N}}$ be the product measure. For each $n\in\mathbb{N}$, define the random variable $X_n:\Omega\to E$ by $\omega\mapsto \omega_n$, where $\omega_n$ represents the $n$th coordinate of $\omega$.
    
    As a consequence of \cref{product measure independence}, $(X_j)_{j\in\mathbb{N}}$ is independent. and the result is proved.
\end{proof}

\begin{definition}
    Let $I$ be an index set and for each $i\in I$, let $X_i$ be a random variable. For any $J\subseteq I$, let $F_{(X_j)_{j\in J}}:\mathbb{R}^J\to[0,1]$ be given by
    $$x\mapsto \Pr[X_j\leq x_j\text{ for each }j\in J]=\Pr\left[\bigcap_{j\in J} X_j\leq x_j\right].$$
    This function is called the \textit{joint distribution function of $(X_j)_{j\in J}$} and is denoted $F_J$. The probability measure $\textbf{P}_{(X_j)_{j\in J}}$ on $\mathbb{R}^J$ is called the \textit{joint distribution of $(X_j)_{j\in J}$}.
\end{definition}

\begin{theorem}
    A family $(X_i)_{i\in I}$ of random variables is independent if and only if for every finite $J\subseteq I$ and $x\in\mathbb{R}^J$,
    $$F_J(x) = \prod_{j\in J}F_{\{j\}}(x_j).$$
\end{theorem}
\begin{proof}
    The forward implication is obvious.
    
    The class of sets $\{(-\infty,a]:a\in\mathbb{R}\}$ is a $\cap$-closed generator of $\mathcal{B}(\mathbb{R})$. The given condition is equivalent to saying that the events $\{X_j\in(-\infty,x_j]\}$ are independent. By \cref{independent if generators independent}, the backward implication is proved.
\end{proof}

\begin{corollary}
    If in addition to the conditions of the previous theorem, each $F_J$ has a continuous density function $f_J=f_{(X_j)_{j\in J}}$, then the family $(X_i)_{i\in I}$ is independent if and only if for any finite $J\subseteq I$ and $x\in\mathbb{R}^J$,
    $$f_J(x) = \prod_{j\in J}f_{\{j\}}(x_j).$$
\end{corollary}

% \begin{theorem}
%     Let $K$ be an arbitrary set and $I_k,k\in K$ be mutually disjoint sets. Define $I=\bigcup_{k\in K}I_k$. If a family $(X_i)_{i\in I}$ of random variables is independent, then the family of $\sigma$-algebras $(\bigcup_{i\in I_k}\sigma(X_i))_{k\in K}$ is independent.
% \end{theorem}
% \begin{proof}
    
% \end{proof}

\subsection{The Convolution}

\begin{definition}
    Let $\mu$ and $\nu$ be probability measures on $(\mathbb{Z},2^\mathbb{Z})$. The \textit{convolution $(\mu * \nu)$} is defined as the probability measure on $(\mathbb{Z},2^\mathbb{Z})$ given by
    $$(\mu * \nu) (\{n\}) = \sum_{m=-\infty}^\infty \mu(\{m\})\nu(\{n-m\}).$$
\end{definition}

We define the $n$th convolution power by $\mu^{*1}=\mu$ and $\mu^{*n}=\mu^{*(n-1)}*\mu$.

\begin{theorem}
    If $X$ and $Y$ are independent $\mathbb{Z}$-valued random variables, then $\textbf{P}_{X+Y}=\textbf{P}_X * \textbf{P}_Y$.
\end{theorem}
\begin{proof}
    For any $n\in\mathbb{Z}$, we have
    \begin{align*}
        \textbf{P}_{X+Y}[\{n\}] &= \Pr[X+Y=n] \\
        &= \Pr\left[\biguplus_{m=-\infty}^\infty \{X=m\}\cap\{Y=n-m\}\right] \\
        &= \sum_{m=-\infty}^\infty \Pr[X=m]\Pr[Y=n-m] \\
        &= \sum_{m=-\infty}^\infty \textbf{P}_X[\{m\}]\textbf{P}_Y[\{n-m\}] = (\textbf{P}_X * \textbf{P}_Y) [\{n\}]
    \end{align*}
\end{proof}

Given the above theorem, we can generalise the convolution as follows.

\begin{definition}[Convolution of Probability Measures]
    Let $X$ and $Y$ be independent random variables on $\mathbb{R}^n$ such that $\mu=\textbf{P}_X$ and $\nu=\textbf{P}_Y$. The \textit{convolution $(\mu * \nu)$} is defined as $\textbf{P}_{X+Y}$.
\end{definition}

We define $\mu^{*k}$ for $k\in\mathbb{N}$ recursively similarly to the first case with $\mu^{*0}=\delta_0$.

\vspace{2mm}
For example, let $\lambda,\mu\in[0,\infty)$. Consider independent random variables $X,Y$ such that $X\sim\Poi_\mu$ and $Y\sim\Poi_\lambda$. Then for $n\in\mathbb{N}_0$
\begin{align*}
    \Pr[X+Y=n] &= e^{-\mu}e^{-\lambda}\sum_{m=0}^n \frac{\mu^m}{m!}\frac{\lambda^m}{(n-m)!} \\
    &= e^{-(\mu+\lambda)}\frac{(\mu+\lambda)^n}{n!}.
\end{align*}
Thus $\Poi_\lambda * \Poi_\mu = \Poi_{\lambda+\mu}$.

\subsection{Kolmogorov's \texorpdfstring{$0$}{TEXT}-\texorpdfstring{$1$}{TEXT} Law}

The \hyperref[borelCantelliLemma]{Borel-Cantelli Lemma} was an example of a so-called $0-1$ law. In this subsection, we study another such $0$-$1$ law.

\begin{definition}[Tail $\sigma$-algebra]
    Let $I$ be a countably infinite index set and $(\mathcal{A}_i)_{i\in I}$ be a family of $\sigma$-algebras. Then
    $$\mathcal{T}((\mathcal{A}_i)_{i\in I})=\bigcap_{\substack{J\subseteq I \\ |J|<\infty}} \sigma\left(\bigcup_{j\in I\setminus J}\mathcal{A}_j\right)$$
    is called the \textit{tail $\sigma$-algebra} of $(\mathcal{A}_i)_{i\in I}$. If $(A_i)_{i\in I}$ is a family of events, we define
    $$\mathcal{T}((A_i)_{i\in I}) = \mathcal{T}((\{\emptyset,A_i,A_i^c,\Omega\})_{i\in I}).$$
    If $(X_i)_{i\in I}$ is a family of random variables, we define
    $$\mathcal{T}((X_i)_{i\in I}) = \mathcal{T}((\sigma(X_i))_{i\in I}).$$
\end{definition}

If the meaning is clear from context, we represent the tail $\sigma$-algebra as just $\mathcal{T}$.

Intuitively, the above means that we consider those events that are independent of the values of any finite subfamily of $(X_i)_{i\in I}$. This shall be made clearer as follows.

\begin{theorem}
    Let $J_1,J_2,\ldots$ be finite sets with $J_n\uparrow I$. Then
    $$\mathcal{T}((\mathcal{A}_i)_{i\in I})=\bigcap_{n=1}^\infty \sigma\left(\bigcup_{m\in I\setminus J_n} \mathcal{A}_m\right).$$
    If $I=\mathbb{N}$, then this says that
    $$\mathcal{T}((\mathcal{A}_i)_{i\in\mathbb{N}}) = \bigcap_{n=1}^\infty \sigma\left(\bigcup_{m=n}^\infty \mathcal{A}_m\right).$$
\end{theorem}
\begin{proof}
    It is obvious from the definition that $\mathcal{T}((\mathcal{A}_i)_{i\in I})$ is a subset of the expression on the right.
    
    Let $J_n\uparrow I$ and $J\subseteq I$ be a finite set. There exists some $n_0\in\mathbb{N}$ such that $J\subseteq J_{n_0}$. We then have
    \begin{align*}
        \bigcap_{n=1}^\infty \sigma\left(\bigcup_{m\in I\setminus J_n} \mathcal{A}_m\right) &\subseteq \bigcap_{n=1}^N \sigma\left(\bigcup_{m\in I\setminus J_n} \mathcal{A}_m\right) \\
        &= \sigma\left(\bigcup_{m\in I\setminus J_N} \mathcal{A}_m\right) \\
        &\subseteq \sigma\left(\bigcup_{m\in I\setminus J} \mathcal{A}_m\right)
    \end{align*}
    Noting that the expression on the left does not depend on $J$ and taking the intersection over all $J$ implies the reverse inclusion and the result follows.
\end{proof}

If we interpret $I=\mathbb{N}$ as a set of times, the above theorem essentially says that any event in $\mathcal{T}$ is independent of the first finitely many time points.

\vspace{2mm}
Now, it is not immediately clear whether $\mathcal{T}$ even contains any nontrivial events (events other than $\emptyset$ and $\Omega$).

\vspace{2mm}
For starters, if $A_1,A_2,\ldots$ are events, then $A^*=\limsup_{n\to\infty} A_n$ and $A_*=\liminf_{n\to\infty} A_n$ are both in $\mathcal{T}((A_i)_{i\in \mathbb{N}})$.

\vspace{1mm}
To see this for $A_*$, define $B_n=\bigcap_{m=n}^\infty A_m$ for $n\in\mathbb{N}$. We then have that $B_n\uparrow A_*$ and $B_n\in\sigma(\bigcup_{m=n_0}^\infty A_m)$ for any $n\geq n_0$. This implies that $A_*\in\sigma(\bigcup_{m=n_0}^\infty A_m)$ for any $n_0\in\mathbb{N}$ and thus, $A_*\in\mathcal{T}$.

\vspace{1mm}
For $A^*$, we must show that for any $m\in\mathbb{N}$, $\limsup_{n\to\infty}A_n \in \sigma(\bigcup_{n=m}^\infty \sigma(A_n))$. This is true as $\limsup_{n\to\infty}A_n=\limsup_{n\to\infty}A_{n+m}$. Since each $A_{n+m}$ is in $\sigma(\bigcup_{n=m}^\infty \sigma(A_{n}))$, the statement is true.

\vspace{2mm}
Let $(X_n)_{n\in\mathbb{N}}$ be real random variables. Then the Ces\`{a}ro limits
$$\liminf_{n\to\infty} \frac{1}{n}\sum_{i=1}^n X_i\text{ and }\limsup_{n\to\infty} \frac{1}{n}\sum_{i=1}^n X_i$$
are $\mathcal{T}((X_n)_{n\in\mathbb{N}})$-measurable.

% To show this, choose some $n_0\in\mathbb{N}$ and note that
% $$X_*=\liminf_{n\to\infty}\frac{1}{n}\sum_{i=1}^n X_i=\liminf_{n\to\infty}\frac{1}{n}\sum_{i=n_0}^n X_i$$
% is $(\sigma((X_n)_{n\geq n_0}))$-measurable. 

\begin{ftheo}[Kolmogorov's $0$-$1$ Law]
    Let $I$ be a countable infinite index set and $((\mathcal{A}_i)_{i\in I})$ be an independent family of $\sigma$-algebras. Then the tail $\sigma$-algebra is $\textbf{P}$-trivial, that is,
    $$\Pr[A]\in\{0,1\}\text{ for any }A\in\mathcal{T}((\mathcal{A}_i)_{i\in I}).$$
\end{ftheo}
\begin{proof}
    Assume w.l.o.g. that $I=\mathbb{N}$. For each $i\in\mathbb{N}$, let
    $$\mathcal{F}_n = \left\{\bigcap_{i=1}^n A_k: A_j\in\mathcal{A}_j\text{ for each }j\in[n]\right\}.$$
    Let $\mathcal{F}=\bigcup_{i=1}^\infty \mathcal{F}_i$. Note that $\mathcal{F}$ is a semiring.
    
    \vspace{1mm}
    Further, for any $m\in\mathbb{N}$ and $A_m\in\mathcal{A}_m$, we have $A_m\in\mathcal{F}$. This implies that $\sigma\left(\bigcup_{i=1}^\infty \mathcal{A}_i\right)\subseteq\sigma(\mathcal{F})$. We also have
    $$\mathcal{F}_m\subseteq\sigma\left(\bigcup_{i=1}^m \mathcal{A}_i\right)\subseteq\sigma\left(\bigcup_{i=1}^\infty \mathcal{A}_i\right).$$
    This implies that $\mathcal{F} = \sigma\left(\bigcup_{i=1}^\infty \mathcal{A}_i\right)$.
    
    Let $A\in\mathcal{T}((\mathcal{A}_n)_{n\in\mathbb{N}})$ and $\varepsilon>0$. By \cref{Approximation Thm for Measures}, there exists $n_0\in\mathbb{N}$ and mutually disjoint sets $F_1,F_2,\ldots,F_{n_0}$ such that $$\Pr\left[A\triangle \bigcup_{i=1}^{n_0}F_i\right]<\varepsilon.$$
    Let $F=\bigcup_{i=1}^{n_0}F_i$. There must be some $n\in\mathbb{N}$ such that $F_1,\ldots,F_{n_0}\in \mathcal{F}_n$. This implies that $F\in\sigma(\bigcup_{i=1}^n \mathcal{A}_i)$. By the definition of the tail $\sigma$-algebra, $A\in\sigma(\bigcup_{i=n+1}^\infty\mathcal{A}_i)$ so $A$ must be independent of $F$. Therefore,
    \begin{align*}
        \varepsilon &> \Pr[A\setminus F] \\
        &= \Pr[A](1-\Pr[F]) \\
        &\geq \Pr[A](1-\Pr[A]-\varepsilon).
    \end{align*}
    As this is true for any $\varepsilon>0$, $0=\Pr[A](1-\Pr[A])$ and the result is proved.
\end{proof}

\begin{corollary}
    Let $(A_n)_{n\in\mathbb{N}}$ be an independent family of events. Then
    $$\Pr\left[\limsup_{n\to\infty} A_n\right]\text{ and }\Pr\left[\liminf_{n\to\infty} A_n\right]\text{ are in }\{0,1\}.$$
\end{corollary}

The above can be inferred from the fact that the $\limsup$ and $\liminf$ lie in the tail $\sigma$-algebra. It also follows from the \hyperref[borelCantelliLemma]{Borel-Cantelli Lemma}.

\begin{corollary}
    Let $(X_n)_{n\in\mathbb{N}}$ be an independent family of $\overline{\mathbb{R}}$-valued random variables. Then $X_*=\liminf_{n\to\infty}X_n$ and $X^*=\limsup_{n\to\infty} X_n$ are almost surely constant, that is, there exist $x_*,x^*\in\overline{\mathbb{R}}$ such that $\Pr[X_*=x_*]=1$ and $\Pr[X^*=x^*]=1$.
\end{corollary}

The above follows from the fact that for any $x\in\overline{\mathbb{R}}$, $\{X_*<x\}\in\mathcal{T}((X_n)_{n\in\mathbb{N}})$ and $\{X^*>x\}\in\mathcal{T}((X_n)_{n\in\mathbb{N}})$.

\clearpage
\section{Generating Functions}

It is a common theme in mathematics to determine relations between objects that are of interest and objects that are easy to compute with. In probability theory, probability generating functions, Laplace transforms and characteristic functions fall in the former category while the mean, median and variance of random variables fall in the latter.

\subsection{Definitions and Basics}

\begin{definition}[Probability Generating Function]
    Let $X$ be an $\mathbb{N}_0$-valued random variable. The \textit{probability generating function} (abbreviated \textit{pgf}) of $\textbf{P}_X$ (or $X$) is the map $\psi_{\textbf{P}_X}=\psi_X:[0,1]\to[0,1]$ is given by (where $0^0=1$)
    $$z\mapsto \sum_{n=0}^\infty \Pr[X=n]z^n.$$
\end{definition}

From the properties of a power series, we have the following result, which we do not prove.

\begin{theorem}
\label{pgfpowerseries}
    ~\begin{enumerate}[(a)]
        \item $\psi_X$ is continuous on $[0,1]$ and infinitely often continuously differentiable on (0,1). For $n\in\mathbb{N}$,
        $$\lim_{z\to 1^-}\psi_X^{(n)}(z)=\sum_{k=n}^\infty \Pr[X=k]\cdot k(k-1)\cdots (k-n+1)$$
        where both sides can equal $\infty$.
        
        \item $\textbf{P}_X$ is uniquely determined by $\psi_X$.
        
        \item For any $r\in(0,1)$, $\psi_X$ is uniquely determined by countably many values $\psi_X(x_i)$ where $x_i\in [0,r]$ for each $i\in\mathbb{N}$. If the series given in $\psi_X$ converges for some $z>1$, then this statement is also true for any $r\in(0,z)$ and 
        $$\lim_{z\to 1^{-}} \psi_{X}^{(n)}(z)=\psi_X^{(n)}(1)<\infty\text{ for }n\in\mathbb{N}.$$
        Here, $\psi_X$ is uniquely determined by $\psi_X^{(n)}(1), n\in\mathbb{N}$.
    \end{enumerate}
\end{theorem}

While this definition of a pgf may seem quite arbitrary, it is useful when we want to add random variables, as is evident from the following theorem.

\begin{theorem}
\label{sum of randvars is prod of pgfs}
    Let $X_1,X_2,\ldots,X_n$ be independent $\mathbb{N}_0$-valued random variables. Then
    $$\psi_{X_1+X_2+\ldots+X_n}=\prod_{i=1}^n \psi_{X_i}.$$
\end{theorem}
\begin{proof}
    We shall prove the claim for $n=2$ and the result will follow inductively. For any $z\in[0,1]$
    \begin{align*}
        \psi_{X_1}(z)\cdot\psi_{X_2}(z) &= \left(\sum_{n=0}^\infty \Pr[X_1=n]z^n\right)\left(\sum_{n=0}^\infty \Pr[X_2=n]z^n\right) \\
        &= \sum_{n=0}^\infty z^n\left(\sum_{m=0}^n\Pr[X_1=m]\Pr[X_2=n-m]\right) \\
        &= \sum_{n=0}^\infty z^n\left(\sum_{m=0}^n\Pr\left[\{X_1=m\}\cap \{X_2=n-m\}\right]\right) \\
        &= \sum_{n=0}^\infty z^n\Pr[X_1+X_2=n] \\
        &= \psi_{X_1+X_2}.
    \end{align*}
\end{proof}

For example,

\begin{itemize}
    \item For some $m,n\in\mathbb{N}$ and $p\in[0,1]$, let $X\sim b_{n,p}$ and $Y\sim b_{m,p}$. Then
    $$\psi_X(z)=\sum_{i=0}^n \binom{n}{i}p^i(1-p)^iz^i=(pz+1-p)^n.$$
    If $X$ and $Y$ are independent, then
    $$\psi_{X+Y}(z)=\psi_X(z)\cdot\psi_Y(z)=(pz+1-p)^{m+n}.$$
    Therefore, $X+Y\sim b_{m+n,p}$, which is not immediately apparent otherwise. (Note that this also implies that $b_{m,p}*b_{n,p}=b_{m+n,p}$)

    \item Let $p\in(0,1]$ and $X_1,\ldots,X_n\sim \gamma_p$ be independent random variables. Define $Y=X_1+\cdots+X_n$. For any $z\in[0,1]$,
    $$\psi_{X_1}(z)=\sum_{i=0}^\infty p(z(1-p))^i=\frac{p}{1-z(1-p)}.$$
    Then
    \begin{align*}
        \psi_Y(z) &= \frac{p^n}{(1-z(1-p))^n} \\
        &= \sum_{i=0}^\infty p^n \binom{-n}{i}(-1)^i(1-p)^i z^i \\
        &= \sum_{i=0}^\infty b^-_{n,p}(\{i\}) z^i.
    \end{align*}
    Therefore, $\gamma_p^{*n}=b^-_{n,p}$. Note that this matches with the intuition we introduced while defining the geometric and negative binomial distributions. The waiting time for the $n$th success is equivalent to waiting for a single success $n$ times, that is, $X_1+\cdots+X_n$.
    
    \item We can also show that for $\lambda,\mu\in[0,\infty)$,
    $$\psi_{\Poi_{\lambda}}(z)=e^{\lambda(z-1)}.$$
    This implies that $\Poi_\lambda * \Poi_{\mu} = \Poi_{\lambda+\mu}$ (Recall that we had also proved this earlier by manually calculating the convolution).
    
    \item For $r,s\in(0,\infty)$ and $p\in(0,1]$, we have $b^-_{r,p}*b^-_{s,p}=b^-_{r+s,p}$. We leave it to the reader to prove this.
\end{itemize}

\subsection{The Poisson Approximation}

\begin{theorem}
    Let $\mu,\mu_1,\mu_2,\ldots$ be probability measures on $(\mathbb{N}_0,2^{\mathbb{N}_0})$ with generating functions $\psi,\psi_1,\psi_2,\ldots$. Then the following are equivalent.
    \begin{enumerate}[(a)]
        \item $\lim_{n\to\infty}\mu_n(\{k\})=\mu(\{k\})$ for all $k\in\mathbb{N}_0$.
        \item $\lim_{n\to\infty}\mu_n(A)=\mu(A)$ for all $A\subseteq\mathbb{N}_0$.
        \item $\lim_{n\to\infty}\psi_n(z)=\psi(z)$ for all $z\in[0,1]$.
        \item $\lim_{n\to\infty}\psi_n(z)=\psi(z)$ for all $z\in[0,\eta)$ for some $\eta\in(0,1)$.
    \end{enumerate}
    If any of the above is true, we write $\lim_{n\to\infty}\mu_n=\mu$ and say that $(\mu_n)_{n\in\mathbb{N}}$ converges weakly to $\mu$.
\end{theorem}
\begin{proof}
~
    \begin{itemize}
        \item (a)$\implies$(b).
        
        Fix $\varepsilon>0$. Choose some $N\in\mathbb{N}$ such that
        $$\mu(\{N+1,N+2,\ldots\})<\frac{\varepsilon}{4}$$
        and sufficiently large $n_0\in\mathbb{N}$ such that
        $$\sum_{k=0}^N |\mu_n(\{k\})-\mu(\{k\})| < \frac{\varepsilon}{4}\text{ for all $n\geq n_0$.}$$
        Note that for any $n\geq n_0$, $\mu_n(\{N+1,N+2,\ldots\})<\frac\varepsilon2$. Therefore, for any $A\subseteq\mathbb{N}$ and $n\geq n_0$,
        \begin{align*}
            |\mu_n(A)-\mu(A)| &\leq \mu_n(\{N+1,N+2,\ldots\}) + \mu(\{N+1,N+2,\ldots\}) + \sum_{k\in A\cap\{0,1,\ldots,N\}} |\mu_n(\{k\})-\mu(\{k\})| \\
            &\leq \varepsilon.
        \end{align*}
        This proves the result.
        
        \item (b)$\implies$(a).
        
        This is trivial.
        
        \item (a)$\iff$(c)$\iff$(d)
        
        This follows from \cref{pgfpowerseries}.
    \end{itemize}
\end{proof}

When we are dealing with the binomial distribution for large $n$, it is usually inconvenient to calculate the terms. However, in the case where $n$ is very large and $p$ is very small such that $np=\lambda$ is of reasonable magnitude, then we can approximate $b_{n,p}$ by $\Poi_\lambda$. This is made rigorous as follows.

\vspace{2mm}
Let $(p_{n,k})_{n,k\in\mathbb{N}}$ such that $p_{n,k}\in[0,1]$. Let
$$\lambda=\lim_{n\to\infty}\sum_{k=1}^\infty p_{n,k}\in(0,\infty) \text{ and }\lim_{n\to\infty}\sum_{k=1}^\infty p_{n,k}^2 = 0.$$
An example of such a family would be $p_{n,k}=\lambda/n$ for $k\leq n$ and $p_{n,k}=0$ otherwise.

For each $n\in\mathbb{N}$, let $(X_{n,k})_{k\in\mathbb{N}}$ be a family of independent random variables such that $X_{n,k}\sim\Ber_{p_{n,k}}$. For $n,k\in\mathbb{N}$, define
$$S^n = \sum_{l=1}^\infty X_{n,l}\text{ and }S_k^n = \sum_{l=1}^k X_{n,l}.$$

\begin{theorem}[Poisson Approximation]
    With the above notation, the distributions $(\textbf{P}_{S^n})_{n\in\mathbb{N}}$ converge weakly to $\Poi_{\lambda}$.
\end{theorem}
\begin{proof}
    % We first claim that for any $z\in[0,1]$, $\psi_{S^n}(z)=\lim_{k\to\infty}\psi_k^n(z)$. Note that $S^n_k$ and $S^n-S^n_k$ are independent and as a result, $\psi_{S^n}=\psi_{S^n_k}\cdot\psi_{S^n-S^n_k}$. This implies that
    % \begin{align*}
    %     1 &\geq \frac{\psi_{S^n}}{\psi_{S^n_k}} \\
    %       &= \psi_{S^n-S^n_k} \\
    %       &\geq \Pr[S^n-S^n_k = 0] \\
    %       &= 1 - \Pr[S^n-S^n_k \geq 1] \\
    %       &\geq 1 - \sum_{l=k+1}^\infty p_{n,l}
    % \end{align*}
    For any $z\in[0,1]$,
    \begin{align*}
        \psi_{S^n}(z) &= \prod_{l=1}^\infty (p_{n,l}z + 1-p_{n,l}) \\
        &= \exp\left(\sum_{l=1}^\infty \log(p_{n,l}z + 1-p_{n,l})\right).
    \end{align*}
    Now, as $|\log(1-x)-x|\leq x^2$ for $|x|<\frac{1}{2}$, we have
    $$
        \left|\left(\sum_{l=1}^\infty \log(p_{n,l}z + 1-p_{n,l})\right) - \left((z-1)\sum_{l=1}^\infty p_{n,l}\right)\right| \leq \sum_{l=1}^\infty p_{n,l}^2.
    $$
    Taking the limit as $n\to\infty$, we have
    $$\lim_{n\to\infty} \psi_{S^n}(z) = \exp\left((z-1)\sum_{l=1}^\infty p_{n,l}\right) = e^{\lambda(z-1)}.$$
    As $\psi_{\Poi_{\lambda}} = e^{\lambda(z-1)}$, this completes the proof.
\end{proof}

For example, let $p_{n,k}$ be $\lambda/n$ if $k\leq n$ and $0$ otherwise. Then by the Poisson approximation, $\lim_{n\to\infty} b_{n,\frac{\lambda}{n}}=\Poi_{\lambda}$.

\subsection{Branching Processes}

Let $T,X_1,X_2,\ldots$ be $\mathbb{N}_0$-valued random variables and $S=\sum_{i=1}^T X_i$. Note that $S$ is measurable (and thus a random variable) as
$$\{S=k\}=\bigcup_{i=0}^\infty \{T=i\}\cap\{X_1+\cdots+X_n=k\}.$$

\begin{theorem}
\label{pgfcomposition}
    With the above notation, if $X_1,X_2,\ldots$ are all identically distributed, then $\psi_S=\psi_T\circ\psi_{X_1}$.
\end{theorem}
\begin{proof}
    We have
    \begin{align*}
        \psi_S(z) &= \sum_{k=0}^\infty\Pr[S=k]z^k \\
        &= \sum_{k=0}^\infty\sum_{i=0}^\infty \Pr[T=i]\Pr[X_1+\cdots+X_n=k]z^k \\
        &= \sum_{i=0}^\infty \Pr[T=i](\psi_{X_1}(z))^n \\
        &= (\psi_T\circ\psi_{X_1})(z).
    \end{align*}
\end{proof}

Let $p_0,p_1,\ldots\in[0,1]$ such that $\sum_{i=0}^\infty p_i = 1$. Let $(X_{n,i})_{n,i\in\mathbb{N}_0}$ be an independent family of random variables such that $\Pr[X_{n,i}=k]=p_k$ for any $k,n,i\in\mathbb{N}$.

\vspace{1mm}
Let $Z_0=1$ and $Z_n=\sum_{i=1}^{Z_{n-1}} X_{n,i}$ for each $n\in\mathbb{N}$. This can be interpreted as the number of members in each generation of a family where the number of offspring each person has is random and given by $X_{n,i}$.

\begin{definition}
    $(Z_n)_{n\in\mathbb{N}_0}$ is called a \textit{Galton-Watson process} or \textit{branching process} with offspring distribution $(p_k)_{k\in\mathbb{N}_0}$
\end{definition}

Branching processes are easier to study with the assistance of generator functions. For $z\in[0,1]$, let
$$\psi(z)=\sum_{k=0}^\infty p_k z^k.$$
Recursively define
$$\psi_1=\psi\text{ and }\psi_{n+1}=\psi_n\text{ for all }n\in\mathbb{N}.$$
\begin{theorem}
    $\psi_{Z_n}=\psi_n$ for all $n\in\mathbb{N}$.
\end{theorem}
\begin{proof}
    We have $\psi_{Z_1}=\psi_1$ (by definition). By \cref{pgfcomposition}, $\psi_{Z_{n+1}} = \psi\circ\psi_{Z_{n}}$. The result follows inductively.
\end{proof}

Now, let us study the probability that the family dies out.

Denote by $q_n=\Pr[Z_n=0]$ the probability that $Z$ is extinct by time $n$. Clearly, $q_n$ is monotone increasing in $n$. In the limiting case, we have the following.

\begin{definition}
    Let $(Z_n)_{n\in\mathbb{N}_0}$ be a branching process. We define its extinction probability by
    $$q=\lim_{n\to\infty}\Pr[Z_n=0].$$
\end{definition}

A natural question to ask is: under what condition does the family definitely die out, that is, $q=1$? We clearly have $q\geq p_0$ since $q_n$ is monotone increasing and $q=\lim_{n\to\infty}q_n$. If $p_0=0$, then $Z_n$ is monotone increasing in $n$ and as a result, $q=0$ as well.

\begin{theorem}[Extinction Probability of a Branching Process]
    Let $(Z_n)_{n\in\mathbb{N}_0}$ be a branching process with offspring distribution $(p_k)_{k\in\mathbb{N}_0}$ such that $p_1\neq 1$. Then
    \begin{enumerate}[(a)]
        \item $\{r\in[0,1]:\psi(r)=r\}=\{q,1\}$.
        \item The following holds:
        $$q<1\iff \lim_{z\to 1}\psi'(z)>1\iff \sum_{k=1}^\infty kp_k>1.$$
    \end{enumerate}
\end{theorem}
\begin{proof}
    ~
    \begin{enumerate}[(a)]
        \item Let $F=\{r\in[0,1]:\psi(r)=r\}$.
        %F for "Fixed"
        Clearly, $1\in F$ as $\psi(1)=1$. Note that $q_n=\psi_n(0)=\psi(q_{n-1})$ for all $n\in\mathbb{N}$. As $\psi$ is continuous,
        $$\psi(q)=\psi\left(\lim_{n\to\infty} q_n\right)=\lim_{n\to\infty}\psi(q_n)=\lim_{n\to\infty} q_{n+1}=q.$$
        Thus $q\in F$.
        
        We next claim that $q=\min F$. Let $r\in F$. Then $r\geq 0= q_0$. If $r\geq q_n$ for some $n\in\mathbb{N}$, then as $\psi$ is monotone increasing, $r=\psi(r)\geq\psi(q_n)=q_{n+1}$.
        
        Inductively, $r\geq q_n$ for every $n\in\mathbb{N}$. The claim follows. We complete the remainder of the proof in the second part.
        
        \item The second equivalence follows by \cref{pgfpowerseries}. For the first equivalence, consider the following two cases.
        \begin{itemize}
            \item $\lim_{z\to 1}\psi'(z)\leq 1$. Since $\psi$ is strictly convex, it follows that $\psi(z)>z$ for all $z\in[0,1)$ and so $F=\{1\}$. By the first part of the proof, $q=1$.
            
            \item $\lim_{z\to 1}\psi'(z)> 1$. Since $\psi$ is strictly convex and $\psi(0)\geq 0$, there is some unique $r\in(0,1)$ such that $\psi(r)=r$. Then $F=\{r,1\}$ and by the first part, $q=\min F=r<1$.
        \end{itemize}
    \end{enumerate}
    This completes the proof.
\end{proof}

\clearpage

\bibliographystyle{plain}   
\bibliography{references}

\end{document}