\documentclass{article}

\title{Probability and Measure Theory}
\author{Amit Rajaraman}
\date{Summer 2020}

\usepackage[utf8]{inputenc}
\usepackage{amsmath,amssymb,amsthm,amsfonts}
\usepackage{bbm}
\usepackage{enumerate}
\usepackage[tmargin=1in, bmargin=1in, lmargin=0.75in, rmargin=0.75in]{geometry}
\usepackage[colorlinks]{hyperref}
\usepackage{tikz}
\usepackage{cleveref}
\usetikzlibrary{arrows}
\usepackage{mathtools}
\usepackage{mdframed}
\usepackage{doi}
\usepackage[square, numbers]{natbib}


\usepackage{titlesec}
\titleformat{\section}[block]{\sffamily\Large\filcenter\bfseries}{\S\thesection.}{0.25cm}{\Large}
\titleformat{\subsection}[block]{\large\bfseries\sffamily}{\thesubsection.}{0.2cm}{\large}

\usepackage{fancyhdr}
\lhead{\sffamily{Probability Theory}}
\chead{\sffamily{\thepage}}
\rhead{\sffamily{-Amit Rajaraman}}
\cfoot{}
\pagestyle{fancy}

\setlength\parindent{0pt}

\newcommand{\Mod}[1]{\ (\mathrm{mod}\ #1)}
\newcommand{\indic}{\mathbbm{1}}
\newcommand{\iddistrib}{\stackrel{\mathcal{D}}{=}}

\renewcommand{\qedsymbol}{$\blacksquare$}
\renewcommand{\emptyset}{\varnothing}

\newcommand{\Ber}{\operatorname{Ber}}
\newcommand{\Rad}{\operatorname{Rad}}
\newcommand{\Poi}{\operatorname{Poi}}
\renewcommand{\d}[1]{\ensuremath{\operatorname{d}\!{#1}}}
\mathchardef\mhyphen="2D
\newcommand{\norm}[1]{\left\lVert#1\right\rVert}
\newcommand{\expec}{\textbf{E}}
\newcommand{\Var}{\operatorname{\textbf{Var}}}
\newcommand{\Cov}{\operatorname{\textbf{Cov}}}

\numberwithin{equation}{section}
\theoremstyle{definition}
\newtheorem{theorem}{Theorem}
\newenvironment{ftheo}
  {\begin{mdframed}\begin{theorem}}
  {\end{theorem}\end{mdframed}}
\newtheorem{lemma}[theorem]{Lemma}
\newtheorem{corollary}[theorem]{Corollary}
\newtheorem{definition}{Definition}
\newenvironment{fdef}
  {\begin{mdframed}\begin{definition}}
  {\end{definition}\end{mdframed}}
\numberwithin{definition}{section}
\numberwithin{theorem}{section}
\newtheorem{exercise}{Exercise}
\newtheorem*{example}{Example}

\theoremstyle{remark}
\numberwithin{exercise}{section}
\newtheorem*{solution}{Solution}

\begin{document}

\maketitle
\thispagestyle{empty}

\setcounter {section}{-1}

\begin{center}
    We primarily use ``Probability Theory: A Comprehensive Course"\cite{klenke} by Achim Klenke as reference for this course
\end{center}

\tableofcontents
\clearpage

\section{Notation}

$\mathbb{N}$ represents the set $\{1,2,\ldots\}$.

\vspace{1mm}
$\mathbb{N}_0$ represents the set $\{0,1,2,\ldots\}$.

\vspace{1mm}
For $x\in\mathbb{R}$ and $n\in\mathbb{N}$,
$$\binom{x}{r}=\frac{x(x-1)\cdots(x-r+1)}{r!}$$
is the generalised binomial coefficient.

\vspace{2mm}
For $n\in\mathbb{N}$, we denote $\{1,2,\ldots,n\}$ as $[n]$ and $\{0,1,2,\ldots,n\}$ as $[n]_0$

\vspace{2mm}
For $a\in\mathbb{R}^n$, we denote the $i$th coordinate of $a$ by $a_i$ for each $i=1,2,\ldots,n$.

\vspace{1mm}
For $a,b\in\mathbb{R}^n$, we write $a<b$ if $a_i<b_i$ for each $i=1,2,\ldots,n$.

\vspace{2mm}
Let $(a_n)_{n\in\mathbb{N}}$ be a sequence of reals. Then
\begin{align*}
    \limsup_{n\to\infty}a_n &= \lim_{n\to\infty}\left(\sup_{m\geq n}a_n\right) =    \inf_{n\to\infty}\left(\sup_{m\geq n}a_n\right) \\
    \liminf_{n\to\infty}a_n &= \lim_{n\to\infty}\left(\inf_{m\geq n}a_n\right) =    \sup_{n\to\infty}\left(\inf_{m\geq n}a_n\right)
\end{align*}

\vspace{2mm}
Let $A$ and $B$ be two sets. We denote by
$$A\triangle B=(A\setminus B)\cup(B\setminus A)$$
the \textit{symmetric difference} of $A$ and $B$.

\clearpage
\section{Measure Theory}

Before beginning a rigorous study of probability theory, it is necessary to understand some parts of basic measure theory.

\subsection{Classes of Sets}

\vspace{2mm}
Let $\Omega$ be a non-empty set and $\mathcal{A}\subseteq2^\Omega$, where $2^\Omega$ is the power set of $\Omega$. Then

\begin{definition}
    $\mathcal{A}$ is called
    \begin{itemize}
        \item $\cap$-closed (closed under intersections) or a $\pi$-system if $A\cap B\in \mathcal{A}$ for all $A,B\in\mathcal{A}$.
        \item $\sigma$-$\cap$-closed (closed under countable intersections) if $\bigcap_{i=1}^\infty A_i\in\mathcal{A}$ for any choice of countably many sets\\ $A_1, A_2, \ldots\in\mathcal{A}$.
        \item $\cup$-closed (closed under unions) if $A\cup B\in \mathcal{A}$ for all $A,B\in\mathcal{A}$.
        \item $\sigma$-$\cup$-closed (closed under countable unions) if $\bigcup_{i=1}^\infty A_i\in\mathcal{A}$ for any choice of countably many sets $A_1, A_2, \ldots\in\mathcal{A}$.
        \item $\setminus$-closed (closed under differences) if $A\setminus B\in\mathcal{A}$ for all $A,B\in\mathcal{A}$.
        \item closed under complements if $A^c=\Omega\setminus A\in\mathcal{A}$ for all $A\in\mathcal{A}$.
    \end{itemize}
\end{definition}

\begin{theorem}
\label{cupclosediffcapclosed}
    Let $\mathcal{A}$ be closed under complements. Then $\mathcal{A}$ is $\cup$-closed ($\sigma$-$\cup$-closed) if and only if $\mathcal{A}$ is closed $\cap$-closed ($\sigma$-$\cap$-closed).
\end{theorem}

The above is relatively straightforward to prove using De Morgan's Laws.

\begin{theorem}
\label{if setminus closed}
    Let $\mathcal{A}$ be $\setminus$-closed. Then
    \begin{enumerate}[(a)]
        \item $\mathcal{A}$ is $\cap$-closed,
        \item if $\mathcal{A}$ is $\sigma$-$\cup$-closed, then $\mathcal{A}$ is $\sigma$-$\cap$-closed.
        \item Any countable union of sets in $\mathcal{A}$ can be expressed as a countable union of pairwise disjoint sets in $\mathcal{A}$.
    \end{enumerate}
\end{theorem}
\begin{proof}
~
\begin{enumerate}[(a)]
    \item For $A,B\in\mathcal{A}$, $A\cap B=A\setminus(A\setminus B)\in\mathcal{A}$.
    \item Let $A_1,A_2,\ldots\in\mathcal{A}$. Then
    \begin{align*}
        \bigcap_{i=1}^\infty A_i &= \bigcap_{i=1}^\infty (A_1\cap A_i) \\
        &= \bigcap_{i=1}^\infty A_1\setminus (A_1\setminus A_i) \\
        &= A_1\setminus\bigcup_{i=1}^\infty (A_1\setminus A_i).
    \end{align*}
    \item Let $A_1,A_2,\ldots\in\mathcal{A}$. We then have
    $$\bigcup_{i=1}^\infty A_i = A_1 \uplus (A_2\setminus A_1)\uplus ((A_3\setminus A_2)\setminus A_1)\uplus\cdots$$
    The result follows.
    \end{enumerate}
\end{proof}

This equivalence between $\cap$ and $\cup$ if the class is $\setminus$-closed is apparent from De Morgan's laws.

\begin{definition}[Algebra]
\label{defAlgebra}
    A class of sets $\mathcal{A}\subseteq2^\Omega$ is called an \textit{algebra} if
    \begin{enumerate}[(i)]
        \item $\Omega\in\mathcal{A}$,
        \item $\mathcal{A}$ is $\setminus$-closed, and
        \item $\mathcal{A}$ is $\cup$-closed.
    \end{enumerate}
\end{definition}

\begin{definition}[$\sigma$-algebra]
\label{defSigAlgebra}
    A class of sets $\mathcal{A}\subseteq 2^\Omega$ is called a \textit{$\sigma$-algebra} if
    \begin{enumerate}[(i)]
        \item $\Omega\in\mathcal{A}$,
        \item $\mathcal{A}$ is closed under complements, and
        \item $\mathcal{A}$ is $\sigma$-$\cup$-closed.
    \end{enumerate}
\end{definition}

$\sigma$-algebras are also known as \textit{$\sigma$-fields}.

Note that any $\sigma$-algebra is an algebra (but the converse is not true).

\begin{theorem}
\label{algebra iff conditions}
    A class of sets $\mathcal{A}\subseteq2^\Omega$ is an algebra if and only if
    \begin{enumerate}[(a)]
        \item $\Omega\in\mathcal{A}$,
        \item $\mathcal{A}$ is closed under complements, and
        \item $\mathcal{A}$ is $\cap$-closed.
    \end{enumerate}
\end{theorem}

The proof of the above is left as an exercise to the reader.

\begin{definition}[Ring]
\label{defRing}
    A class of sets $\mathcal{A}\subseteq2^\Omega$ is called a \textit{ring} if
    \begin{enumerate}[(i)]
        \item $\emptyset\in\mathcal{A}$,
        \item $\mathcal{A}$ is $\setminus$-closed, and
        \item $\mathcal{A}$ is $\cup$-closed.
    \end{enumerate}
\end{definition}

Further, a ring is a \textit{$\sigma$-ring} if it is $\sigma$-$\cup$-closed.

\begin{definition}[Semiring]
\label{defSemiring}
    A class of sets $\mathcal{A}\subseteq2^\Omega$ is called a \textit{semiring} if
    \begin{enumerate}[(i)]
        \item $\emptyset\in\mathcal{A}$, 
        \item for any $A,B\in\mathcal{A}$, $A\setminus B$ is a finite union of mutually disjoint sets in $\mathcal{A}$, and
        \item $\mathcal{A}$ is $\cap$-closed.
    \end{enumerate}
\end{definition}

\begin{definition}[$\lambda$-system]
\label{defLamSystem}
    A class of sets $\mathcal{A}\subseteq2^\Omega$ is called a \textit{$\lambda$-system} (or \textit{Dynkin's $\lambda$-system}) if
    \begin{enumerate}[(i)]
        \item $\Omega\in\mathcal{A}$,
        \item for any $A,B\in\mathcal{A}$ with $B\subseteq A$, $A\setminus B\in\mathcal{A}$, and
        \item $\displaystyle\biguplus_{i=1}^\infty A_i\in\mathcal{A}$ for any choice of countably many pairwise disjoint sets $A_1,A_2,\ldots\in\mathcal{A}$.
    \end{enumerate}
\end{definition}

Among the above classes of sets, $\sigma$-algebras in particular are extremely important as we shall use them when defining probability.

\begin{theorem}
~
    \begin{enumerate}[(a)]
        \item Every $\sigma$-algebra is also a $\lambda$-system, an algebra and a $\sigma$-ring.
        \item Every $\sigma$-ring is a ring, and every ring is a semiring.
        \item Every algebra is a ring. An algebra on a finite set $\Omega$ is a $\sigma$-algebra.
    \end{enumerate}
\end{theorem}
\begin{proof}
~
    \begin{enumerate}[(a)]
        \item Let $\mathcal{A}$ be a $\sigma$-algebra. Then for any $A,B\in\mathcal{A}$, $A\setminus B = (A^c\cup B)^c \in \mathcal{A}$ and $A\cap B=(A^c\cup B^c)^c\in\mathcal{A}$, that is, $\mathcal{A}$ is $\setminus$-closed and $\cup$-closed. The result follows.
        
        \item Let $\mathcal{A}$ be a ring. Then \cref{cupclosediffcapclosed} implies that $\mathcal{A}$ is $\cap$-closed. The result follows.
        
        \item Let $\mathcal{A}$ be an algebra. With proof similar to the first part of this theorem, it is seen that $\mathcal{A}$ is $\setminus$-closed. We have $\emptyset=\Omega\setminus\Omega\in\mathcal{A}$ and thus, it is a ring. If $\Omega$ is finite, then $\mathcal{A}$ is finite. Thus any countable union of sets is a finite union of sets and the result follows.
        \end{enumerate}
\end{proof}

\begin{definition}
\label{defLimes}
    Let $A_1,A_2,\ldots$ be subsets of $\Omega$. Then
    $$\liminf_{n\to\infty}A_n:=\bigcup_{i=1}^\infty\bigcap_{j=i}^\infty A_j\text{ and }\limsup_{n\to\infty}A_n:=\bigcap_{i=1}^\infty\bigcup_{j=i}^\infty A_j$$
    are respectively called the \textit{limit inferior} and \textit{limit superior}, of the sequence $(A_n)_{n\in\mathbb{N}}$.
\end{definition}

The above may be rewritten as
\begin{align*}
    A_*:=\liminf_{n\to\infty}A_n &= \{\omega\in\Omega : |n\in\mathbb{N}:\omega\not\in A_n|<\infty\} \\
    A^*:=\limsup_{n\to\infty}A_n &= \{\omega\in\Omega : |n\in\mathbb{N}:\omega\in A_n|=\infty\}
\end{align*}

That is, $A_*$ represents the set of elements that do not appear in a finite number of sets and $A^*$ represents the set of elements that appear in an infinite number of sets. This implies that $A_*\subseteq A^*$. (Why is the opposite not necessarily true?)

\begin{definition}[Indicator function]
    Let $A$ be a subset of $\Omega$. The \textit{indicator function on $A$} is defined by
    $$\indic_A(x)
    =
    \begin{cases}
    1, & x\in A \\
    0, & x\not\in A
    \end{cases}
    $$
\end{definition}

With the above notation, it may be shown that
$$\indic_{A_*}=\liminf_{n\to\infty}\indic_{A_n}\text{ and }\indic_{A^*}=\limsup_{n\to\infty}\indic_{A_n}.$$

If $\mathcal{A}\subseteq2^\Omega$ is a $\sigma$-algebra and if $A_n\in\mathcal{A}$ for every $n\in\mathbb{N}$, then $A_*\in\mathcal{A}$ and $A^*\in\mathcal{A}$. This is clear from the fact that $\sigma$-algebras are closed under countable unions and intersections.

Proving the above statements is left as an exercise to the reader.

\begin{theorem}
\label{capofSigmaAisSigmaA}
    Let $I$ be some index set and $\mathcal{A}_i$ be a $\sigma$-algebra for each $i\in I$. Then the intersection $\mathcal{A}_I=\bigcap_{i\in I}\mathcal{A}_i$ is also a $\sigma$-algebra.
\end{theorem}
\begin{proof}
    We can prove this by using the three conditions in the definition of a $\sigma$-algebra.
    \begin{enumerate}[(i)]
        \item Since $\Omega\in \mathcal{A}_i$ for every $i\in I$, $\Omega\in \mathcal{A}_I$.
        \item Let $A\in \mathcal{A}_I$. Then $A\in \mathcal{A}_i$ for each $i\in I$ and thus $A^c\in \mathcal{A}_i$ for each $i\in I$. Therefore, $A^c\in \mathcal{A}_I$.
        \item Let $A_1,A_2,\ldots\in \mathcal{A}_I$. Then $A_n\in\mathcal{A}_i$ for each $n\in\mathbb{N}$ and $i\in I$. Thus $A=\bigcup_{n=1}^\infty A_n\in\mathcal{A}_i$ for each $i$ as well. The result follows.
    \end{enumerate}
\end{proof}

A similar statement holds for $\lambda$-systems.

\begin{theorem}
    Let $\mathcal{E}\subseteq2^\Omega$. Then there exists a smallest $\sigma$-algebra $\sigma(\mathcal{E})$ with $\mathcal{E}\subseteq\sigma(\mathcal{E})$:
    $$\sigma(\mathcal{E})=\bigcap_{\substack{\mathcal{A}\subseteq2^\Omega\text{ is a $\sigma$-algebra} \\ \mathcal{E}\subseteq\mathcal{A}}}\mathcal{A}.$$
    $\sigma(\mathcal{E})$ is called the \textit{$\sigma$-algebra generated by $\mathcal{E}$} and $\mathcal{E}$ is called a \textit{generator of $\sigma(\mathcal{E})$}. 
\end{theorem}
\begin{proof}
    $2^\Omega$ is a $\sigma$-algebra that contains $\mathcal{E}$ so the intersection is non-empty. By \cref{capofSigmaAisSigmaA}, $\sigma(\mathcal{E})$ is a $\sigma$-algebra.
\end{proof}

Similar to the above, $\delta(\mathcal{E})$ is defined as the $\lambda$-system generated by $\mathcal{E}$.

\vspace{2mm}
We always have the following:
\begin{enumerate}
    \item $\mathcal{E}\subseteq \sigma(\mathcal{E})$.
    \item If $\mathcal{E}_1\subseteq\mathcal{E}_2$, then $\sigma(\mathcal{E}_1)\subseteq\sigma(\mathcal{E}_2)$.
    \item $\mathcal{A}$ is a $\sigma$-algebra if and only if $\sigma(\mathcal{A})=\mathcal{A}$.
\end{enumerate}
Similar statements hold for $\lambda$-systems. Further, $\delta(\mathcal{E})\subseteq\sigma(\mathcal{E})$. This is to be expected as $\sigma$-algebras have more ``structure" than $\lambda$-systems.

\begin{theorem}[$\cap$-closed $\lambda$-system]
\label{cap closed lam sys}
    Let $\mathcal{D}\subseteq2^\Omega$ be a $\lambda$-system. Then $\mathcal{D}$ is a $\pi$-system if and only if $\mathcal{D}$ is a $\sigma$-algebra.
\end{theorem}
\begin{proof}
    If $\mathcal{D}$ is a $\sigma$-algebra, then it is obviously a $\pi$-system. Let $\mathcal{D}$ be a $\pi$-system. Then
    \begin{enumerate}[(a)]
        \item As $\mathcal{D}$ is a $\lambda$-system, $\Omega\in\mathcal{D}$.
        \item Let $A\in\mathbb{D}$. Since $\Omega\in\mathcal{D}$ and $\mathcal{D}$ is a $\lambda$-system, $A^c=\Omega\setminus A\in\mathcal{D}$.
        \item Let $A,B\in\mathcal{D}$. We have $A\cap B\in\mathcal{D}$. We now have $A\setminus B = A\setminus (A\cap B)\in\mathcal{D}$, that is, $\mathcal{D}$ is $\setminus$-closed.
        
        Let $A_1,A_2,\ldots\in\mathcal{D}$. Then by \cref{if setminus closed}, there exist $B_1,B_2,\ldots\in\mathcal{D}$ such that
        $$\bigcup_{i=1}^\infty A_i=\biguplus_{i=1}^\infty B_i\in\mathcal{D}.$$
    \end{enumerate}
    This completes the proof.
\end{proof}

\begin{theorem}[Dynkin's $\pi$-$\lambda$ theorem]
\label{dynkins pi lam theorem}
    If $\mathcal{E}\subseteq2^\Omega$ is a $\pi$-system, then $\delta(\mathcal{E})=\sigma(\mathcal{E})$.
\end{theorem}
\begin{proof}
    We already have $\delta(\mathcal{E})\subseteq\sigma(\mathcal{E})$. We must now prove the reverse inclusion. We shall show that $\delta(\mathcal{E})$ is a $\pi$-system.
    
    For each $E\in\delta(\mathcal{E})$, let
    $$\mathcal{D}_E=\{A\in\delta(\mathcal{E}) : A\cap E\in\delta(\mathcal{E})\}.$$
    
    To show that $\delta(\mathcal{E})$ is a $\pi$-system, it suffices to show that $\delta(\mathcal{E})\subseteq\mathcal{D}_E$ for all $E\in\delta(\mathcal{E})$. We shall first show that $\mathcal{D}_E$ is a $\lambda$-system for each $E\in\mathcal{E}$ by checking each of the conditions in \cref{defLamSystem}.
    \begin{enumerate}[(a)]
        \item We clearly have $\Omega\in\mathcal{D}_E$ as $\Omega\cap E=E$.
        \item For any $A,B\in\mathcal{D}_E$ with $A\subseteq B$, 
        $$(B\setminus A)\cap E = (B\cap E)\setminus(A\cap E)\in\delta(\mathcal{E}).$$
        \item Let $A_1,A_2,\ldots\in\mathcal{D}_E$ be mutually disjoint sets. Then
        $$\left(\biguplus_{i=1}^\infty A_i\right)\cap E = \biguplus_{i=1}^\infty\left(A_i\cap E\right)\in\delta(\mathcal{E}).$$
    \end{enumerate}
    Now since $\mathcal{D}_E$ is a $\lambda$-system and $\mathcal{E}\subseteq\mathcal{D}_E$ (Why?), $\delta(\mathcal{E})\subseteq\mathcal{D}_E$.
    
    Now that we have shown that $\delta(\mathcal{E})$ is a $\pi$-system, the result follows by \cref{cap closed lam sys}.
\end{proof}

\begin{definition}[Topology]
    Let $\Omega\neq\emptyset$ be an arbitrary set. A class of sets $\tau\subseteq2^\Omega$ is called a \textit{topology} on $2^\Omega$ if
    \begin{enumerate}[(i)]
        \item $\emptyset,\Omega\in\tau$,
        \item $\tau$ is $\cap$-closed, and
        \item for any $\mathcal{F}\subseteq\tau$, $\bigcup_{A\in\mathcal{F}}A\in\tau.$
    \end{enumerate}
\end{definition}

In the above case, the pair $(\Omega,\tau)$ is called a \textit{topological space}. The sets $A\in\tau$ are called \textit{open} and the sets $A\subseteq\Omega$ with $A^c\in\tau$ are called \textit{closed}.

\vspace{1mm}
Note that in contrast with $\sigma$-algebras, topologies are closed under only finite intersections but are also closed under arbitrary unions.

\vspace{2mm}
For example, consider the natural topology on $\mathbb{R}$ which consists of all open intervals in $\mathbb{R}$ and any arbitrary union of them.

\begin{definition}[Borel $\sigma$-algebra]
    Let $(\Omega,\tau)$ be a topological space. The $\sigma$-algebra
    $$\mathcal{B}(\Omega)=\mathcal{B}(\Omega,\tau)=\sigma(\tau)$$
    that is generated by the open sets is called the \textit{Borel $\sigma$-algebra on $\Omega$}. The elements $A\in\mathcal{B}(\Omega,\tau)$ are called \textit{Borel sets} or \textit{Borel measurable sets}.
\end{definition}


A Borel $\sigma$-algebra that we shall often encounter is $\mathcal{B}(\mathbb{R}^n)$ for $n\in\mathbb{N}$. Consider the following classes of sets:
\begin{align*}
    \mathcal{A}_1 &= \{A\subseteq\mathbb{R}^n:A\text{ is open}\} \\
    \mathcal{A}_2 &= \{A\subseteq\mathbb{R}^n:A\text{ is closed}\} \\
    \mathcal{A}_3 &= \{A\subseteq\mathbb{R}^n:A\text{ is compact}\} \\
    \mathcal{A}_4 &= \{(a,b):a,b\in\mathbb{Q}^n\text{ and }a<b\} \\
    \mathcal{A}_5 &= \{(a,b]:a,b\in\mathbb{Q}^n\text{ and }a<b\} \\
    \mathcal{A}_6 &= \{[a,b):a,b\in\mathbb{Q}^n\text{ and }a<b\} \\
    \mathcal{A}_7 &= \{[a,b]:a,b\in\mathbb{Q}^n\text{ and }a<b\} \\
    \mathcal{A}_8 &= \{(-\infty,b):b\in\mathbb{Q}^n\} \\
    \mathcal{A}_9 &= \{(-\infty,b]:b\in\mathbb{Q}^n\} \\
    \mathcal{A}_{10} &= \{(a,\infty):a\in\mathbb{Q}^n\} \\
    \mathcal{A}_{11} &= \{[a,\infty):a\in\mathbb{Q}^n\} \\
\end{align*}
It may be proved that $\mathcal{B}(\mathbb{R}^n)$ is generated by any of the classes of sets $\mathcal{A}_1,\mathcal{A}_2,\ldots,\mathcal{A}_{11}$.

\vspace{2mm}
For $A\in\mathcal{B}(\mathbb{R})$, we represent by $\left.\mathcal{B}(\mathbb{R})\right|_{A}$ the restriction of $\mathcal{B}(\mathbb{R})$ to $A$. It may be proved that this is equal to $\mathcal{B}(A)$, the $\sigma$-algebra generated by the open subsets of $A$.

\subsection{Measure}

\begin{definition}
    Let $\mathcal{A}\subseteq2^\Omega$ and let $\mu:\mathcal{A}\to[0,\infty]$ be a set function. We say that $\mu$ is
    \begin{enumerate}[(i)]
        \item \textit{monotone} if for any $A,B\in\mathcal{A}$, $A\subseteq B$ implies that $\mu(A)\leq\mu(B)$,
        
        \item \textit{additive} if for any choice of finitely many mutually disjoint sets $A_1,\ldots,A_n\in\mathcal{A}$ with $\biguplus_{i=1}^n A_i\in\mathcal{A}$,
        $$\mu\left(\biguplus_{i=1}^nA_i\right)=\sum_{i=1}^n\mu(A_i),$$
        
        \item $\sigma$-additive if for any choice of countably many mutually disjoint sets $A_1,A_2,\ldots\in\mathcal{A}$ with $\biguplus_{i=1}^\infty A_i\in\mathcal{A}$,
        $$\mu\left(\biguplus_{i=1}^\infty A_i\right)=\sum_{i=1}^\infty\mu(A_i),$$
        
        \item subadditive if for any choice of finitely many sets $A,A_1,A_2,\ldots,A_n\in\mathcal{A}$ with $A\subseteq\bigcup_{i=1}^nA_i$, we have $$\mu(A)\leq\sum_{i=1}^n\mu(A_i),\text{ and}$$
        
        \item $\sigma$-subadditive if for any choice of countably many sets $A,A_1,A_2,\ldots\in\mathcal{A}$ with $A\subseteq\bigcup_{i=1}^\infty A_i$, we have
        $$\mu(A)\leq\sum_{i=1}^\infty \mu(A_i).$$
    \end{enumerate}
\end{definition}

\begin{definition}
\label{measureDef}
    Let $\mathcal{A}$ be a semiring and $\mu:\mathcal{A}\to[0,\infty]$ be a set function with $\mu(\emptyset)=0$. $\mu$ is called a
    \begin{enumerate}[(i)]
        \item \textit{content} if $\mu$ is additive,
        \item \textit{premeasure} if $\mu$ is $\sigma$-additive, and
        \item \textit{measure} if $\mu$ is $\sigma$-additive and $\mathcal{A}$ is a $\sigma$-algebra.
    \end{enumerate}
\end{definition}

\begin{theorem}[Properties of contents]
\label{properties of content}
    Let $\mathcal{A}$ be a semiring and $\mu$ be a content on $\mathcal{A}$. Then
    \begin{enumerate}[(a)]
        \item If $\mathcal{A}$ is a ring, then $\mu(A\cup B)+\mu(A\cap B)=\mu(A)+\mu(B)$ for any $A,B\in\mathcal{A}$.
        
        \item $\mu$ is monotone. If $\mathcal{A}$ is a ring, then $\mu(B)=\mu(A)+\mu(B\setminus A)$ for any $A,B\in\mathcal{A}$ with $A\subseteq B$.
        
        \item $\mu$ is subadditive. If $\mu$ is $\sigma$-additive, then it is also $\sigma$-subadditive.
        
        \item If $\mathcal{A}$ is a ring, then $$\sum_{n=1}^\infty \mu(A_n)\leq\mu\left(\bigcup_{n=1}^\infty A_n\right)$$ for any choice of countably many mutually disjoint sets $A_1,A_2,\ldots\in\mathcal{A}$ with $\bigcup_{i=1}^\infty A_i\in\mathcal{A}$.
    \end{enumerate}
\end{theorem}
\begin{proof}
~
    \begin{enumerate}[(a)]
        \item Note that $A\cup B=A\uplus (B\setminus A)$ and $B=(A\cap B)\uplus (B\setminus A)$. As $\mu$ is additive,
        $$\mu(A\cup B)=\mu(A)+\mu(B\setminus A)\text{ and }\mu(B)=\mu(A\cap B)+\mu(B\setminus A).$$
        The result follows.
        
        \item Let $A\subseteq B$. If $B\setminus A\in\mathcal{A}$ (which is true in the case of a ring), we have $B=A\uplus (B\setminus A)$ and thus
        $$\mu(B)=\mu(A)+\mu(B\setminus A).$$
        If $\mathcal{A}$ is just a semiring, then there exist $n\in\mathbb{N}$ and mutually disjoint sets $C_1,C_2,\ldots,C_n\in\mathcal{A}$ such that $$B\setminus A=\biguplus_{i=1}^n C_i.$$
        In either case, we have $\mu(A)\leq \mu(B)$.
        
        \item Let $A, A_1,A_2,\ldots,A_n\in\mathcal{A}$ such that $A\subseteq\bigcup_{i=1}^n A_i$. Let $B_1=A_1$ and for each $k=2,3,\ldots,n$, let
        $$B_k = A_k\setminus\left(\bigcup_{i=1}^{k-1}A_i\right).$$
        Note that any two $B_i$s are disjoint. As $\mu$ is additive and monotone, we have
        \begin{align*}
            \mu(A) &\leq \mu\left(\bigcup_{i=1}^n A_i\right) \\
            &= \mu\left(\biguplus_{i=1}^n B_i\right) \\
            &= \sum_{i=1}^n \mu(B_i) \leq \sum_{i=1}^n \mu(A_i).
        \end{align*}
        We can similarly prove that if $\mu$ is $\sigma$-additive, then it is $\sigma$-subadditive.
        
        \item Let $A=\bigcup_{i=1}^\infty A_i\in\mathcal{A}$. Since $\mu$ is additive and monotone,
        $$\sum_{i=1}^m\mu(A_i)=\mu\left(\biguplus_{i=1}^m A_i\right)\leq \mu(A)\text{ for any $m\in\mathbb{N}$.}$$
        The result follows.
    \end{enumerate}
\end{proof}

Note that if equality holds in the fourth part of the above theorem, $\mu$ is a premeasure.

\begin{definition}[Finite content]
    Let $\mathcal{A}$ be a semiring. A content $\mu$ on $A$ is called
    \begin{enumerate}[(i)]
        \item \textit{finite} if $\mu(A)<\infty$ for all $A\in\mathcal{A}$ and
        \item \textit{$\sigma$-finite} if there exists a sequence of sets $\Omega_1,\Omega_2,\ldots\in\mathcal{A}$ such that $\Omega=\bigcup_{i=1}^\infty\Omega_i$ and $\mu(\Omega_i)<\infty$ for every $i\in\mathbb{N}$.
    \end{enumerate}
\end{definition}

\begin{definition}
    Let $A,A_1,A_2,\ldots$ be sets. We write
    \begin{enumerate}[(i)]
        \item $A_n\uparrow A$ if $A_1\subseteq A_2\subseteq A_3\subseteq\cdots$ and $\bigcup_{i=1}^\infty A_i=A$. In this case, we say that $A_n$ increases to $A$.
        \item $A_n\downarrow A$ if $A_1\supseteq A_2\supseteq A_3\supseteq\cdots$ and $\bigcap_{i=1}^\infty A_i=A$. In this case, we say that $A_n$ decreases to $A$.
    \end{enumerate}
\end{definition}

For example, if $A_n=\left(-\frac{1}{n},\frac{1}{n}\right)$ for $n\in\mathbb{N}$, then $A_n\downarrow \{0\}$.

\begin{definition}[Continuity of contents]
    Let $\mu$ be a content on the ring $\mathcal{A}$. $\mu$ is called
    \begin{enumerate}[(i)]
        \item \textit{lower semicontinuous} if $\lim_{n\to\infty}\mu(A_n)=\mu(A)$ for any $A\in\mathcal{A}$ and sequence $(A_n)_{n\in\mathbb{N}}$ such that $A_n\uparrow A$,
        \item \textit{upper semicontinuous} if $\lim_{n\to\infty}\mu(A_n)=\mu(A)$ for any $A\in\mathcal{A}$ and sequence $(A_n)_{n\in\mathbb{N}}$ such that $\mu(A_n)<\infty$ for some $n$ (this implies that it holds for all $n\in\mathbb{N}$) and $A_n\downarrow A$,
        \item \textit{$\emptyset$-continuous} if (ii) holds for $A=\emptyset$.
    \end{enumerate}
\end{definition}

\begin{theorem}
\label{tripledoubleEquivalence}
    Let $\mu$ be a content on the ring $\mathcal{A}$. The following properties are equivalent:
    \begin{enumerate}[(a)]
        \item $\mu$ is $\sigma$-additive (and hence a premeasure).
        \item $\mu$ is $\sigma$-subadditive.
        \item $\mu$ is lower semicontinuous.
        \item $\mu$ is $\emptyset$-continuous.
        \item $\mu$ is upper semicontinuous.
    \end{enumerate}
    Then (a)$\iff$(b)$\iff$(c)$\implies$(d)$\iff$(e).
    If $\mu$ is finite, then all five statements are equivalent.
\end{theorem}
\begin{proof}
~
    \begin{itemize}
        \item (a)$\implies$(b) ($\sigma$-additivity implies $\sigma$-subadditivity).
        
        This follows from \cref{properties of content}(c).
        
        \item (b)$\implies$(a) ($\sigma$-subadditivity implies $\sigma$-additivity).
        
        This follows from \cref{properties of content}(d).
        
        \item (a)$\implies$(c) ($\sigma$-additivity implies lower semicontinuity).
        
        Let $\mu$ be a premeasure and $A\in\mathcal{A}$. Let $A_1,A_2,\ldots\in\mathcal{A}$ such that $A_n\uparrow A$ and let $A_0=\emptyset$. Then
        $$\mu(A)=\sum_{i=1}^\infty \mu(A_i\setminus A_{i-1})=\lim_{n\to\infty}\sum_{i=1}^n\mu(A_i\setminus A_{i-1})=\lim_{n\to\infty}\mu(A_n).$$
        
        \item (c)$\implies$(a) (lower semicontinuity implies $\sigma$-additivity).
        
        Let $B_1,B_2,\ldots\in\mathcal{A}$ be mutually disjoint and let $B=\biguplus_{n=1}^\infty B_n\in\mathcal{A}$. Let $A_n=\biguplus_{i=1}^n B_i$ for each $n\in\mathbb{N}$. Then
        $$\mu(B)=\lim_{n\to\infty}\mu(A_n)=\sum_{i=1}^\infty\mu(B_i).$$
        Thus $\mu$ is $\sigma$-additive.
        
        \item (d)$\implies$(e) ($\emptyset$-continuity implies upper semicontinuity).
        
        Let $A,A_1,A_2,\ldots\in\mathcal{A}$ with $A_n\downarrow A$ and $\mu(A_1)<\infty$. Define $B_n=A_n\setminus A\in\mathcal{A}$ for valid $n$. Then $B_n\downarrow\emptyset$. Thus
        $$\lim_{n\to\infty}\mu(A_n)-\mu(A)=\lim_{n\to\infty}\mu(B_n)=0$$
        and the result is proved.
        
        \item (e)$\implies$(d) (upper semicontinuity implies $\emptyset$-continuity).
        
        This is obvious.
        
        \item (c)$\implies$(d) (lower semicontinuity implies $\emptyset$-continuity).
        
        Let $A_1,A_2,\ldots\in\mathcal{A}$ with $A_n\downarrow\emptyset$ and $\mu(A_1)<\infty$. Then $A_1\setminus A_n\in\mathcal{A}$ for all $n\in\mathbb{N}$ and $A_1\setminus A_n\uparrow A_1$. Thus
        $$\mu(A_1)=\lim_{n\to\infty}\mu(A_1)-\mu(A_n).$$
        Since $\mu(A_1)<\infty$, $\lim_{n\to\infty}A_n=0$ and the result is proved.
        
        \item (d)$\implies$(c) ($\emptyset$-continuity implies lower semicontinuity) if $\mu$ is finite.
        
        Let $A,A_1,A_2,\ldots\in\mathcal{A}$ with $A_n\uparrow A$. Then $A\setminus A_n\downarrow\emptyset$ and
        $$\lim_{n\to\infty}\mu(A)-\mu(A_n)=\lim_{n\to\infty}\mu(A\setminus A_n)=0.$$
        The result follows.
    \end{itemize}
\end{proof}

\begin{definition}[Measurable spaces]
~
\begin{enumerate}[(i)]
    \item A pair $(\Omega,\mathcal{A})$ consisting of a nonempty set $\Omega$ and a $\sigma$-algebra $\mathcal{A}\subseteq2^\Omega$ is called a \textit{measurable space}. The sets $A\in\mathcal{A}$ are called \textit{measurable sets}. If $\Omega$ is countable and $\mathcal{A}=2^\Omega$, then the space $(\Omega,2^\Omega)$ is called \textit{discrete}.
    
    \item A triple $(\Omega,\mathcal{A},\mu)$ is called a \textit{measure space} if $(\Omega,\mathcal{A})$ is a measurable space and $\mu$ is a measure on $\mathcal{A}$.
\end{enumerate}
\end{definition}


\subsection{Measurable Maps}

In measure theory, the homomorphisms (structure-preserving maps between objects) are studied as measurable maps.

\begin{definition}[Measurable map]
    Let $(\Omega,\mathcal{A})$ and $(\Omega',\mathcal{A}')$ be measurable spaces. A map $X:\Omega\to\Omega'$ is called \textit{$\mathcal{A}-\mathcal{A}'$-measurable} (or just measurable) if
    $$X^{-1}(A')\in\mathcal{A}\text{ for any }A'\in\mathcal{A}'.$$
\end{definition}

In this case, we write $X:(\Omega,\mathcal{A})\to(\Omega',\mathcal{A}')$.

\begin{theorem}[Generated $\sigma$-algebra]
    Let $(\Omega',\mathcal{A}')$ be a measurable space and $\Omega$ be a nonempty set. Let $X:\Omega\to\Omega'$ be a map. Then
    $$X^{-1}(\mathcal{A}')=\{X^{-1}(A'):A'\in\mathcal{A}'\}$$
    is the smallest $\sigma$-algebra with respect to which $X$ is measurable. We call $X^{-1}(\mathcal{A}')$ the \textit{$\sigma$-algebra generated by $X$} and denote it as $\sigma(X)$.
\end{theorem}
\begin{proof}
    Let $X$ be measurable with respect to some $\sigma$-algebra $\mathcal{A}$. Then $X^{-1}(A')\in\mathcal{A}$ for any $A'\in\mathcal{A}'$, that is, $\sigma(X)\subseteq\mathcal{A}$. Let us now prove that $\sigma(X)$ is a $\sigma$-algebra by checking each of the axioms in \cref{defSigAlgebra}.
    \begin{enumerate}
        \item As $\Omega'\in\mathcal{A}'$ and $X^{-1}(\Omega')=\Omega$, $\Omega\in\sigma(X)$.
        
        \item Let $A\in\sigma(X)$ and $A'\in\mathcal{A}'$ such that $X^{-1}(A')=A$. Then as $\mathcal{A}'$ is closed under complements,
        $$\Omega\setminus A = X^{-1}(\Omega')\setminus X^{-1}(A') = X^{-1}(\Omega'\setminus A')\in\sigma(X).$$
        Therefore, $\sigma(X)$ is closed under complements.
        
        \item Let $A_1,A_2\ldots\in\sigma(X)$ and $A_1',A_2',\ldots\in\mathcal{A}'$ such that $A_i=X^{-1}(A_i')$ for each $i\in\mathbb{N}$. Then as $\mathcal{A}'$ is $\sigma$-$\cup$-closed,
        $$\bigcup_{i\in\mathbb{N}} A_i = \bigcup_{i\in\mathbb{N}} X^{-1} (A_i') = X^{-1}\left(\bigcup_{i\in\mathbb{N}} A_i'\right) \in\sigma(X)$$
    \end{enumerate}
    Therefore, $\sigma(X)$ is a $\sigma$-algebra.
\end{proof}

\begin{theorem}
\label{generating pi system fixed under preimage}
    Let $(\Omega,\mathcal{A})$ and $(\Omega',\mathcal{A}')$ be measurable spaces and $X:\Omega\to\Omega'$ be a map.
    Let $\mathcal{E}'\subseteq\mathcal{A}'$ be a class of sets. Then $\sigma(X^{-1}(\mathcal{E}'))=X^{-1}(\sigma(\mathcal{E}'))$.
\end{theorem}
\begin{proof}
    We have that $X^{-1}(\mathcal{E})\subseteq X^{-1}(\sigma(\mathcal{E}))=\sigma(X^{-1}(\sigma(\mathcal{E})))$. This implies that
    $$\sigma(X^{-1}(\mathcal{E}))\subseteq X^{-1}(\sigma(\mathcal{E})).$$
    To establish the reverse inclusion, consider
    $$\mathcal{A}_0'=\{A'\in\sigma(\mathcal{E}'):X^{-1}(A')\in\sigma(X^{-1}(\mathcal{E}'))\}$$
    We shall show that $\mathcal{A}_0'$ is a $\sigma$-algebra.
    \begin{enumerate}[(a)]
        \item Clearly, $\Omega'\in\mathcal{A}_0'$ as $\Omega\in\sigma(X^{-1}(\mathcal{E}'))$ and $\Omega'\in\sigma(\mathcal{E}')$.
        
        \item Let $A_0'\in\mathcal{A}_0'$. Then
        $$X^{-1}((A_0')^c) = (X^{-1}(A_0'))^c \in \sigma(X^{-1}(\mathcal{E}'))$$
        and thus $\mathcal{A_0}'$ is closed under complements.
        
        \item Let $A_1',A_2',\ldots\in\mathcal{A}_0'$. Then
        $$X^{-1}\left(\bigcup_{i=1}^\infty A_i'\right) = \bigcup_{i=1}^\infty X^{-1}\left(A_i'\right)\in \sigma(X^{-1}(\mathcal{E}')).$$
        Thus, $\mathcal{A}_0'$ is $\sigma$-$\cup$-closed.
    \end{enumerate}
    
    Now, note that $\mathcal{E}'\subseteq\mathcal{A}_0'$ and $\mathcal{A}_0'\subseteq\sigma(\mathcal{E}')$. This implies that $\mathcal{A}_0'=\sigma(\mathcal{E}')$, and thus $X^{-1}(\sigma(\mathcal{E}'))\subseteq\sigma(X^{-1}(\mathcal{E}'))$.
    
    This proves the result.
\end{proof}

\begin{corollary}
\label{measurable if generator inv}
    Let $(\Omega,\mathcal{A})$ and $(\Omega',\mathcal{A}')$ be measurable spaces and $X:\Omega\to\Omega'$ be a map. Let $\mathcal{E}'\subseteq\mathcal{A}'$ be a class of sets. Then $X$ is $\mathcal{A}$-$\sigma(\mathcal{E}')$ measurable if and only if $X^{-1}(\mathcal{E}')\in\mathcal{A}$. If in particular $\sigma(\mathcal{E}')=\mathcal{A}'$, then $X$ is $\mathcal{A}-\mathcal{A}'$-measurable if and only if $X^{-1}(\mathcal{E}')\subseteq\mathcal{A}'$.
\end{corollary}

\begin{theorem}
        Let $(\Omega,\mathcal{A})$, $(\Omega',\mathcal{A}')$, and $(\Omega'',\mathcal{A}'')$ be measurable spaces and let $X:\Omega\to\Omega'$ and $X':\Omega'\to\Omega''$ be measurable. Then $Y=X'\circ X:\Omega\to\Omega''$ is $\mathcal{A}-\mathcal{A}''$-measurable.
\end{theorem}
\begin{proof}
    This is due to the fact that
    $$Y^{-1}(\mathcal{A}'')\subseteq X^{-1}((X^{-1})(\mathcal{A}''))\subseteq X^{-1}(\mathcal{A}')\subseteq \mathcal{A}.$$
\end{proof}

The above theorem just states that the composition of two measurable maps is measurable.

\begin{theorem}[Measurability of Continuous Maps]
    Let $(\Omega,\tau)$ and $(\Omega',\tau')$ be topological spaces and let $f:\Omega\to\Omega'$ be a continuous map. Then $f$ is $\mathcal{B}(\Omega)-\mathcal{B}(\Omega')$-measurable.
\end{theorem}
\begin{proof}
    As $\mathcal{B}(\Omega')=\sigma(\tau')$, by \cref{measurable if generator inv} it is enough to show that $f^{-1}(A')\in\sigma(\tau)$ for all $A'\in\tau'$. However, since $f$ is continuous, $f^{-1}(A')\in\tau$ for all $A'\in\tau'$ so the result follows.
\end{proof}

\begin{theorem}
    Let $X_1,X_2,\ldots$ be measurable maps $(\Omega,\mathcal{A})\to(\overline{\mathbb{R}}, \mathcal{B}(\overline{\mathbb{R}}))$. Then $\inf_{n\in\mathbb{N}}X_n$, $\sup_{n\in\mathbb{N}}X_n$, $\liminf_{n\in\mathbb{N}}X_n$, and $\limsup_{n\in\mathbb{N}}X_n$ are also measurable.
\end{theorem}
\begin{proof}
    For any $x\in\overline{\mathbb{R}}$,
    $$\left(\inf_{n\in\mathbb{N}}X_n\right)^{-1}([-\infty,x))=\bigcup_{n=1}^\infty (X_n)^{-1}([-\infty,x))\in\mathcal{A}.$$
    The first part of the result follows by \cref{measurable if generator inv}. The proof for $\sup_{n\in\mathbb{N}}X_n$ is similar.
    
    For $\limsup_{n\in\mathbb{N}}$, consider the sequence $(Y_n)_{n\in\mathbb{N}}$ where $Y_n=\sup_{m\geq n}X_m$. Each $Y_m$ is measurable. Then since $\inf_{n\in\mathbb{N}}Y_m$ is measurable, the result follows.
\end{proof}

\begin{definition}[Simple Function]
    Let $(\Omega,\mathcal{A})$ be a measurable space. A map $f:\Omega\to\mathbb{R}$ is called \textit{simple} if there exists some $n\in\mathbb{N}$, mutually disjoint sets $A_1,\ldots,A_n\in\mathcal{A}$, and $\alpha_1,\ldots,\alpha_n\in\mathbb{R}$ such that
    $$f=\sum_{i=1}^n \alpha_i\indic_{A_i}.$$
\end{definition}

\begin{definition}
    Let $f,f_1,f_2,\ldots$ be maps $\Omega\to\overline{\mathbb{R}}$ such that
    $$f_1(\omega)\leq f_2(\omega)\leq\cdots\text{ and }\lim_{n\to\infty}f_n(\omega)=\omega\text{ for all }\omega\in\Omega.$$
    We then write $f_n\uparrow f$. Similarly, we write $f_n\downarrow f$ if $(-f_n)\uparrow(-f)$.
\end{definition}

\begin{theorem}
\label{measurable function simple sequence}
    Let $(\Omega,\mathcal{A})$ be a measurable space and let $f:\Omega\to[0,\infty]$ be measurable. Then
    \begin{enumerate}[(a)]
        \item There exists a sequence $(f_n)_{n\in\mathbb{N}}$ of non-negative simple functions such that $f_n\uparrow f$.
        \item There are measurable sets $A_1,A_2,\ldots\in\mathcal{A}$ and $\alpha_1,\alpha_2,\ldots\in[0,\infty)$ such that $f=\sum_{i=1}^n \alpha_i\indic_{A_i}$.
    \end{enumerate}
\end{theorem}
\begin{proof}
    ~
    \begin{enumerate}[(a)]
        \item For $n\in\mathbb{N}_0$, define
        $$f_n=\min\{n, 2^{-n}\lfloor2^nf\rfloor\}.$$
        Each $f_n$ is measurable. (Why?) Since it can take at most $n2^n+1$ distinct values, each $f_n$ is simple. Clearly, $f_n\uparrow f$.
        
        \item Let $f_n$ be the same as above. For $n\in\mathbb{N}$ and $i\in[2^n]$, define
        $$B_{n,i}=\{\omega:f_n(\omega)-f_{n-1}(\omega)=i2^{-n}\}\text{ and }\beta_{n,i}=i2^{-n}.$$
        Then $f_n-f_{n-1}=\sum_{i=1}^{2^n} \beta_{n,i}\indic_{B_{n,i}}$. Changing the enumeration from $(n,i)$ to $m$, we get some $(\alpha_m)_{m\in\mathbb{N}}$ and $(A_m)_{m\in\mathbb{N}}$ such that
        $$f=f_0 + \sum_{n=1}^\infty(f_n-f_{n-1})=\sum_{m=1}^\infty \alpha_m\indic_{A_m}.$$
    \end{enumerate}
\end{proof}

\subsection{Outer Measure}

\begin{lemma}
\label{uniquely defined by base pi sys}
    Let $(\Omega,\mathcal{A},\mu)$ be a $\sigma$-finite measure space and $\mathcal{E}\subseteq\mathcal{A}$ be a $\pi$-system that generates $\mathcal{A}$. Assume there exists sequence $\Omega_1,\Omega_2\ldots\in\mathcal{E}$ such that $\bigcup_{i=1}^\infty\Omega_i=\Omega$ and $\mu(\Omega_i)<\infty$ for all $i\in\mathbb{N}$. Then $\mu$ is uniquely determined by the values $\mu(E)$, $E\in\mathcal{E}$.
    
    If $\Omega\in\mathcal{A}$ and $\mu(\Omega)=1$, then the existence of the sequence $(\Omega_n)_{n\in\mathbb{N}}$ is not required.
\end{lemma}
\begin{proof}
    Let $\nu$ be a $\sigma$-finite measure on $(\Omega,\mathcal{A})$ such that $\mu(E)=\nu(E)$ for all $E\in\mathcal{E}$.
    
    \vspace{1mm}
    Let $E\in\mathcal{E}$ with $\mu(E)<\infty$. Consider
    $$\mathcal{D}_E=\{A\in\mathcal{A}: \mu(A\cap E)=\nu(A\cap E)\}.$$
    
    We claim that $\mathcal{D}_E$ is a $\lambda$-system. We shall prove this by checking each of the conditions of \cref{defLamSystem}.
    
    \begin{enumerate}[(a)]
        \item Clearly, $\Omega\in\mathcal{D}_E$.
        \item Let $A,B\in\mathcal{D}_E$ with $B\subseteq A$. Then
        \begin{align*}
            \mu((A\setminus B)\cap E) &= \mu(A\cap E) - \mu(B\cap E)\quad\text{(using \cref{properties of content})} \\
            &= \nu(A\cap E) - \nu(B\cap E) \\
            &= \nu((A\setminus B)\cap E).
        \end{align*}
        That is, $(A\setminus B)\in\mathcal{D}_E$.
        \item Let $A_1,A_2,\ldots\in\mathcal{D}_E$ be mutually disjoint sets. Then
        \begin{align*}
            \mu\left(\left(\biguplus_{i=1}^\infty A_i\right)\cap E\right) &= \sum_{i=1}^\infty \mu(A_i\cap E) \\
            &= \sum_{i=1}^\infty\nu(A_i\cap E) \\
            &= \nu\left(\left(\biguplus_{i=1}^\infty A_i\right)\cap E\right).
        \end{align*}
        Therefore, $\biguplus_{i=1}^\infty A_i\in\mathcal{D}_E$ and $\mathcal{D}_E$ is a $\lambda$-system.
    \end{enumerate}
    As $\mathcal{E}\subseteq\mathcal{D}_E$ (Why?), $\delta(\mathcal{E})\subseteq\mathcal{D}_E$. Since $\mathcal{E}$ is a $\pi$-system, \cref{dynkins pi lam theorem} implies that
    $$\mathcal{A}\supseteq\mathcal{D}_E\supseteq\delta(\mathcal{E})=\sigma(\mathcal{E})=\mathcal{A}.$$
    Hence $\mathcal{D}_E=\mathcal{A}$.
    
    Therefore, $\mu(A\cap E)=\nu(A\cap E)$ for any $A\in\mathcal{A}$ and $E\in\mathcal{E}$ with $\mu(E)<\infty$.
    
    Now, let $\Omega_1,\Omega_2,\ldots\in\mathcal{E}$ be a sequence such that $\bigcup_{i=1}^\infty\Omega_i=\Omega$ and $\mu(\Omega_i)<\infty$ for all $i\in\mathbb{N}$. Let $E_0=\emptyset$ and $E_n=\bigcup_{i=1}^n\Omega_i$ for each $n\in\mathbb{N}$. Note that
    $$E_n=\biguplus_{i=1}^n(E_{i-1}^c\cap \Omega_i).$$
    Therefore for any $A\in\mathcal{A}$ and $n\in\mathbb{N}$,
    \begin{align*}
        \mu(A\cap E_n) &= \sum_{i=1}^n\mu((A\cap E_{i-1}^c)\cap\Omega_i) \\
        &= \sum_{i=1}^n\nu((A\cap E_{i-1}^c)\cap\Omega_i) = \nu(A\cap E_n).
    \end{align*}
    
    Now, since $E_n\uparrow\Omega$ and $\mu,\nu$ are lower semicontinuous (by \cref{tripledoubleEquivalence}),
    \begin{align*}
        \mu(A) &= \lim_{n\to\infty}\mu(A\cap E_n) \\
        &= \lim_{n\to\infty}\nu(A\cap E_n) = \nu(A)
    \end{align*}
    
    This proves the result.
    
    \vspace{2mm}
    The second part of the theorem is trivial as $\mathcal{E}\cup\{\Omega\}$ is a $\pi$-system that generates $\mathcal{A}$. Hence one can choose the constant sequence $E_n=\Omega, n\in\mathbb{N}$.
\end{proof}

\begin{definition}[Outer Measure]
    A function $\mu^*:2^\Omega\to[0,\infty]$ is called an \textit{outer measure} if
    \begin{enumerate}[(i)]
        \item $\mu^*(\emptyset)=0$,
        \item $\mu^*$ is monotone, and
        \item $\mu^*$ is $\sigma$-subadditive.
    \end{enumerate}
\end{definition}

\begin{lemma}
\label{set of countable coverings outer measure}
    Let $\mathcal{A}\subseteq 2^\Omega$ be an arbitrary class of sets with $\emptyset\in\mathcal{A}$ and let $\mu$ be a nonnegative set function on $\mathcal{A}$ with $\mu(\emptyset)=0$. For $A\subseteq\Omega$, define the set of countable coverings $\mathcal{F}$ with sets $F\in\mathcal{A}$
    $$\mathcal{U}(A)=\left\{\mathcal{F}\subseteq\mathcal{A} : \mathcal{F}\text{ is countable and }A\subseteq\bigcup_{F\in\mathcal{F}}F\right\}.$$
    Define
    $$\mu^*(A)=\inf\left\{\sum_{F\in\mathcal{F}}\mu(F) : \mathcal{F}\in\mathcal{U}(A) \right\}$$
    where $\inf\emptyset=\infty$. Then $\mu^*$ is an outer measure. If $\mu$ is $\sigma$-subadditive then $\mu^*(A)=\mu(A)$ for all $A\in\mathcal{A}$.
\end{lemma}
\begin{proof}
    Let us check each of the three conditions in the definition of an outer measure.
    \begin{enumerate}[(a)]
        \item Since $\emptyset\in\mathcal{A}$, we have $\{\emptyset\}\in\mathcal{U}(\emptyset)$ and hence $\mu(\emptyset)=0$.
        
        \item If $A\subseteq B$, then $\mathcal{U}(A)\subseteq\mathcal{U}(B)$, and hence $\mu^*(A)\leq\mu^*(B)$.
        
        \item Let $A,A_1,A_2,\ldots\subseteq\Omega$ such that $A\subseteq\bigcup_{i=1}^\infty A_i$. We claim that $\mu^*(A)\leq\sum_{i=1}^\infty \mu^*(A_i)$.
        
        Without loss of generality, assume that $\mu^*(A_i)<\infty$ and hence $\mathcal{U}(A_i)\neq\emptyset$ for all $i\in\mathbb{N}$. Fix some $\varepsilon>0$. Now, for every $n\in\mathbb{N}$, we may choose a covering $\mathcal{F}_n\in\mathcal{U}(A_n)$ such that
        $$\sum_{F\in\mathcal{F}_n}\mu(F)\leq\mu^*(A_n)+\varepsilon2^{-n}.$$
        Then let $\mathcal{F}=\bigcup_{n=1}^\infty \mathcal{F}_n\in\mathcal{U}(A)$.
        $$\mu^*(A)\leq\sum_{F\in\mathcal{F}}\mu(F)\leq\sum_{n=1}^\infty\sum_{F\in\mathcal{F}_n}\mu(F)\leq\sum_{n=1}^\infty \mu^*(A_n) + \varepsilon.$$
        
        This proves the first part of the result.
    \end{enumerate}
    
    To prove the next part of the result, first note that since $\{A\}\in\mathcal{U}(A)$, we have $\mu^*(A)\leq\mu(A)$. If $\mu$ is $\sigma$-subadditive, then for any $\mathcal{F}\in\mathcal{U}(A)$,
    $$\sum_{F\in\mathcal{F}}\mu(F)\geq\mu(A).$$
    It follows that $\mu^*(A)\geq\mu(A).$
\end{proof}

\begin{definition}[$\mu^*$-measurable sets]
    Let $\mu^*$ be an outer measure. A set $A\in 2^\Omega$ is called \textit{$\mu^*$-measurable} if
    $$\mu^*(A\cap E) + \mu^*(A^c\cap E) = \mu^*(E)\text{ for any }E\in 2^\Omega.$$
    We write $\mathcal{M}(\mu^*)=\{A\subseteq\Omega:A\text{ is }\mu^*\text{-measurable}\}.$
\end{definition}

\begin{lemma}
\label{mu measurable iff leq}
    $A\in\mathcal{M}(\mu^*)$ if and only if
    $$\mu^*(A\cap E) + \mu^*(A^c\cap E) \leq \mu^*(E)\text{ for any }E\in 2^\Omega.$$
\end{lemma}
\begin{proof}
    As $\mu^*$ is subadditive, we trivially have
    $$\mu^*(A\cap E) + \mu^*(A^c\cap E) \geq \mu^*(E)\text{ for any }E\in 2^\Omega.$$
    The result follows.
\end{proof}

\begin{lemma}
    $\mathcal{M}(\mu^*)$ is an algebra.
\end{lemma}
\begin{proof}
    We shall check the conditions given in the definition of an algebra \cref{defAlgebra}.
    \begin{enumerate}[(a)]
        \item We clearly have $\Omega\in\mathcal{M}(\mu^*)$.
        \item By definition, $\mathcal{M}(\mu^*)$ is closed under complements.
        \item We must check that $\mathcal{M}(\mu^*)$ is closed under intersections. Let $A,B\in\mathcal{M}(\mu^*)$ and $E\subseteq\Omega$. Then
        \begin{align*}
            \mu^*((A\cap B)\cap E)+\mu^*((A\cap B)^c\cap E) &= \mu^*((A\cap B)\cap E) \\ &\hspace{4mm}+\mu^*((A\cap B^c\cap E)\cup(A^c\cap B\cap E)\cup(A^c\cap B^c\cap E)) \\
            &\leq \mu^*(A\cap (B\cap E))+\mu^*(A\cap (B^c\cap E))\\ &\hspace{4mm}+\mu^*(A^c\cap (B\cap E)) + \mu^*(A^c\cap (B^c\cap E)) \\
            &= \mu^*(B\cap E) + \mu^*(B^c\cap E)\quad\text{(since $A\in\mathcal{M}(\mu^*)$)} \\
            &= \mu^*(E). \quad\text{(since $B\in\mathcal{M}(\mu^*)$)}
        \end{align*}
    \end{enumerate}
    This proves the result.
\end{proof}

\begin{lemma}
\label{outer measure is sig additive on M}
    An outer measure $\mu^*$ is $\sigma$-additive on $\mathcal{M}(\mu^*)$.
\end{lemma}
\begin{proof}
    Let $A,B\in\mathcal{M}(\mu^*)$ with $A\cap B=\emptyset$. Then
    \begin{align*}
        \mu^*(A\cup B) &= \mu^*(A\cap(A\cup B)) + \mu^*(A^c\cap(A\cup B)) \\
        &= \mu^*(A) + \mu^*(B).
    \end{align*}
    That is, $\mu^*$ is additive (and thus a content) on $\mathcal{M}(\mu^*)$. Since $\mu^*$ is $\sigma$-subadditive, \cref{tripledoubleEquivalence} gives the required result.
\end{proof}

\begin{lemma}
    If $\mu^*$ is an outer measure, $\mathcal{M}(\mu^*)$ is a $\sigma$-algebra.
\end{lemma}
\begin{proof}
    We have already shown that $\mathcal{M}(\mu^*)$ is an algebra (and thus a $\pi$-system). Using \cref{cap closed lam sys}, it is sufficient to show that $\mathcal{M}(\mu^*)$ is a $\lambda$-system.
    
    Let $A_1,A_2,\ldots\in\mathcal{M}(\mu^*)$ be mutually disjoint sets and let $A=\biguplus_{i=1}^\infty A_i$. Further, for each $n\in\mathbb{N}$, let $B_n=\biguplus_{i=1}^n A_i$. We must show that $M\in\mathcal{M}(\mu^*)$.
    
    For any $E$ and valid $n\in\mathbb{N}$, we have
    \begin{align*}
        \mu^*(E\cap B_{n+1}) &= \mu^*((E\cap B_{n+1})\cap B_n) + \mu^*((E\cap B_{n+1})\cap B_n^c) \\
        &= \mu^*(E\cap B_n) + \mu^*(E\cap A_{n+1}).
    \end{align*}
    By a simple induction, it follows that
    $$\mu(E\cap B_{n})=\sum_{i=1}^n \mu^*(E\cap A_i).$$
    Since $\mu^*$ is monotonic, we have
    \begin{align*}
        \mu^*(E) &= \mu^*(E\cap B_n) + \mu^*(E\cap B_n^c) \\
        &\geq \mu^*(E\cap B_n) + \mu^*(E\cap A^c) \\
        &= \sum_{i=1}^n\mu^*(E\cap A_i) + \mu^*(E\cap A^c).
    \end{align*}
    Letting $n\to\infty$ and using the fact that $\mu^*$ is $\sigma$-subadditive, we have
    \begin{align*}
        \mu^*(E) &\geq \sum_{i=1}^\infty\mu^*(E\cap A_i) + \mu^*(E\cap A^c) \\
        &\geq \mu^*(E\cap A) + \mu^*(E\cap A^c)
    \end{align*}
    Therefore, $A\in\mathcal{M}(\mu^*)$ and this completes the proof.
\end{proof}

\subsection{The Approximation and Extension Theorems}

\begin{theorem}[Approximation Theorem for Measures]
\label{Approximation Thm for Measures}
    Let $\mathcal{A}\subseteq 2^\Omega$ be a semiring and let $\mu$ be a measure on $\sigma(\mathcal{A})$ that is $\sigma$-finite on $\mathcal{A}$.
    For any $A\in\sigma(\mathcal{A})$ with $\mu(\mathcal{A})<\infty$ and any $\varepsilon>0$, there exists $n\in\mathbb{N}$ and mutually disjoint sets $A_1,A_2,\ldots,A_n\in \mathcal{A}$ such that $\mu\left(A\triangle\bigcup_{i=1}^n A_n\right)<\varepsilon$.
\end{theorem}
\begin{proof}
    Consider the outer measure $\mu^*$ as defined in \cref{set of countable coverings outer measure}. Note that by \cref{set of countable coverings outer measure} and \cref{uniquely defined by base pi sys}, $\mu$ and $\mu^*$ are equal on $\sigma(\mathcal{A})$. By the definition of $\mu^*$, for any $A\in\mathcal{A}$, there exists a covering $B_1,B_2,\ldots\in\mathcal{A}$ of $A$ such that
    $$\mu(A)\geq\sum_{i=1}^\infty\mu(B_i) - \varepsilon/2.$$
    Since $\mu(A)<\infty$, there exists some $n\in\mathbb{N}$ such that $\sum_{i=n+1}^\infty \mu(B_i) < \varepsilon/2$. Now, let $D=\bigcup_{i=1}^n B_i$ and $E=\bigcup_{i=n+1}^\infty B_i$. We have
    \begin{align*}
        A\triangle D &= (D\setminus A)\cup(A\setminus D) \\
        &\subseteq (D\setminus A)\cup (A\setminus (D\cup E))\cup E \\
        &\subseteq (A\triangle (D\cup E))\cup E.
    \end{align*}
    This together with the fact that $A\subseteq\bigcup_{i=1}^\infty B_i$ implies that
    \begin{align*}
        \mu(A\triangle D) &\leq \mu(A\triangle (D\cup E)) + \mu(E) \\
        &\leq  \mu\left(\bigcup_{i=1}^\infty B_i\right) - \mu(A) + \frac{\varepsilon}{2} \\
        &\leq \varepsilon.
    \end{align*}
    Now define $A_1=B_1$ and for each $i\geq 2$, $A_i = B_i\setminus \bigcup_{j=1}^{i=1} B_j$. By definition, $A_1,A_2\ldots$ are mutually disjoint. This proves the result.
    
\end{proof}

The following theorem allows us to ``extend" measures from a semiring to the $\sigma$-algebra generated by it. This allows us to define measures over an entire $\sigma$-algebra by defining its values over just a semiring that generates it.

\begin{ftheo}[Measure Extension Theorem]
\label{MeasureExtensionTh}
    Let $\mathcal{A}$ be a semiring and let $\mu:\mathcal{A}\to[0,\infty]$ be an additive, $\sigma$-subadditive and $\sigma$-finite set function with $\mu(\emptyset)=0$. Then there is a unique $\sigma$-finite measure $\tilde\mu:\sigma(\mathcal{A})\to[0,\infty]$ such that $\tilde\mu(A)=\mu(A)$ for all $A\in\mathcal{A}$.
\end{ftheo}
\begin{proof}
    Since $\mathcal{A}$ is a $\pi$-system, if such a $\tilde\mu$ exists, it is uniquely defined due to \cref{uniquely defined by base pi sys}.
    
    We shall explicitly construct a function that satisfies the given conditions. In order to do so, define as in \cref{set of countable coverings outer measure}
    $$\mu^*(A)=\inf\left\{\sum_{F\in\mathcal{F}}\mu(F) : \mathcal{F}\in\mathcal{U}(A) \right\}\text{ for any $A\subseteq\Omega$.}$$
    
    By \cref{set of countable coverings outer measure}, $\mu^*$ is an outer measure and $\mu^*(A)=\mu(A)$ for any $A\in\mathcal{A}$.
    
    \vspace{2mm}
    We first claim that $\mathcal{A}\subseteq\mathcal{M}(\mu^*)$.
    
    To prove this, let $A\in\mathcal{A}$ and $E\subseteq\Omega$ with $\mu^*(E)<\infty$. Fix some $\varepsilon>0$. Then by the definition of $\mu^*$, there exists a sequence $E_1,E_2,\ldots\in\mathcal{A}$ such that
    $$E\subseteq\bigcup_{i=1}^\infty E_i\text{ and }\sum_{i=1}^\infty \mu(E_i)\leq \mu^*(E)+\varepsilon.$$
    
    For each $n$, define $B_n=E_n\cap A$. Since $\mathcal{A}$ is a semiring, there exists for each $n$ some $m_n\in\mathbb{N}$ and mutually disjoint sets $C_{n, 1},C_{n, 2},\ldots,C_{n, m_n}$ such that
    $$E_n\setminus A = E_n\setminus B_n = \biguplus_{i=1}^{m_n}C_{n, i}$$
    Then we have that
    \begin{align*}
        E\cap A &\subseteq \bigcup_{n=1}^\infty B_n, \\
        E\cap A^c &\subseteq \bigcup_{n=1}^\infty\biguplus_{i=1}^{m_n}C_{n,i},\text{ and } \\
        E_n &= B_n\uplus\biguplus_{i=1}^{m_n}C_{n,i}.
    \end{align*}
    
    This implies that
    \begin{align*}
        \mu^*(E\cap A) + \mu^*(E\cap A^c) &\leq \sum_{n=1}^\infty\mu(B_n) + \sum_{n=1}^\infty\sum_{i=1}^{m_n}\mu(C_{n,i}) \quad\text{(since $\mu$ is $\sigma$-subadditive)} \\
        &= \sum_{n=1}^\infty\left(\mu(B_n) + \sum_{i=1}^{m_n}\mu(C_{n,i})\right) \\
        &= \sum_{n=1}^\infty \mu(E_n) \quad\text{(since $\mu$ is additive)} \\
        &   \leq \mu^*(E) + \varepsilon.
    \end{align*}
    
    \cref{mu measurable iff leq} implies that $A\in\mathcal{M}(\mu^*)$, that is, $\mathcal{A}\subseteq\mathcal{M}(\mu^*)$. This in turn in implies that $\sigma(\mathcal{A})\subseteq\mathcal{M}(\mu^*)$. Define the required function by $\tilde\mu:\sigma(\mathcal{A})\to[0,\infty]$, $A\mapsto\mu^*(A)$. By \cref{outer measure is sig additive on M}, $\tilde\mu$ is $\sigma$-additive. Since $\mu$ is $\sigma$-finite, $\tilde\mu$ is $\sigma$-finite as well. This proves the result.
\end{proof}

\subsection{Important Examples of Measures}

Now that we have the Measure Extension Theorem, we may introduce the Lebesgue-Stieltjes measure, a very useful measure on $(\mathbb{R},\mathcal{B}(\mathbb{R}))$, which is given as follows.

\begin{definition}[Lebesgue-Stieltjes Measure]
\label{defLebStielMeasure}
    Let $F:\mathbb{R}\to\mathbb{R}$ be monotone increasing and right continuous. The measure $\mu_F$ on $(\mathbb{R},\mathcal{B}(\mathbb{R}))$ defined by 
    $$\mu_F((a,b])= F(b)-F(a)\text{ for all $a,b\in\mathbb{R}$ such that $a<b$}$$
    is called the \textit{Lebesgue-Stieltjes measure} with distribution function $F$.
\end{definition}

The Lebesgue-Stieltjes measure is well-defined due to the Measure Extension Theorem \cref{MeasureExtensionTh}.

\vspace{1mm}
To see this more clearly, let $\mathcal{A}=\{(a,b\,]:a,b\in\mathbb{R}\text{ and }a\leq b\}$. It may be checked that $\mathcal{A}$ is a semiring. Further, $\sigma(\mathcal{A})=\mathcal{B}(\mathbb{R})$. Now, define the function $\tilde\mu_F:\mathcal{A}\to[0,\infty)$ by $(a,b]\mapsto F(b)-F(a)$. Clearly $\tilde\mu_F(\emptyset)=0$ and the function is additive. It remains to check that $\tilde\mu_F$ is $\sigma$-subadditive.

\vspace{1mm}
Let $(a,b],(a_1,b_1],(a_2,b_2],\ldots\in\mathcal{A}$ such that $(a,b\,]\subseteq\bigcup_{i=1}^\infty (a_i, b_i]$. Fix some $\varepsilon>0$ and choose $a_\varepsilon\in(a,b)$ such that
$$F(a_\varepsilon)-F(a)< \varepsilon/2\implies \tilde\mu_F((a,b]) - \tilde\mu_F((a_\varepsilon,b\,]) < \varepsilon/2.$$
It is possible to choose such an $\varepsilon$ due to the right continuity of $F$. Also, for any $k\in\mathbb{N}$, choose $b_{k, \varepsilon}$ such that $$F(b_{k,\varepsilon})-F(b_k)<\varepsilon 2^{-k-1}\implies \tilde\mu_F((a_k,b_{k,\varepsilon}]) - \tilde\mu_F((a_k,b_k]) < \varepsilon 2^{-k-1}.$$
We now have
$$[a_\varepsilon, b\,]\subseteq (a,b]\subseteq \bigcup_{i=1}^\infty (a_k,b_k]\subseteq \bigcup_{k=1}^\infty (a_k,b_{k,\varepsilon}]$$
Due to the compactness of $[a_\varepsilon,b]$, there then exists some $k_0\in\mathbb{N}$ such that
$$(a_{\varepsilon},b\,]\subseteq\bigcup_{k=1}^{k_0}(a_k,b_{k,\varepsilon}].$$
This implies that
\begin{align*}
    \tilde\mu_F((a,b\,]) &\leq \frac{\varepsilon}{2} + \tilde\mu_F((a,b]) \\
    &\leq \frac{\varepsilon}{2} + \sum_{k=1}^{k_0} \tilde\mu_F((a_k,b_{k,\varepsilon}]) \\
    &\leq \frac{\varepsilon}{2} + \sum_{k=1}^{k_0} \left(\tilde\mu_F((a_k,b_k]) + \varepsilon2^{-k-1}\right) \\
    &\leq \varepsilon + \sum_{k=1}^{\infty} \tilde\mu_F((a_k,b_k])
\end{align*}
As this is true for any choice of $\varepsilon$, $\tilde\mu_F$ is $\sigma$-subadditive.

\vspace{2mm}
Then the extension of $\tilde\mu_F$ uniquely to a $\sigma$-finite measure is guaranteed by \cref{MeasureExtensionTh}. This measure is known as the Lebesgue-Stieltjes measure.

\vspace{2mm}
The measure that results when the function $F$ is equal to the identity function is referred to the \textit{Lebesgue measure} on $\mathbb{R}^1$. Similar to this, we can define the Lebesgue measure in general as follows.

\begin{definition}[Lebesgue Measure]
    There exists a unique measure $\lambda^n$ on $(\mathbb{R}^n,\mathcal{B}(\mathbb{R}^n))$ such that for all $a,b\in\mathbb{R}^n$ with $a<b$,
    $$\lambda^n((a,b])=\prod_{i=1}^n (b_i-a_i).$$
    $\lambda^n$ is called the \textit{Lebesgue measure} on $(\mathbb{R}^n, \mathcal{B}(\mathbb{R}^n))$ or the \textit{Lebesgue-Borel measure}.
\end{definition}

\vspace{2mm}
Let $E$ be a finite nonempty set and $\Omega = E^\mathbb{N}$. If $\omega_1,\omega_2,\ldots,\omega_n\in E$, we define the following.
$$[\omega_1,\omega_2,\ldots,\omega_n]=\{\omega'\in\Omega : \omega'_i=\omega_i\text{ for }i\in[n]\}.$$
This represents the set of all sequences whose first $n$ elements are $\omega_1,\omega_2,\ldots,\omega_n$.

\begin{theorem}[Finite Products of Measures]
        Let $n\in\mathbb{N}$ and $\mu_1,\mu_2,\ldots,\mu_n$ be Lebesgue-Stieltjes measures on $(\mathbb{R},\mathcal{B}(\mathbb{R}))$. Then there exists a unique $\sigma$-finite measure $\mu$ on $(\mathbb{R}^n, \mathcal{B}(\mathbb{R}^n))$ such that for all $a,b\in\mathbb{R}^n$ with $a<b$,
        $$\mu((a,b])=\prod_{i=1}^n \mu_i((a_i,b_i])$$
        We call $\mu$ the \textit{product measure of $\mu_1,\mu_2,\ldots,\mu_n$} and denote it by $\bigotimes_{i=1}^n\mu_i$.
\end{theorem}

The proof of the above is similar to that of \cref{MeasureExtensionTh}. We choose intervals $(a,b_\varepsilon]$ and so on such that $\mu((a,b_\varepsilon])<\mu((a,b])+\varepsilon$. Such $b_\varepsilon$ exists due to the right continuity of each of the $F_i$s corresponding to each of the $\mu_i$s.

\clearpage
\section{Introduction to Probability}

\subsection{Basic Definitions}

\begin{definition}[Probability Measure]
    Let $\mathcal{A}$ be a $\sigma$-algebra and $\mu$ be a measure on $\mathcal{A}$. $\mu$ is called a \textit{probability measure} if $\mu(\Omega)=1$.
\end{definition}

Let $\Omega$ be a countable non-empty set and $\mathcal{A}=2^\Omega$. Further let $(p_\omega)_{\omega\in\Omega}$ be non-negative numbers. The map given by $A\mapsto\mu(A)=\sum_{\omega\in A}p_\omega$ defines a $\sigma$-finite measure on $2^\Omega$. $p=(p_\omega)_{\omega\in\Omega}$ is called the \textit{weight function} of $\mu$. $p_\omega$ is called the \textit{weight of $\mu$ at point $\omega$}.

In the case where $\sum_{\omega\in\Omega}p_\omega=1$, $\mu$ is a probability measure. Then the vector $(p_\omega)_{\omega\in\Omega}$ is called a \textit{probability vector}. 

\begin{definition}[Probability Distribution Function]
    A right continuous monotonically increasing function $F:\mathbb{R}\to[0,1]$ such that $\lim_{x\to-\infty}F(x)=0$ and $\lim_{x\to\infty}F(x)=1$ is called a \textit{(proper) probability distribution function}, often abbreviated as \textit{p.d.f.} If we instead have $\lim_{x\to\infty}F(x)\leq 1$, $F$ is called a \textit{(possibly) defective p.d.f.} If $\mu$ is a probability measure on $(\mathbb{R},\mathcal{B}(\mathbb{R}))$, then the function $F_\mu$ given by $x\mapsto \mu((\infty,x])$ is called the \textit{distribution function} of $\mu$.
\end{definition}

Note that a probability measure is uniquely determined by its distribution function.

\begin{definition}[Probability Space]
    Let $(\Omega,\mathcal{A},\mu)$ be a measure space. If in addition $\mu(\Omega)=1$, then $(\Omega,\mathcal{A},\mu)$ is called a \textit{probability space}.
\end{definition}

In the above definition, $\Omega$ is called the \textit{sample space}, $\mathcal{A}$ is called the \textit{event space} (and its elements are called \textit{events}, and $\mu$ is called the \textit{probability function}.

\vspace{2mm}
Let $\Omega$ be a finite nonempty set. Let $\mathcal{A}=2^\Omega$ and consider the function $\mu:\mathcal{A}\to[0,1]$ given by
$$\mu(A)=\frac{|A|}{|\Omega|}\text{ for each $A\subseteq\Omega$.}$$
This defines a probability measure on $\mathcal{A}$. This function $\mu$ is called the \textit{uniform distribution on $\Omega$} and is denoted $\mathcal{U}_\Omega$. The resulting probability space $(\Omega,\mathcal{A},\mathcal{U}_\Omega)$ is called a \textit{Laplace space}.

\vspace{2mm}
Another example is as follows. Let $\omega\in\Omega$ and $\delta_\omega(A)=\indic(\{\omega\})$. Then $\delta_\omega$ is a probability measure on any $\sigma$-algebra $\mathcal{A}\subseteq2^\Omega$. $\delta_\omega$ is called the \textit{Dirac measure} for the point $\omega$.

The Dirac measure is useful in constructing discrete probability distributions.

\vspace{2mm}
Consider the example of a coin toss. The sample space $\Omega$ has two elements, $\text{H}$ (for heads) and $\text{T}$ (for tails). The event space $\mathcal{A}$ then has four elements $\emptyset$, $\{H\}$, $\{T\}$, and $\{H,T\}$. Each of these events have associated probabilities $0,\frac{1}{2},\frac{1}{2}$, and $1$ respectively. Note that $\{H,T\}$ represents the event that either a heads or a tails occurs.`

\begin{definition}[Random Variable]
    Let $(\Omega,\mathcal{A},\textbf{P})$ be a probability space, $(\Omega',\mathcal{A}')$ a measurable space, and $X:\Omega\to\Omega'$ be measurable. Then
    \begin{enumerate}[(i)]
        \item $X$ is called a \textit{random variable} with values in $(\Omega',\mathcal{A}')$. If $(\Omega',\mathcal{A}')=(\mathbb{R},\mathcal{B}(\mathbb{R}))$, then $X$ is called a \textit{real random variable}.
        
        \item For $A'\in\mathcal{A}'$, we often denote
        $$\textbf{P}[X^{-1}(A')]\text{ as }\textbf{P}[X\in A']\text{ and } X^{-1}(A')\text{ as }\{X\in A'\}.$$
        In particular, we let $\{X\geq 0\}=X^{-1}([0,\infty))$ and define $\{X\leq b\}$ and other terms similarly.
    \end{enumerate}
\end{definition}

As we shall primarily deal with real random variables in our study of probability, we often drop the ``real" and refer to them as just random variables.

\begin{definition}
    Let $X$ be a random variable with underlying probability space $(\Omega,\mathcal{A},\textbf{P})$.
    \begin{enumerate}[(i)]
        \item The probability measure $\textbf{P}_X=\textbf{P}\circ X^{-1}$ is called the \textit{distribution} of $X$.
        \item For a real random variable $X$, the map $F_X$ given by $x\mapsto\textbf{P}[X\leq x]$ is called the \textit{distribution function} of $P_X$ (or $X$). If $\mu=\textbf{P}_X$, we write $X\sim\mu$ and say that $X$ has distribution $\mu$.
        \item A family $(X_i)_{i\in I}$ of random variables is called \textit{identically distributed} if $\textbf{P}_{X_i}=\textbf{P}_{X_j}$ for all $i,j\in I$. We write $X\iddistrib Y$ if $\textbf{P}_X=\textbf{P}_Y$ ($\mathcal{D}$ for \textit{distribution}).
    \end{enumerate}
\end{definition}

\begin{theorem}
    For any p.d.f. $F$, there exists a real random variable $X$ with $F_X=F$.
\end{theorem}

\begin{proof}
    We shall explicitly construct a probability space $(\Omega,\mathcal{A},\textbf{P})$ and random variable $X:\Omega\to\mathbb{R}$ such that $F_X=F$.
    
    \vspace{1mm}
    One choice that might come to mind is to take $(\Omega,\mathcal{A}) = (\mathbb{R},\mathcal{B}(\mathbb{R}))$, $X:\mathbb{R}\to\mathbb{R}$ as the identity function, and \textbf{$P$} the Lebesgue-Stieltjes measure with distribution function $F$.
    
    \vspace{1.5mm}
    While this choice of ours works, let us attempt to construct another more ``standard" choice that is perhaps more enlightening. Let $\Omega=(0,1),\mathcal{A}=\left.\mathcal{B}(\Omega)\right|_\Omega$ and $\textbf{P}$ be the Lebesgue measure on $(\Omega,\mathcal{A})$. This is standard in the sense that given any $F$, we construct a random variable over the same probability space. Define the left continuous inverse of $F$ as
    $$F^{-1}(t)=\inf\{x\in\mathbb{R}:F(x)\geq t\}\text{ for }t\in (0,1).$$
    Note that $F^{-1}(t)\leq x$ if and only if $F(x)\geq t$.
    In particular,
    $$\{t:F^{-1}(t)\leq x\}=(0,F(x)]\cap(0,1)$$
    and so $F^{-1}:(\Omega,\mathcal{A})\to(\mathbb{R},\mathcal{B}(\mathbb{R}))$ is measurable. Thus
    $$\textbf{P}\left[\{t:F^{-1}(t)\leq x\}\right]=F(x).$$
    This implies that $F^{-1}$ is the random variable we wish to construct.
\end{proof}

Note that the above implies that there is a bijection between probability distribution functions and distribution functions corresponding to random variables.

\begin{definition}
    If a distribution $F:\mathbb{R}^n\to[0,1]$ is of the form
    $$F(x)=\int_{-\infty}^{x_1}\d{t_1}\int_{-\infty}^{x_2}\d{t_2}\cdots\int_{-\infty}^{x_n}\d{t_n}\, f(t_1,t_2,\ldots,t_n)\text{ for }(x_1,x_2,\ldots,x_n)\in\mathbb{R}^n$$
    for some integrable function $f:\mathbb{R}^n\to[0,\infty)$, then $f$ is called the \textit{density of the distribution}.
\end{definition}

\subsection{Important Examples of Random Variables}

We now give several important examples of random variables that we shall encounter several times in our study of probability.

\begin{enumerate}
    \item \textit{Bernoulli Distribution.}
    
    Let $p\in[0,1]$ and $\textbf{P}[X=1]=p$, $P[X=0]=1-p$. Then $\textbf{P}_X$ is called the \textit{Bernoulli distribution with parameter $p$} and is denoted $\Ber_p$. More formally,
    $$\Ber_p=(1-p)\delta_0 + p\delta_1.$$
    Its distribution function is
    $$
    F_X(x) = 
    \begin{cases}
    0, &x<0 \\
    1-p, &x\in[0,1) \\
    1, &x\geq 1
    \end{cases}
    $$
    
    Note that the above can be likened to the outcome of a weighted coin, with heads and tails corresponding to $0$ and $1$.
    
    \vspace{2mm}
    The distribution $\textbf{P}_Y$ of $Y=2X-1$ is called the \textit{Rademacher distribution with parameter $p$}. More formally,
    $$\Rad_p = (1-p)\delta_{-1}+p\delta_1.$$
    $\Rad_{1/2}$ is simply called the Rademacher distribution.
    
    \item \textit{Binomial Distribution.}
    
    Let $p\in[0,1]$ and $n\in\mathbb{N}$. Let $X:\Omega\to\{0,1,2,\ldots,n\}$ be such that for each valid $k$,
    $$\textbf{P}[X=k]=\binom{n}{k}p^k(1-p)^{n-k}.$$
    Then $\textbf{P}_X$ is called the \textit{binomial distribution with parameters $n$ and $p$} and is denoted $b_{n,p}$. More formally,
    $$b_{n,p}=\sum_{k=0}^n \binom{n}{k}p^k(1-p)^{n-k}\delta_k.$$
    
    \item \textit{Geometric Distribution.}
    
    Let $p\in(0,1]$ and $X:\Omega\to\mathbb{N}_0$ be such that for each $n\in\mathbb{N}_0$,
    $$\textbf{P}[X=n]=p(1-p)^n.$$
    Then $\textbf{P}_X$ is called the \textit{geometric distribution with parameter $p$} and is denoted $\gamma_p$ or $b_{1,p}^{-}$. More formally,
    $$\gamma_p=\sum_{n=0}^\infty p(1-p)^n\delta_n.$$
    
    \item \textit{Negative Binomial Distribution.}
    
    Let $r>0$ and $p\in(0,1]$. We denote by
    $$b^{-}_{r,p}=\sum_{k=0}^\infty \binom{-r}{k}(-1)^kp^r(1-p)^k\delta_k$$
    the \textit{negative binomial distribution} or \textit{Pascal distribution} with parameters $r$ and $p$. Note that $r$ need not be an integer.
    
    \item \textit{Poisson Distribution.}
    
    Let $\lambda\in[0,\infty)$ and $X:\Omega\to\mathbb{N}_0$ be such that for each $n\in\mathbb{N}_0$,
    $$P[X=n]=e^{-\lambda}\frac{\lambda^n}{n!}.$$
    Then $\textbf{P}_x=\Poi_\lambda$ is called the \textit{Poisson distribution with parameter $\lambda$}.
    
    \item \textit{Hypergeoemetric Distribution.}
    
    Consider a basket with $B\in\mathbb{N}$ black balls and $W\in\mathbb{N}$ white balls. If we draw $n\in\mathbb{N}$ balls from the basket, some simple combinatorics shows that the probability of drawing (exactly) $b\in\{0,1,2,\ldots,n\}$ black balls is given by the \textit{hypergeometric distribution with parameters $B,W,n$}:
    $$\operatorname{Hyp}_{B,W;n}(\{b\})=\frac{\binom{B}{b}\binom{W}{n-b}}{\binom{B+W}{n}}.$$
    In general, if we have $k$ colors with $B_i$ balls of colour $i$ for each $i$, the probability of drawing exactly $b_i$ balls of colour $i$ for each $i$ is given by the \textit{generalised hypergeometric distribution}:
    $$\operatorname{Hyp}_{B_1,B_2,\ldots,B_k;n}(\left\{(b_1,b_2,\ldots,b_k)\right\}) = \frac{\binom{B_1}{b_1}\binom{B_2}{b_2}\cdots\binom{B_k}{b_k}}{\binom{B_1+B_2+\cdots+B_k}{n}}$$
    where $n=b_1+b_2+\cdots+b_k$.
    
    \item \textit{Gaussian Normal Distribution.}
    
    Let $\mu\in\mathbb{R}, \sigma^2>0$. Let $X$ be a real random variable such that for $x\in\mathbb{R}$,
    $$\textbf{P}[X\leq x]=\frac{1}{\sqrt{2\pi\sigma^2}}\int_{-\infty}^x \exp\left(-\frac{(t-\mu)^2}{2\sigma^2}\right)\d{t}$$
    Then $\textbf{P}_X$ is called the \textit{Gaussian normal distribution} (or just \textit{normal distribution}) \textit{with parameters $\mu$ and $\sigma^2$} and is denoted $\mathcal{N}_{\mu,\sigma^2}$. In particular, $\mathcal{N}_{0,1}$ is the standard normal distribution. 
    
    \item \textit{Exponential Distribution.}
    
    Let $\theta>0$ and $X$ be a nonnegative random variable such that for each $x\geq 0$,
    $$\textbf{P}[X\leq x]=\textbf{P}\left[X\in[0,x]\right]=\int_{0}^x\theta e^{-\theta t}\d{t}.$$
    Then $\textbf{P}_X$ is called the \textit{exponential distribution with parameter $\theta$} and is denoted $\exp_\theta$.
    
    % \item \textit{$d$-dimensional Normal Distribution.}
    
    % Let $\mu\in\mathbb{R}^d$ and $\Sigma$ be a positive definite symmetric $d\times d$ matrix. Let $X$ be an $\mathbb{R}^d$-valued random variable such that for each $x\in\mathbb{R}^d$,
    % $$\textbf{P}[X\leq x] = \det(2\pi\Sigma)^{-1/2}\int_{-\infty}^x \exp\left(-\frac{1}{2}\langle t-\mu, \Sigma^{-1}(t-\mu)\rangle\right)\lambda^d\d{t}$$
    % where $\langle\cdot,\cdot\rangle$ represents the standard inner product in $\mathbb{R}^d$. Then $\textbf{P}_X$ is called the \textit{$d$-dimensional normal distribution with parameters $\mu$ and $\Sigma$} and is denoted $\mathcal{N}_{\mu,\Sigma}$.
    
\end{enumerate}

\subsection{The Product Measure}

Let $E$ be a finite set and $\Omega=E^\mathbb{N}$. Let $(p_e)_{e\in E}$ be a probability vector. Define
$$\mathcal{A}=\{[\omega_1,\ldots,\omega_n]:\omega_1,\ldots,\omega_n\text{ and }n\in\mathbb{N}\}$$
and a content $\mu$ on $\mathcal{A}$ by
$$\mu([\omega_1,\omega_2,\ldots,\omega_n])=\prod_{i=1}^n p_{\omega_i}$$
We wish to extend $\mu$ to a measure on $\sigma(\mathcal{A})$. Similar to how we proved the existence of the Lebesgue-Stieltjes measure \ref{defLebStielMeasure}, we use a compactness argument to show that $\mu$ is $\sigma$-subadditive.

Let $A,A_1,A_2,\ldots\in\mathcal{A}$ such that $A\subseteq\bigcup_{i=1}^\infty A_i$. We claim that there exists $n\in\mathbb{N}$ such that $A\subseteq\bigcup_{i=1}^n A_i.$

For each $n\in\mathbb{N}$, let $B_n=A\setminus\bigcup_{i=1}^n A_i$. We assume that $B_n\neq\emptyset$ for all $n\in\mathbb{N}$ and prove the required by contradiction.

Due to the pigeonhole principle, there exists some $\omega_1\in E$ such that $[\omega_1]\cap B_n\neq\emptyset$ for infinitely many $n\in\mathbb{N}$. Since $B_1\supseteq B_2\supseteq\cdots$, we have that
$$[\omega_1]\cap B_n\neq\emptyset\text{ for all }n\in\mathbb{N}.$$
Similarly, there exist $\omega_2,\omega_3,\ldots\in E$ such that
$$[\omega_1,\ldots,\omega_k]\cap B_n\neq\emptyset\text{ for all }k,n\in\mathbb{N}.$$

Each $B_n$ is a disjoint union of sets $C_{n,1},\ldots,C_{n,m_n}\in\mathcal{A}$. Thus for each $n\in\mathbb{N}$, there is some $i_n\in\{1,2,\ldots,m_n\}$ such that
$$[\omega_1,\omega_2,\ldots,\omega_k]\cap C_{n,i_n}\neq\emptyset\text{ for infinitely many }k\in\mathbb{N}.$$
As $[\omega_1]\supseteq[\omega_1,\omega_2]\supseteq\cdots$, this implies that
$$[\omega_1,\omega_2,\ldots,\omega_k]\cap C_{n,i_n}\neq\emptyset\text{ for all } k\in\mathbb{N}$$

As $C_{n,i_n}\in\mathcal{A}$, for fixed $n$ and large $k$ ($k\geq m_n$), we have $$[\omega_1,\omega_2,\ldots,\omega_k]\subseteq C_{n,i_n}.$$
This implies that $\omega=(\omega_1,\omega_2,\ldots)\in C_{n,i_n}\subseteq B_n$. This in turn implies that $\bigcap_{i=0}^\infty B_i\neq\emptyset$, which yields a contradiction.

Therefore, $A\subseteq\bigcup_{i=1}^n A_n$ for some $n\in\mathbb{N}$. Since $\mu$ is known to be (finite) subadditive, we have
$$\mu(A)\leq \sum_{i=1}^n\mu(A_i)\leq \sum_{i=1}^\infty \mu(A_i),$$
which is the required result.

\begin{definition}[Product Measure]
\label{defProductMeasure}
    Let $E$ be a finite nonempty set and $\Omega=E^\mathbb{N}$. Let $(p_e)_{e\in E}$ be a probability vector. There then is a unique probability measure $\mu$ on $\sigma(\mathcal{A})=\mathcal{B}(\Omega)$ (where $\mathcal{A}$ is defined as above) such that
    $$\mu([\omega_1,\omega_2,\ldots,\omega_n])=\prod_{i=1}^n p_{\omega_i}\text{ for all $\omega_i\in E$ and $n\in\mathbb{N}$}.$$
    $\mu$ is called the \textit{product measure} or \textit{Bernoulli measure} on $\Omega$ with weights $(p_e)_{e\in E}$ and is denoted by $\left(\sum_{e\in E}p_e\delta_e\right)^{\otimes\mathbb{N}}$. The $\sigma$-algebra $\sigma(\mathcal{A})$ is called the \textit{product $\sigma$-algebra on $\Omega$} and is denoted by $(2^E)^{\otimes\mathbb{N}}$.
\end{definition}

We explain the above more intuitively in the following subsection.

\subsection{Independent Events}

In the following, let $(\Omega,\mathcal{A},\textbf{P})$ be a probability space and the sets $A\in\mathcal{A}$ be events. The $\Pr$ or $\textbf{P}$ symbol will denote the universal object of a probability measure (we used $\textbf{P}$ to denote this until now), and the probabilities $\Pr[\cdot]$ are always written in (square) brackets.
    
\begin{definition}
    Two events $A$ and $B$ are said to be \textit{independent} if
    $$\Pr[A\cap B] = \Pr[A]\Pr[B].$$
\end{definition}

An common example of independent events is that of rolling a die twice, where the outcome of the first roll is independent of the second.

Here, $\Omega=\{1,2,\ldots,6\}^2, \mathcal{A}=2^\Omega$ and the probability distribution is $\textbf{P}=\mathcal{U}_\Omega$. Our claim may be verified as follows. Let $\tilde A,\tilde B\subseteq\Omega, A=\tilde A\times\Omega$ and $B=\Omega\times\tilde B$. We must show that $\Pr[A]\Pr[B]=\Pr[A\cap B]$. This is obvious as follows:
\begin{align*}
    \Pr[A] &= \frac{|A|}{36}=\frac{|\tilde A|}{6} \\
    \Pr[B] &= \frac{|B|}{36}=\frac{|\tilde B|}{6} \\
    \Pr[A\cap B] &= \frac{|A\cap B|}{36} = \frac{|\tilde A||\tilde B|}{36} = \Pr[A]\Pr[B].
\end{align*}

While in the above example it is intuitively clear that the two events must be independent, we can have less obvious examples as well. For example, the event that the sum of the two rolls is odd and the event that the first roll gives at most a three are independent. We leave it to the reader to verify this claim.

\vspace{2mm}
We extend this definition of two independent events to any number of independent events as follows.

\begin{definition}[Independence of Events]
    Let $I$ be an index set and $(A_i)_{i\in I}$ be a family of events. The family $(A_i)_{i\in I}$ is called \textit{independent} if for any finite subset $J\subseteq I$, the following holds:
    $$\Pr\left[\bigcap_{j\in J}A_j\right]=\prod_{j\in J}\Pr[A_j].$$
\end{definition}

Let us now return to the product measure discussed in \ref{defProductMeasure}, which can be understood intuitively as follows. If $E$ is a finite set of outcomes, consider the probability space comprising $\Omega=E^\mathbb{N}$, the $\sigma$-algebra
$$\mathcal{A}=\sigma([\omega_1,\ldots,\omega_n]:\omega_1,\ldots,\omega_n\in E\text{ and }n\in\mathbb{N})$$
and the product measure $\textbf{P}=\left(\sum_{e\in E}p_e\delta_e\right)^{\otimes\mathbb{N}}$. This basically represents that we repeatedly conduct the experiment of choosing an outcome from $E$. Let $\tilde A_i\subseteq E$ for any $i\in\mathbb{N}$ and let $A_i$ be the event such that $\tilde A_i$ occurs in the $i$th experiment, given by
$$A_i = \{\omega\in\Omega:\omega_i\in\tilde A_i\}
  =\biguplus_{(\omega_1,\ldots,\omega_i)\in E^{i-1}\times \tilde A_i} [\omega_1,\ldots,\omega_i]$$

Intuitively, the family $(A_i)_{i\in\mathbb{N}}$ should be independent, since the outcome of one of the conducted experiments does not depend on the outcomes of the other experiments.

Let us check this. Let $J\subseteq\mathbb{N}$. For $j\in J$, let $B_j=A_j$ and $\tilde B_j=\tilde A_j$ and for $j\in\{1,2,\ldots,n\}\setminus J$, let $B_j=\Omega$ and $\tilde B_j=E$. Then
\begin{align*}
    \Pr\left[\bigcap_{j\in J}A_j\right] &= \Pr\left[\bigcap_{j=1}^n B_j\right] \\
    &= \Pr\left[\left\{\omega\in\Omega:\omega_j\in\tilde B_j\text{ for each }j\in\{1,2,\ldots,n\}\right\}\right] \\
    &= \sum_{e_1\in\tilde B_1} \cdots \sum_{e_n\in\tilde B_n} \prod_{j=1}^n p_{e_j} \\
    &= \prod_{j=1}^n \left(\sum_{e\in\tilde B_j}p_e\right) \\
    &= \prod_{j\in J} \left(\sum_{e\in\tilde A_j}p_e\right)
\end{align*}

As this is true in particular for $|J|=1$, we have for some fixed $i\in\{1,2,\ldots,n\}$,
$$\Pr[A_i] = \left(\sum_{e\in\tilde A_i}p_e\right).$$
Substituting the above, we have
$$
\Pr\left[\bigcap_{j\in J}A_j\right] 
= \prod_{j\in J} \left(\sum_{e\in\tilde A_j}p_e\right)
= \prod_{j\in J} \Pr[A_j].
$$
This proves the result.

\vspace{2mm}
Note that if events $A$ and $B$ are independent, then the events $A^c$ and $B$ are independent as well. This can be described more precisely as follows.

\begin{theorem}
    Let $I$ be an index set and $(A_i)_{i\in I}$ be a family of events. Define $B_i^0 = A_i$ and $B_i^1 = A_i^c$ for each $i\in I$. Then the following statements are equivalent.
    \begin{enumerate}[(a)]
        \item The family $(A_i)_{i\in I}$ is independent.
        \item There is some $\alpha\in\{0,1\}^I$ such that the family $(B_i^{\alpha_i})_{i\in I}$ is independent.
        \item For all $\alpha\in\{0,1\}^I$, the family $(B_i^{\alpha_i})_{i\in I}$ is independent.
    \end{enumerate}
\end{theorem}

We leave the proof of the above to the reader.

Now, recall the limes superior, defined in \ref{defLimes}. The limes superior represents the event that a particular event occurs an infinite amount of times. For example, if we roll a die an infinite number of times, we could consider the event that we roll a four an infinite number of times. This is formalized in the following.

\begin{theorem}[Borel-Cantelli Lemma]
\label{borelCantelliLemma}
    Let $A_1,A_2,\ldots$ be events and define $A^*=\limsup_{n\to\infty} A_n$. Then
    \begin{enumerate}[(a)]
        \item If $\sum_{n=1}^\infty \Pr[A_n]<\infty$, $\Pr[A^*]=0$.
        \item If $(A_n)_{n\in\mathbb{N}}$ is independent and $\sum_{n=1}^\infty \Pr[A_n]=\infty$, then $\Pr[A^*]=1$.
    \end{enumerate}
\end{theorem}
\begin{proof}
    By \ref{tripledoubleEquivalence}, $\textbf{P}$ is upper semicontinuous, lower semicontinuous and $\sigma$-subadditive.
    \begin{enumerate}
        \item As $\textbf{P}$ is upper semicontinuous and $\sigma$-subadditive,
        \begin{align*}
            \Pr[A^*] &= \lim_{n\to\infty}\Pr\left[\bigcup_{m=n}^\infty A_m\right] \\
            &\leq \lim_{n\to\infty} \Pr[A_m] = 0
        \end{align*}
        The result follows.
        
        \item As $\textbf{P}$ is lower semicontinuous and the family $(A_n)_{n\in\mathbb{N}}$ is independent, we have
        \begin{align*}
            \Pr[(A^*)^c] &= \Pr\left[\bigcup_{n=1}^\infty\bigcap_{m=n}^\infty A_m^c\right] \\
            &= \lim_{n\to\infty} \Pr\left[\bigcap_{m=n}^\infty A_m^c\right] \\
            &= \lim_{n\to\infty} \prod_{m=n}^\infty \left(1-\Pr[A_m]\right) \\
        \end{align*}
        Now for any $n\in\mathbb{N}$, as $\log(1-x)\leq -x$
        \begin{align*}
            \prod_{m=n}^\infty \left(1-\Pr[A_m]\right)&= \exp\left(\sum_{m=n}^\infty \log(1-\Pr[A_m])\right) \\
            &\leq \exp\left(-\sum_{m=n}^\infty \Pr[A_m]\right) = 0
        \end{align*}
        The result follows.
    \end{enumerate}
\end{proof}

The reader may verify using the Borel-Cantelli Lemma that if we roll a die an infinite number of times, the probability of rolling a four an infinite number of times is $1$ by considering the events $A_n = \{\omega\in\Omega:\omega_n=6\}$ for each $n\in\mathbb{N}$ where $\Omega=\{1,\ldots,6\}^\mathbb{N}$.

\vspace{2mm}
We now extend the definition of independence of a family of events as follows.

\begin{definition}[Independence of classes of events]
\label{independence of classes of events}
    Let $I$ be an index set and $\mathcal{E}_i\subseteq\mathcal{A}$ for all $i\in I$. The family $(\mathcal{E}_i)_{i\in I}$ is called \textit{independent} if for any finite $J\subseteq I$ and any choice of $E_j\in\mathcal{E}_j$ and $j\in J$, the family $(E_j)_{j\in J}$ is independent.
\end{definition}

For example, if we roll a die an infinite number of times, for each $i\in\mathbb{N}$, consider the class of events given by $\mathcal{E}_i=\{\{\omega\in\Omega:\omega_i\in A\}:A\subseteq\{1,\ldots,6\}\}$
where $\Omega=\{1,\ldots,6\}^\mathbb{N}$. Then the family $(\mathcal{E}_i)_{i\in I}$ is independent.

\begin{theorem}
~
    Let $I$ be an index set and for each $i\in I$, let $\mathcal{E}_i\subseteq \mathcal{A}$. Then
    \begin{enumerate}[(a)]
        \item Let $I$ be finite. If $\Omega\in\mathcal{E}_i$ for each $i$, then $(\mathcal{E}_i)_{i\in I}$ is independent if and only if $(E_i)_{i\in I}$ is independent for any choice of $E_i\in\mathcal{E}_i,i\in I$.
        
        \item If $(\mathcal{E}_i\cup\{\emptyset\})$ is $\cap$-closed for each $i$, then $(\mathcal{E}_i)_{i\in I}$ is independent if and only if $(\sigma(\mathcal{E}_i))_{i\in I}$ is independent.
        
        % \item Let $K$ be an arbitrary set and let $(I_k)_{k\in K}$ be mutually disjoint subsets of $I$. If $(\mathcal{E}_i)_{i\in I}$ is independent, then $\left(\bigcup_{i\in I_k}\mathcal{E}_i\right)_{k\in K}$ is also independent.
    \end{enumerate}
\end{theorem}

\begin{proof}
    ~
    \begin{enumerate}[(a)]
        \item The forward implication is obvious from the definition. To prove the backward implication, for $J\subseteq I$ and $j\in I\setminus J$, choose $E_j=\Omega$.
        
        \item The backward implication is obvious. Let us now prove the forward implication.
        
        First, we claim that for any $J\subseteq J'\subseteq I$ where $J$ is finite, 
        $$\Pr\left[\bigcap_{i\in J'}E_i\right] = \prod_{i\in J'}\Pr[E_i]$$
        for any choice of $E_i\in\sigma(\mathcal{E}_i)$ if $i\in J$ and
        $E_i\in\mathcal{E}_i$ if $i\in J'\setminus J$.
        
        We shall prove the above claim by induction on $|J|$. If $|J|=0$, then the claim is true as $(\mathcal{E}_i)_{i\in I}$ is independent. Now assume that the claim is true for all $J\subseteq I$ with $|J|=n$ and all finite $J'\supseteq J$. Fix such a $J$. Let $j\in I\setminus J$. Define $\tilde J=J\cup\{j\}$ and choose some $J'\supseteq\tilde J$. We shall show that the claim is true if we replace $J$ with $\tilde J$, thus proving the inductive step.
    
        Fix $E_i\in\sigma(\mathcal{E}_i)$ for each $i\in J$ and $E_i\in\mathcal{E}_i$ for each $i\in J'\setminus\tilde J$. Consider measures $\mu,\nu$ on $(\Omega,\mathcal{A})$ such that
        \begin{align*}
            \mu &: E_j\mapsto \Pr\left[\bigcap_{i\in J'}E_i\right] \\
            \nu &: E_j\mapsto \prod_{i\in J'}\Pr[E_i]
        \end{align*}
        By the induction hypothesis, $\mu(E_j)=\nu(E_j)$ for all $E_j\in\mathcal{E}_j\cup\{\emptyset,\Omega\}$. As $\mathcal{E}_j\cup\{\emptyset\}$ is $\cap$-closed, \ref{uniquely defined by base pi sys} implies that $\mu(E_j)=\nu(E_j)$ for all $E_j\in\sigma(\mathcal{E}_j)$.
        
        This proves our claim. Setting $J=J'$ yields the required result.
    
    \end{enumerate}
\end{proof}

\section{Independence}

\subsection{Independent Events}

In the following, let $(\Omega,\mathcal{A},\textbf{P})$ be a probability space and the sets $A\in\mathcal{A}$ be events.
    
\begin{definition}
    Two events $A$ and $B$ are said to be \textit{independent} if
    $$\Pr[A\cap B] = \Pr[A]\Pr[B].$$
\end{definition}

For example, if we roll a die twice, the event of rolling a $6$ the first time is independent of the event of rolling a $6$ the second time.

Here, $\Omega=[6]^2, \mathcal{A}=2^\Omega$ and the probability distribution is $\textbf{P}=\mathcal{U}_\Omega$. Our claim may be verified as follows. Towards showing that the outcome of the first roll is independent of that of the second, let $\tilde A,\tilde B\subseteq\Omega, A=\tilde A\times\Omega$ and $B=\Omega\times\tilde B$. We must show that $\Pr[A]\Pr[B]=\Pr[A\cap B]$. This is obvious as follows: 
\begin{align*}
    \Pr[A] &= \frac{|A|}{36}=\frac{|\tilde A|}{6} \\
    \Pr[B] &= \frac{|B|}{36}=\frac{|\tilde B|}{6} \\
    \Pr[A\cap B] &= \frac{|A\cap B|}{36} = \frac{|\tilde A||\tilde B|}{36} = \Pr[A]\Pr[B].
\end{align*}

While in the above example it is intuitively clear that the two events must be independent, we can have less obvious examples as well. For example, the event that the sum of the two rolls is odd and the event that the first roll gives at most a three are independent. We leave it to the reader to verify this claim.

\vspace{2mm}
We extend this definition of two independent events to any number of independent events as follows.

\begin{definition}[Independence of Events]
    Let $I$ be an index set and $(A_i)_{i\in I}$ be a family of events. The family $(A_i)_{i\in I}$ is called \textit{independent} if for any finite subset $J\subseteq I$, the following holds:
    $$\Pr\left[\bigcap_{j\in J}A_j\right]=\prod_{j\in J}\Pr[A_j].$$
\end{definition}

Note that pairwise independence does not guarantee overall independence. For example, For example, consider $(X,Y,Z)$ chosen uniformly from $\{(0,0,0),(1,1,0),(1,0,1),(0,1,1)\}$. Then $X,Y,Z$ are pairwise independent but they are not overall independent.

\vspace{2mm}
Let us now return to the product measure defined in \cref{defProductMeasure}, which can be understood intuitively as follows. If $E$ is a finite set of outcomes, consider the probability space comprising $\Omega=E^\mathbb{N}$, the $\sigma$-algebra
$$\mathcal{A}=\sigma([\omega_1,\ldots,\omega_n]:\omega_1,\ldots,\omega_n\in E\text{ and }n\in\mathbb{N})$$
and the product measure $\textbf{P}=\left(\sum_{e\in E}p_e\delta_e\right)^{\otimes\mathbb{N}}$. This basically represents that we repeatedly conduct the experiment of choosing an outcome from $E$. Let $\tilde A_i\subseteq E$ for each $i\in\mathbb{N}$ and let $A_i$ be the event such that $\tilde A_i$ occurs in the $i$th experiment, given by
$$A_i = \{\omega\in\Omega:\omega_i\in\tilde A_i\}
  =\biguplus_{(\omega_1,\ldots,\omega_i)\in E^{i-1}\times \tilde A_i} [\omega_1,\ldots,\omega_i]$$

Intuitively, the family $(A_i)_{i\in\mathbb{N}}$ should be independent, since the outcome of one of the conducted experiments does not depend on the outcomes of the other experiments.

We shall check this. Let $J\subseteq\mathbb{N}$ and For $j\in J$, let $B_j=A_j$ and $\tilde B_j=\tilde A_j$ and for $j\in[n]\setminus J$, let $B_j=\Omega$ and $\tilde B_j=E$. Then
\begin{align*}
    \Pr\left[\bigcap_{j\in J}A_j\right] &= \Pr\left[\bigcap_{j=1}^n B_j\right] \\
    &= \Pr\left[\left\{\omega\in\Omega:\omega_j\in\tilde B_j\text{ for each }j\in[n]\right\}\right] \\
    &= \sum_{e_1\in\tilde B_1} \cdots \sum_{e_n\in\tilde B_n} \prod_{j=1}^n p_{e_j} \\
    &= \prod_{j=1}^n \left(\sum_{e\in\tilde B_j}p_e\right) \\
    &= \prod_{j\in J} \left(\sum_{e\in\tilde A_j}p_e\right)
\end{align*}

In particular, as this is true for $|J|=1$, we have for some fixed $i\in[n]$,
$$\Pr[A_i] = \left(\sum_{e\in\tilde A_i}p_e\right).$$
Substituting the above, we have
$$
\Pr\left[\bigcap_{j\in J}A_j\right] 
= \prod_{j\in J} \left(\sum_{e\in\tilde A_j}p_e\right)
= \prod_{j\in J} \Pr[A_j].
$$
This proves the result. We state the result for future reference as follows.

\begin{theorem}
\label{product measure independence}
    Let $E$ be a finite set. Consider the probability space $(\Omega,\mathcal{A},\textbf{P})$ where $\Omega=E^\mathbb{N}$,
    $$\mathcal{A}=\sigma([\omega_1,\ldots,\omega_n]:\omega_1,\ldots,\omega_n\in E\text{ and }n\in\mathbb{N})$$
    and $\textbf{P}=\left(\sum_{e\in E}p_e\delta_e\right)^{\otimes\mathbb{N}}$. For each $i\in\mathbb{N}$, let $\tilde A_i\subseteq E$ and $A_i$ be the event such that $\tilde A_i$ occurs in the $i$th experiment, given by
    $$A_i = \{\omega\in\Omega:\omega_i\in\tilde A_i\}
    =\biguplus_{(\omega_1,\ldots,\omega_i)\in E^{i-1}\times \tilde A_i} [\omega_1,\ldots,\omega_i]$$
  Then the family $(A_i)_{i\in\mathbb{N}}$ is independent.
\end{theorem}


\vspace{2mm}
Note that if events $A$ and $B$ are independent, then the events $A^c$ and $B$ are independent as well. This can be described more precisely as follows.

\begin{theorem}
    Let $I$ be an index set and $(A_i)_{i\in I}$ be a family of events. Define $B_i^0 = A_i$ and $B_i^1 = A_i^c$ for each $i\in I$. Then the following statements are equivalent.
    \begin{enumerate}[(a)]
        \item The family $(A_i)_{i\in I}$ is independent.
        \item There is some $\alpha\in\{0,1\}^I$ such that the family $(B_i^{\alpha_i})_{i\in I}$ is independent.
        \item For all $\alpha\in\{0,1\}^I$, the family $(B_i^{\alpha_i})_{i\in I}$ is independent.
    \end{enumerate}
\end{theorem}

We leave the proof of the above to the reader.

Now, recall the limit superior defined in \cref{defLimes}. The limit superior represents the event that a particular event occurs an infinite amount of times (for example, the event that we roll a $4$ an infinite number of times when we roll a die a countably infinite number of times). This is formalized in the following.

\begin{ftheo}[Borel-Cantelli Lemma]
\label{borelCantelliLemma}
    Let $A_1,A_2,\ldots$ be events and let $A^*=\limsup_{n\to\infty} A_n$. Then
    \begin{enumerate}[(a)]
        \item If $\sum_{n=1}^\infty \Pr[A_n]<\infty$, $\Pr[A^*]=0$.
        \item If $(A_n)_{n\in\mathbb{N}}$ is independent and $\sum_{n=1}^\infty \Pr[A_n]=\infty$, then $\Pr[A^*]=1$.
    \end{enumerate}
\end{ftheo}
\begin{proof}
    By \cref{tripledoubleEquivalence}, $\textbf{P}$ is upper semicontinuous, lower semicontinuous and $\sigma$-subadditive.
    \begin{enumerate}[(a)]
        \item As $\textbf{P}$ is upper semicontinuous and $\sigma$-subadditive,
        \begin{align*}
            \Pr[A^*] &= \Pr\left[\bigcap_{n=1}^\infty\bigcup_{m=n}^\infty A_m\right] \\
            \lim_{n\to\infty}\Pr\left[\bigcup_{m=n}^\infty A_m\right] \\
            &\leq \lim_{n\to\infty} \sum_{m=n}^\infty \Pr[A_m] = 0
        \end{align*}
        The result follows.
        
        \item As $\textbf{P}$ is lower semicontinuous and the family $(A_n)_{n\in\mathbb{N}}$ is independent, we have
        \begin{align*}
            \Pr[(A^*)^c] &= \Pr\left[\bigcup_{n=1}^\infty\bigcap_{m=n}^\infty A_m^c\right] \\
            &= \lim_{n\to\infty} \Pr\left[\bigcap_{m=n}^\infty A_m^c\right] \\
            &= \lim_{n\to\infty} \prod_{m=n}^\infty \left(1-\Pr[A_m]\right) \\
        \end{align*}
        Now for any $n\in\mathbb{N}$, as $\log(1-x)\leq -x$
        \begin{align*}
            \prod_{m=n}^\infty \left(1-\Pr[A_m]\right)&= \exp\left(\sum_{m=n}^\infty \log(1-\Pr[A_m])\right) \\
            &\leq \exp\left(-\sum_{m=n}^\infty \Pr[A_m]\right) = 0
        \end{align*}
        The result follows.
    \end{enumerate}
\end{proof}

A saying the reader might have come across is that if a monkey is left with a typewriter for an infinite amount of time, it will eventually type the complete works of Shakespeare. This is in fact a consequence of the Borel-Cantelli Lemma, and is often referred to as the \href{https://en.wikipedia.org/wiki/Infinite_monkey_theorem}{Infinite Monkey Theorem}!

\vspace{2mm}
We now extend the definition of independence of a family of events as follows.

\begin{definition}[Independence of classes of events]
\label{independence of classes of events}
    Let $I$ be an index set and $\mathcal{E}_i\subseteq\mathcal{A}$ for all $i\in I$. The family $(\mathcal{E}_i)_{i\in I}$ is called \textit{independent} if for any finite $J\subseteq I$ and any choice of $j\in J$ and $E_j\in\mathcal{E}_j$, the family $(E_j)_{j\in J}$ is independent.
\end{definition}

For example, if we roll a die an infinite number of times, for each $i\in\mathbb{N}$, consider the class of events given by $\mathcal{E}_i=\{\{\omega\in\Omega:\omega_i\in A\}:A\subseteq[6]\}$
where $\Omega=[6]^\mathbb{N}$. Then the family $(\mathcal{E}_i)_{i\in I}$ is independent.

\begin{theorem}
\label{independent set classes subset}
~
    Let $I$ be an index set and for each $i\in I$, let $\mathcal{E}_i\subseteq \mathcal{A}$. Then
    \begin{enumerate}[(a)]
        \item Let $I$ be finite. If $\Omega\in\mathcal{E}_i$ for each $i$, then $(\mathcal{E}_i)_{i\in I}$ is independent if and only if $(E_i)_{i\in I}$ is independent for any choice of $E_i\in\mathcal{E}_i, i\in I$.
        
        \item If $(\mathcal{E}_i\cup\{\emptyset\})$ is $\cap$-closed for each $i$, then $(\mathcal{E}_i)_{i\in I}$ is independent if and only if $(\sigma(\mathcal{E}_i))_{i\in I}$ is independent.
        
        % \item Let $K$ be an arbitrary set and let $(I_k)_{k\in K}$ be mutually disjoint subsets of $I$. If $(\mathcal{E}_i)_{i\in I}$ is independent, then $\left(\bigcup_{i\in I_k}\mathcal{E}_i\right)_{k\in K}$ is also independent.
    \end{enumerate}
\end{theorem}

\begin{proof}
    ~
    \begin{enumerate}[(a)]
        \item The forward implication is obvious from the definition. To prove the backward implication, for $J\subseteq I$ and $j\in I\setminus J$, choose $E_j=\Omega$.
        
        \item The backward implication is obvious. Let us now prove the forward implication.
        
        First, we claim that for any $J\subseteq J'\subseteq I$ where $J$ is finite, 
        $$\Pr\left[\bigcap_{i\in J'}E_i\right] = \prod_{i\in J'}\Pr[E_i]$$
        for any choice of $E_i\in\sigma(\mathcal{E}_i)$ if $i\in J$ and
        $E_i\in\mathcal{E}_i$ if $i\in J'\setminus J$.
        
        We shall prove the above claim by induction on $|J|$. If $|J|=0$, then the claim is true as $(\mathcal{E}_i)_{i\in I}$ is independent. Now assume that the claim is true for all $J\subseteq I$ with $|J|=n$ and all finite $J'\supseteq J$. Fix such a $J$. Let $j\in I\setminus J$. Define $\tilde J=J\cup\{j\}$ and choose some $J'\supseteq\tilde J$. We shall show that the claim is true if we replace $J$ with $\tilde J$, thus proving the inductive step.
    
        Fix $E_i\in\sigma(\mathcal{E}_i)$ for each $i\in J$ and $E_i\in\mathcal{E}_i$ for each $i\in J'\setminus\tilde J$. Consider measures $\mu,\nu$ on $(\Omega,\mathcal{A})$ such that
        \begin{align*}
            \mu &: E_j\mapsto \Pr\left[\bigcap_{i\in J'}E_i\right] \\
            \nu &: E_j\mapsto \prod_{i\in J'}\Pr[E_i]
        \end{align*}
        By the induction hypothesis, $\mu(E_j)=\nu(E_j)$ for all $E_j\in\mathcal{E}_j\cup\{\emptyset,\Omega\}$. As $\mathcal{E}_j\cup\{\emptyset\}$ is $\cap$-closed, \cref{uniquely defined by base pi sys} implies that $\mu(E_j)=\nu(E_j)$ for all $E_j\in\sigma(\mathcal{E}_j)$.
        
        This proves our claim. Setting $J=J'$ yields the required result.
    
    \end{enumerate}
\end{proof}

\subsection{Independence of Random Variables}

Let $I$ be an index set and $(\Omega,\mathcal{A})$ be measurable space. For each $i\in I$, let $(\Omega_i,\mathcal{A}_i)$ be a measurable space and $X_i:(\Omega,\mathcal{A})\to(\Omega_i,\mathcal{A}_i)$ be a random variable with generated $\sigma$-algebra $\sigma(X_i)$.

\begin{definition}
    The family $(X_i)_{i\in I}$ of random variables is said to be independent if the family $(\sigma(X_i))_{i\in I}$ of generated $\sigma$-algebras is independent.
\end{definition}

We say that a family $(X_i)_{i\in I}$ of random variables is said to be i.i.d. (independent and identically distributed) if the family $(X_i)_{i\in I}$ is independent and $\textbf{P}_{X_i}=\textbf{P}_{X_j}$ for any $i,j\in I$.

\vspace{2mm}
The meaning of independence of random variables might be more clear from the following restructuring of the definition. A family $(X_i)_{i\in I}$ of random variables is independent if and only if for any finite set $J\subseteq I$ and any choice of $A_j\in\mathcal{A}_j,j\in J$, we have
$$\Pr\left[\bigcap_{j\in J}\{X_j\in A_j\}\right] = \prod_{j\in J}\Pr[X_j\in A_j].$$

\vspace{1mm}
For example, let us flip a coin four times. Let $X$ be the random variable be the number of heads that show up in the first two tosses and $Y$ be the random variable given by the number of tails that show up in the next two tosses. Then $X$ and $Y$ are independent.

\vspace{2mm}
For each $i\in I$, let $(\Omega_i',\mathcal{A}_i')$ be another measurable space and assume that $f_i:(\Omega,\mathcal{A})\to(\Omega_i',\mathcal{A}_i')$ is a measurable map. If $(X_i)_{i\in I}$ is independent, then $(f_i\circ X_i)_{i\in I}$ is independent as well. This is a consequence of the fact that $f_i\circ X_i$ is $\sigma(X_i)-\mathcal{A}_i'$-measurable.

\begin{theorem}
\label{independent if generators independent}
    For each $i\in I$, let $\mathcal{E}_i$ be a $\pi$-system that generates $\mathcal{A}$. If $(X^{-1}(\mathcal{E}_i))$ is independent, then $(X_i)_{i\in I}$ is independent.
\end{theorem}
\begin{proof}
    By \cref{generating pi system fixed under preimage}, $X^{-1}(\mathcal{E}_i)$ is a $\pi$-system that generates $X^{-1}(\mathcal{A}_i)=\sigma(X_i)$. The result follows from \cref{independent set classes subset}.
\end{proof}

\begin{theorem}
    Let $E$ be a finite set $(p_e)_{e\in E}$ be a probability vector on $E$. There then exists a probability space $(\Omega,\mathcal{A},\textbf{P})$ and an independent family $(X_n)_{n\in\mathbb{N}}$ of $E$-valued random variables on $(\Omega,\mathcal{A},\textbf{P})$ such that $\Pr[X_n = e] = p_e$ for each $e\in E$.
\end{theorem}
\begin{proof}
    We shall prove this by constructing the required probability space $(\Omega,\mathcal{A},\textbf{P})$. Let $\Omega=E^\mathbb{N}$ and $\mathcal{A}=\sigma\left(\{[\omega_1,\omega_2,\ldots,\omega_k]:\omega_i\in E\text{ for each }i\in[k]\text{ and }k\in\mathbb{N}\}\right)$. Let $\textbf{P}=\left(\sum_{e\in E}p_e\delta_e\right)^{\otimes\mathbb{N}}$ be the product measure. For each $n\in\mathbb{N}$, define the random variable $X_n:\Omega\to E$ by $\omega\mapsto \omega_n$, where $\omega_n$ represents the $n$th coordinate of $\omega$.
    
    As a consequence of \cref{product measure independence}, $(X_j)_{j\in\mathbb{N}}$ is independent. and the result is proved.
\end{proof}

\begin{definition}
    Let $I$ be an index set and for each $i\in I$, let $X_i$ be a random variable. For any $J\subseteq I$, let $F_{(X_j)_{j\in J}}:\mathbb{R}^J\to[0,1]$ be given by
    $$x\mapsto \Pr[X_j\leq x_j\text{ for each }j\in J]=\Pr\left[\bigcap_{j\in J} X_j\leq x_j\right].$$
    This function is called the \textit{joint distribution function of $(X_j)_{j\in J}$} and is denoted $F_J$. The probability measure $\textbf{P}_{(X_j)_{j\in J}}$ on $\mathbb{R}^J$ is called the \textit{joint distribution of $(X_j)_{j\in J}$}.
\end{definition}

\begin{theorem}
    A family $(X_i)_{i\in I}$ of random variables is independent if and only if for every finite $J\subseteq I$ and $x\in\mathbb{R}^J$,
    $$F_J(x) = \prod_{j\in J}F_{\{j\}}(x_j).$$
\end{theorem}
\begin{proof}
    The forward implication is obvious.
    
    The class of sets $\{(-\infty,a]:a\in\mathbb{R}\}$ is a $\cap$-closed generator of $\mathcal{B}(\mathbb{R})$. The given condition is equivalent to saying that the events $\{X_j\in(-\infty,x_j]\}$ are independent. By \cref{independent if generators independent}, the backward implication is proved.
\end{proof}

\begin{corollary}
    If in addition to the conditions of the previous theorem, each $F_J$ has a continuous density function $f_J=f_{(X_j)_{j\in J}}$, then the family $(X_i)_{i\in I}$ is independent if and only if for any finite $J\subseteq I$ and $x\in\mathbb{R}^J$,
    $$f_J(x) = \prod_{j\in J}f_{\{j\}}(x_j).$$
\end{corollary}

% \begin{theorem}
%     Let $K$ be an arbitrary set and $I_k,k\in K$ be mutually disjoint sets. Define $I=\bigcup_{k\in K}I_k$. If a family $(X_i)_{i\in I}$ of random variables is independent, then the family of $\sigma$-algebras $(\bigcup_{i\in I_k}\sigma(X_i))_{k\in K}$ is independent.
% \end{theorem}
% \begin{proof}
    
% \end{proof}

\subsection{The Convolution}

\begin{definition}
    Let $\mu$ and $\nu$ be probability measures on $(\mathbb{Z},2^\mathbb{Z})$. The \textit{convolution $(\mu * \nu)$} is defined as the probability measure on $(\mathbb{Z},2^\mathbb{Z})$ given by
    $$(\mu * \nu) (\{n\}) = \sum_{m=-\infty}^\infty \mu(\{m\})\nu(\{n-m\}).$$
\end{definition}

We define the $n$th convolution power by $\mu^{*1}=\mu$ and $\mu^{*n}=\mu^{*(n-1)}*\mu$.

\begin{theorem}
    If $X$ and $Y$ are independent $\mathbb{Z}$-valued random variables, then $\textbf{P}_{X+Y}=\textbf{P}_X * \textbf{P}_Y$.
\end{theorem}
\begin{proof}
    For any $n\in\mathbb{Z}$, we have
    \begin{align*}
        \textbf{P}_{X+Y}[\{n\}] &= \Pr[X+Y=n] \\
        &= \Pr\left[\biguplus_{m=-\infty}^\infty \{X=m\}\cap\{Y=n-m\}\right] \\
        &= \sum_{m=-\infty}^\infty \Pr[X=m]\Pr[Y=n-m] \\
        &= \sum_{m=-\infty}^\infty \textbf{P}_X[\{m\}]\textbf{P}_Y[\{n-m\}] = (\textbf{P}_X * \textbf{P}_Y) [\{n\}]
    \end{align*}
\end{proof}

Given the above theorem, we can generalise the convolution as follows.

\begin{definition}[Convolution of Probability Measures]
    Let $X$ and $Y$ be independent random variables on $\mathbb{R}^n$ such that $\mu=\textbf{P}_X$ and $\nu=\textbf{P}_Y$. The \textit{convolution $(\mu * \nu)$} is defined as $\textbf{P}_{X+Y}$.
\end{definition}

We define $\mu^{*k}$ for $k\in\mathbb{N}$ recursively similarly to the first case with $\mu^{*0}=\delta_0$.

\vspace{2mm}
For example, let $\lambda,\mu\in[0,\infty)$. Consider independent random variables $X,Y$ such that $X\sim\Poi_\mu$ and $Y\sim\Poi_\lambda$. Then for $n\in\mathbb{N}_0$
\begin{align*}
    \Pr[X+Y=n] &= e^{-\mu}e^{-\lambda}\sum_{m=0}^n \frac{\mu^m}{m!}\frac{\lambda^m}{(n-m)!} \\
    &= e^{-(\mu+\lambda)}\frac{(\mu+\lambda)^n}{n!}.
\end{align*}
Thus $\Poi_\lambda * \Poi_\mu = \Poi_{\lambda+\mu}$.

\subsection{Kolmogorov's \texorpdfstring{$0$}{TEXT}-\texorpdfstring{$1$}{TEXT} Law}

The \hyperref[borelCantelliLemma]{Borel-Cantelli Lemma} was an example of a so-called $0-1$ law. In this subsection, we study another such $0$-$1$ law.

\begin{definition}[Tail $\sigma$-algebra]
    Let $I$ be a countably infinite index set and $(\mathcal{A}_i)_{i\in I}$ be a family of $\sigma$-algebras. Then
    $$\mathcal{T}((\mathcal{A}_i)_{i\in I})=\bigcap_{\substack{J\subseteq I \\ |J|<\infty}} \sigma\left(\bigcup_{j\in I\setminus J}\mathcal{A}_j\right)$$
    is called the \textit{tail $\sigma$-algebra} of $(\mathcal{A}_i)_{i\in I}$. If $(A_i)_{i\in I}$ is a family of events, we define
    $$\mathcal{T}((A_i)_{i\in I}) = \mathcal{T}((\{\emptyset,A_i,A_i^c,\Omega\})_{i\in I}).$$
    If $(X_i)_{i\in I}$ is a family of random variables, we define
    $$\mathcal{T}((X_i)_{i\in I}) = \mathcal{T}((\sigma(X_i))_{i\in I}).$$
\end{definition}

If the meaning is clear from context, we represent the tail $\sigma$-algebra as just $\mathcal{T}$.

Intuitively, the above means that we consider those events that are independent of the values of any finite subfamily of $(X_i)_{i\in I}$. This shall be made clearer as follows.

\begin{theorem}
    Let $J_1,J_2,\ldots$ be finite sets with $J_n\uparrow I$. Then
    $$\mathcal{T}((\mathcal{A}_i)_{i\in I})=\bigcap_{n=1}^\infty \sigma\left(\bigcup_{m\in I\setminus J_n} \mathcal{A}_m\right).$$
    If $I=\mathbb{N}$, then this says that
    $$\mathcal{T}((\mathcal{A}_i)_{i\in\mathbb{N}}) = \bigcap_{n=1}^\infty \sigma\left(\bigcup_{m=n}^\infty \mathcal{A}_m\right).$$
\end{theorem}
\begin{proof}
    It is obvious from the definition that $\mathcal{T}((\mathcal{A}_i)_{i\in I})$ is a subset of the expression on the right.
    
    Let $J_n\uparrow I$ and $J\subseteq I$ be a finite set. There exists some $n_0\in\mathbb{N}$ such that $J\subseteq J_{n_0}$. We then have
    \begin{align*}
        \bigcap_{n=1}^\infty \sigma\left(\bigcup_{m\in I\setminus J_n} \mathcal{A}_m\right) &\subseteq \bigcap_{n=1}^N \sigma\left(\bigcup_{m\in I\setminus J_n} \mathcal{A}_m\right) \\
        &= \sigma\left(\bigcup_{m\in I\setminus J_N} \mathcal{A}_m\right) \\
        &\subseteq \sigma\left(\bigcup_{m\in I\setminus J} \mathcal{A}_m\right)
    \end{align*}
    Noting that the expression on the left does not depend on $J$ and taking the intersection over all $J$ implies the reverse inclusion and the result follows.
\end{proof}

If we interpret $I=\mathbb{N}$ as a set of times, the above theorem essentially says that any event in $\mathcal{T}$ is independent of the first finitely many time points.

\vspace{2mm}
Now, it is not immediately clear whether $\mathcal{T}$ even contains any nontrivial events (events other than $\emptyset$ and $\Omega$).

\vspace{2mm}
For starters, if $A_1,A_2,\ldots$ are events, then $A^*=\limsup_{n\to\infty} A_n$ and $A_*=\liminf_{n\to\infty} A_n$ are both in $\mathcal{T}((A_i)_{i\in \mathbb{N}})$.

\vspace{1mm}
To see this for $A_*$, define $B_n=\bigcap_{m=n}^\infty A_m$ for $n\in\mathbb{N}$. We then have that $B_n\uparrow A_*$ and $B_n\in\sigma(\bigcup_{m=n_0}^\infty A_m)$ for any $n\geq n_0$. This implies that $A_*\in\sigma(\bigcup_{m=n_0}^\infty A_m)$ for any $n_0\in\mathbb{N}$ and thus, $A_*\in\mathcal{T}$.

\vspace{1mm}
For $A^*$, we must show that for any $m\in\mathbb{N}$, $\limsup_{n\to\infty}A_n \in \sigma(\bigcup_{n=m}^\infty \sigma(A_n))$. This is true as $\limsup_{n\to\infty}A_n=\limsup_{n\to\infty}A_{n+m}$. Since each $A_{n+m}$ is in $\sigma(\bigcup_{n=m}^\infty \sigma(A_{n}))$, the statement is true.

\vspace{2mm}
Let $(X_n)_{n\in\mathbb{N}}$ be real random variables. Then the Ces\`{a}ro limits
$$\liminf_{n\to\infty} \frac{1}{n}\sum_{i=1}^n X_i\text{ and }\limsup_{n\to\infty} \frac{1}{n}\sum_{i=1}^n X_i$$
are $\mathcal{T}((X_n)_{n\in\mathbb{N}})$-measurable.

% To show this, choose some $n_0\in\mathbb{N}$ and note that
% $$X_*=\liminf_{n\to\infty}\frac{1}{n}\sum_{i=1}^n X_i=\liminf_{n\to\infty}\frac{1}{n}\sum_{i=n_0}^n X_i$$
% is $(\sigma((X_n)_{n\geq n_0}))$-measurable. 

\begin{ftheo}[Kolmogorov's $0$-$1$ Law]
    Let $I$ be a countable infinite index set and $((\mathcal{A}_i)_{i\in I})$ be an independent family of $\sigma$-algebras. Then the tail $\sigma$-algebra is $\textbf{P}$-trivial, that is,
    $$\Pr[A]\in\{0,1\}\text{ for any }A\in\mathcal{T}((\mathcal{A}_i)_{i\in I}).$$
\end{ftheo}
\begin{proof}
    Assume w.l.o.g. that $I=\mathbb{N}$. For each $i\in\mathbb{N}$, let
    $$\mathcal{F}_n = \left\{\bigcap_{i=1}^n A_k: A_j\in\mathcal{A}_j\text{ for each }j\in[n]\right\}.$$
    Let $\mathcal{F}=\bigcup_{i=1}^\infty \mathcal{F}_i$. Note that $\mathcal{F}$ is a semiring.
    
    \vspace{1mm}
    Further, for any $m\in\mathbb{N}$ and $A_m\in\mathcal{A}_m$, we have $A_m\in\mathcal{F}$. This implies that $\sigma\left(\bigcup_{i=1}^\infty \mathcal{A}_i\right)\subseteq\sigma(\mathcal{F})$. We also have
    $$\mathcal{F}_m\subseteq\sigma\left(\bigcup_{i=1}^m \mathcal{A}_i\right)\subseteq\sigma\left(\bigcup_{i=1}^\infty \mathcal{A}_i\right).$$
    This implies that $\mathcal{F} = \sigma\left(\bigcup_{i=1}^\infty \mathcal{A}_i\right)$.
    
    Let $A\in\mathcal{T}((\mathcal{A}_n)_{n\in\mathbb{N}})$ and $\varepsilon>0$. By \cref{Approximation Thm for Measures}, there exists $n_0\in\mathbb{N}$ and mutually disjoint sets $F_1,F_2,\ldots,F_{n_0}$ such that $$\Pr\left[A\triangle \bigcup_{i=1}^{n_0}F_i\right]<\varepsilon.$$
    Let $F=\bigcup_{i=1}^{n_0}F_i$. There must be some $n\in\mathbb{N}$ such that $F_1,\ldots,F_{n_0}\in \mathcal{F}_n$. This implies that $F\in\sigma(\bigcup_{i=1}^n \mathcal{A}_i)$. By the definition of the tail $\sigma$-algebra, $A\in\sigma(\bigcup_{i=n+1}^\infty\mathcal{A}_i)$ so $A$ must be independent of $F$. Therefore,
    \begin{align*}
        \varepsilon &> \Pr[A\setminus F] \\
        &= \Pr[A](1-\Pr[F]) \\
        &\geq \Pr[A](1-\Pr[A]-\varepsilon).
    \end{align*}
    As this is true for any $\varepsilon>0$, $0=\Pr[A](1-\Pr[A])$ and the result is proved.
\end{proof}

\begin{corollary}
    Let $(A_n)_{n\in\mathbb{N}}$ be an independent family of events. Then
    $$\Pr\left[\limsup_{n\to\infty} A_n\right]\text{ and }\Pr\left[\liminf_{n\to\infty} A_n\right]\text{ are in }\{0,1\}.$$
\end{corollary}

The above can be inferred from the fact that the $\limsup$ and $\liminf$ lie in the tail $\sigma$-algebra. It also follows from the \hyperref[borelCantelliLemma]{Borel-Cantelli Lemma}.

\begin{corollary}
    Let $(X_n)_{n\in\mathbb{N}}$ be an independent family of $\overline{\mathbb{R}}$-valued random variables. Then $X_*=\liminf_{n\to\infty}X_n$ and $X^*=\limsup_{n\to\infty} X_n$ are almost surely constant, that is, there exist $x_*,x^*\in\overline{\mathbb{R}}$ such that $\Pr[X_*=x_*]=1$ and $\Pr[X^*=x^*]=1$.
\end{corollary}

The above follows from the fact that for any $x\in\overline{\mathbb{R}}$, $\{X_*<x\}\in\mathcal{T}((X_n)_{n\in\mathbb{N}})$ and $\{X^*>x\}\in\mathcal{T}((X_n)_{n\in\mathbb{N}})$.

\clearpage
\section{Generating Functions}

It is a common theme in mathematics to determine relations between objects that are of interest and objects that are easy to compute with. In probability theory, probability generating functions, Laplace transforms and characteristic functions fall in the former category while the mean, median and variance of random variables fall in the latter.

\subsection{Definitions and Basics}

\begin{definition}[Probability Generating Function]
    Let $X$ be an $\mathbb{N}_0$-valued random variable. The \textit{probability generating function} (abbreviated \textit{pgf}) of $\textbf{P}_X$ (or $X$) is the map $\psi_{\textbf{P}_X}=\psi_X:[0,1]\to[0,1]$ is given by (where $0^0=1$)
    $$z\mapsto \sum_{n=0}^\infty \Pr[X=n]z^n.$$
\end{definition}

From the properties of a power series, we have the following result, which we do not prove.

\begin{theorem}
\label{pgfpowerseries}
    ~\begin{enumerate}[(a)]
        \item $\psi_X$ is continuous on $[0,1]$ and infinitely often continuously differentiable on (0,1). For $n\in\mathbb{N}$,
        $$\lim_{z\to 1^-}\psi_X^{(n)}(z)=\sum_{k=n}^\infty \Pr[X=k]\cdot k(k-1)\cdots (k-n+1)$$
        where both sides can equal $\infty$.
        
        \item $\textbf{P}_X$ is uniquely determined by $\psi_X$.
        
        \item For any $r\in(0,1)$, $\psi_X$ is uniquely determined by countably many values $\psi_X(x_i)$ where $x_i\in [0,r]$ for each $i\in\mathbb{N}$. If the series given in $\psi_X$ converges for some $z>1$, then this statement is also true for any $r\in(0,z)$ and 
        $$\lim_{z\to 1^{-}} \psi_{X}^{(n)}(z)=\psi_X^{(n)}(1)<\infty\text{ for }n\in\mathbb{N}.$$
        Here, $\psi_X$ is uniquely determined by $\psi_X^{(n)}(1), n\in\mathbb{N}$.
    \end{enumerate}
\end{theorem}

While this definition of a pgf may seem quite arbitrary, it is useful when we want to add random variables, as is evident from the following theorem.

\begin{theorem}
\label{sum of randvars is prod of pgfs}
    Let $X_1,X_2,\ldots,X_n$ be independent $\mathbb{N}_0$-valued random variables. Then
    $$\psi_{X_1+X_2+\ldots+X_n}=\prod_{i=1}^n \psi_{X_i}.$$
\end{theorem}
\begin{proof}
    We shall prove the claim for $n=2$ and the result will follow inductively. For any $z\in[0,1]$
    \begin{align*}
        \psi_{X_1}(z)\cdot\psi_{X_2}(z) &= \left(\sum_{n=0}^\infty \Pr[X_1=n]z^n\right)\left(\sum_{n=0}^\infty \Pr[X_2=n]z^n\right) \\
        &= \sum_{n=0}^\infty z^n\left(\sum_{m=0}^n\Pr[X_1=m]\Pr[X_2=n-m]\right) \\
        &= \sum_{n=0}^\infty z^n\left(\sum_{m=0}^n\Pr\left[\{X_1=m\}\cap \{X_2=n-m\}\right]\right) \\
        &= \sum_{n=0}^\infty z^n\Pr[X_1+X_2=n] \\
        &= \psi_{X_1+X_2}.
    \end{align*}
\end{proof}

For example,

\begin{itemize}
    \item For some $m,n\in\mathbb{N}$ and $p\in[0,1]$, let $X\sim b_{n,p}$ and $Y\sim b_{m,p}$. Then
    $$\psi_X(z)=\sum_{i=0}^n \binom{n}{i}p^i(1-p)^iz^i=(pz+1-p)^n.$$
    If $X$ and $Y$ are independent, then
    $$\psi_{X+Y}(z)=\psi_X(z)\cdot\psi_Y(z)=(pz+1-p)^{m+n}.$$
    Therefore, $X+Y\sim b_{m+n,p}$, which is not immediately apparent otherwise. (Note that this also implies that $b_{m,p}*b_{n,p}=b_{m+n,p}$)

    \item Let $p\in(0,1]$ and $X_1,\ldots,X_n\sim \gamma_p$ be independent random variables. Define $Y=X_1+\cdots+X_n$. For any $z\in[0,1]$,
    $$\psi_{X_1}(z)=\sum_{i=0}^\infty p(z(1-p))^i=\frac{p}{1-z(1-p)}.$$
    Then
    \begin{align*}
        \psi_Y(z) &= \frac{p^n}{(1-z(1-p))^n} \\
        &= \sum_{i=0}^\infty p^n \binom{-n}{i}(-1)^i(1-p)^i z^i \\
        &= \sum_{i=0}^\infty b^-_{n,p}(\{i\}) z^i.
    \end{align*}
    Therefore, $\gamma_p^{*n}=b^-_{n,p}$. Note that this matches with the intuition we introduced while defining the geometric and negative binomial distributions. The waiting time for the $n$th success is equivalent to waiting for a single success $n$ times, that is, $X_1+\cdots+X_n$.
    
    \item We can also show that for $\lambda,\mu\in[0,\infty)$,
    $$\psi_{\Poi_{\lambda}}(z)=e^{\lambda(z-1)}.$$
    This implies that $\Poi_\lambda * \Poi_{\mu} = \Poi_{\lambda+\mu}$ (Recall that we had also proved this earlier by manually calculating the convolution).
    
    \item For $r,s\in(0,\infty)$ and $p\in(0,1]$, we have $b^-_{r,p}*b^-_{s,p}=b^-_{r+s,p}$. We leave it to the reader to prove this.
\end{itemize}

\subsection{The Poisson Approximation}

\begin{theorem}
    Let $\mu,\mu_1,\mu_2,\ldots$ be probability measures on $(\mathbb{N}_0,2^{\mathbb{N}_0})$ with generating functions $\psi,\psi_1,\psi_2,\ldots$. Then the following are equivalent.
    \begin{enumerate}[(a)]
        \item $\lim_{n\to\infty}\mu_n(\{k\})=\mu(\{k\})$ for all $k\in\mathbb{N}_0$.
        \item $\lim_{n\to\infty}\mu_n(A)=\mu(A)$ for all $A\subseteq\mathbb{N}_0$.
        \item $\lim_{n\to\infty}\psi_n(z)=\psi(z)$ for all $z\in[0,1]$.
        \item $\lim_{n\to\infty}\psi_n(z)=\psi(z)$ for all $z\in[0,\eta)$ for some $\eta\in(0,1)$.
    \end{enumerate}
    If any of the above is true, we write $\lim_{n\to\infty}\mu_n=\mu$ and say that $(\mu_n)_{n\in\mathbb{N}}$ converges weakly to $\mu$.
\end{theorem}
\begin{proof}
~
    \begin{itemize}
        \item (a)$\implies$(b).
        
        Fix $\varepsilon>0$. Choose some $N\in\mathbb{N}$ such that
        $$\mu(\{N+1,N+2,\ldots\})<\frac{\varepsilon}{4}$$
        and sufficiently large $n_0\in\mathbb{N}$ such that
        $$\sum_{k=0}^N |\mu_n(\{k\})-\mu(\{k\})| < \frac{\varepsilon}{4}\text{ for all $n\geq n_0$.}$$
        Note that for any $n\geq n_0$, $\mu_n(\{N+1,N+2,\ldots\})<\frac\varepsilon2$. Therefore, for any $A\subseteq\mathbb{N}$ and $n\geq n_0$,
        \begin{align*}
            |\mu_n(A)-\mu(A)| &\leq \mu_n(\{N+1,N+2,\ldots\}) + \mu(\{N+1,N+2,\ldots\}) + \sum_{k\in A\cap\{0,1,\ldots,N\}} |\mu_n(\{k\})-\mu(\{k\})| \\
            &\leq \varepsilon.
        \end{align*}
        This proves the result.
        
        \item (b)$\implies$(a).
        
        This is trivial.
        
        \item (a)$\iff$(c)$\iff$(d)
        
        This follows from \cref{pgfpowerseries}.
    \end{itemize}
\end{proof}

When we are dealing with the binomial distribution for large $n$, it is usually inconvenient to calculate the terms. However, in the case where $n$ is very large and $p$ is very small such that $np=\lambda$ is of reasonable magnitude, then we can approximate $b_{n,p}$ by $\Poi_\lambda$. This is made rigorous as follows.

\vspace{2mm}
Let $(p_{n,k})_{n,k\in\mathbb{N}}$ such that $p_{n,k}\in[0,1]$. Let
$$\lambda=\lim_{n\to\infty}\sum_{k=1}^\infty p_{n,k}\in(0,\infty) \text{ and }\lim_{n\to\infty}\sum_{k=1}^\infty p_{n,k}^2 = 0.$$
An example of such a family would be $p_{n,k}=\lambda/n$ for $k\leq n$ and $p_{n,k}=0$ otherwise.

For each $n\in\mathbb{N}$, let $(X_{n,k})_{k\in\mathbb{N}}$ be a family of independent random variables such that $X_{n,k}\sim\Ber_{p_{n,k}}$. For $n,k\in\mathbb{N}$, define
$$S^n = \sum_{l=1}^\infty X_{n,l}\text{ and }S_k^n = \sum_{l=1}^k X_{n,l}.$$

\begin{theorem}[Poisson Approximation]
    With the above notation, the distributions $(\textbf{P}_{S^n})_{n\in\mathbb{N}}$ converge weakly to $\Poi_{\lambda}$.
\end{theorem}
\begin{proof}
    % We first claim that for any $z\in[0,1]$, $\psi_{S^n}(z)=\lim_{k\to\infty}\psi_k^n(z)$. Note that $S^n_k$ and $S^n-S^n_k$ are independent and as a result, $\psi_{S^n}=\psi_{S^n_k}\cdot\psi_{S^n-S^n_k}$. This implies that
    % \begin{align*}
    %     1 &\geq \frac{\psi_{S^n}}{\psi_{S^n_k}} \\
    %       &= \psi_{S^n-S^n_k} \\
    %       &\geq \Pr[S^n-S^n_k = 0] \\
    %       &= 1 - \Pr[S^n-S^n_k \geq 1] \\
    %       &\geq 1 - \sum_{l=k+1}^\infty p_{n,l}
    % \end{align*}
    For any $z\in[0,1]$,
    \begin{align*}
        \psi_{S^n}(z) &= \prod_{l=1}^\infty (p_{n,l}z + 1-p_{n,l}) \\
        &= \exp\left(\sum_{l=1}^\infty \log(p_{n,l}z + 1-p_{n,l})\right).
    \end{align*}
    Now, as $|\log(1-x)-x|\leq x^2$ for $|x|<\frac{1}{2}$, we have
    $$
        \left|\left(\sum_{l=1}^\infty \log(p_{n,l}z + 1-p_{n,l})\right) - \left((z-1)\sum_{l=1}^\infty p_{n,l}\right)\right| \leq \sum_{l=1}^\infty p_{n,l}^2.
    $$
    Taking the limit as $n\to\infty$, we have
    $$\lim_{n\to\infty} \psi_{S^n}(z) = \exp\left((z-1)\sum_{l=1}^\infty p_{n,l}\right) = e^{\lambda(z-1)}.$$
    As $\psi_{\Poi_{\lambda}} = e^{\lambda(z-1)}$, this completes the proof.
\end{proof}

For example, let $p_{n,k}$ be $\lambda/n$ if $k\leq n$ and $0$ otherwise. Then by the Poisson approximation, $\lim_{n\to\infty} b_{n,\frac{\lambda}{n}}=\Poi_{\lambda}$.

\subsection{Branching Processes}

Let $T,X_1,X_2,\ldots$ be $\mathbb{N}_0$-valued random variables and $S=\sum_{i=1}^T X_i$. Note that $S$ is measurable (and thus a random variable) as
$$\{S=k\}=\bigcup_{i=0}^\infty \{T=i\}\cap\{X_1+\cdots+X_n=k\}.$$

\begin{theorem}
\label{pgfcomposition}
    With the above notation, if $X_1,X_2,\ldots$ are all identically distributed, then $\psi_S=\psi_T\circ\psi_{X_1}$.
\end{theorem}
\begin{proof}
    We have
    \begin{align*}
        \psi_S(z) &= \sum_{k=0}^\infty\Pr[S=k]z^k \\
        &= \sum_{k=0}^\infty\sum_{i=0}^\infty \Pr[T=i]\Pr[X_1+\cdots+X_n=k]z^k \\
        &= \sum_{i=0}^\infty \Pr[T=i](\psi_{X_1}(z))^n \\
        &= (\psi_T\circ\psi_{X_1})(z).
    \end{align*}
\end{proof}

Let $p_0,p_1,\ldots\in[0,1]$ such that $\sum_{i=0}^\infty p_i = 1$. Let $(X_{n,i})_{n,i\in\mathbb{N}_0}$ be an independent family of random variables such that $\Pr[X_{n,i}=k]=p_k$ for any $k,n,i\in\mathbb{N}$.

\vspace{1mm}
Let $Z_0=1$ and $Z_n=\sum_{i=1}^{Z_{n-1}} X_{n,i}$ for each $n\in\mathbb{N}$. This can be interpreted as the number of members in each generation of a family where the number of offspring each person has is random and given by $X_{n,i}$.

\begin{definition}
    $(Z_n)_{n\in\mathbb{N}_0}$ is called a \textit{Galton-Watson process} or \textit{branching process} with offspring distribution $(p_k)_{k\in\mathbb{N}_0}$
\end{definition}

Branching processes are easier to study with the assistance of generator functions. For $z\in[0,1]$, let
$$\psi(z)=\sum_{k=0}^\infty p_k z^k.$$
Recursively define
$$\psi_1=\psi\text{ and }\psi_{n+1}=\psi_n\text{ for all }n\in\mathbb{N}.$$
\begin{theorem}
    $\psi_{Z_n}=\psi_n$ for all $n\in\mathbb{N}$.
\end{theorem}
\begin{proof}
    We have $\psi_{Z_1}=\psi_1$ (by definition). By \cref{pgfcomposition}, $\psi_{Z_{n+1}} = \psi\circ\psi_{Z_{n}}$. The result follows inductively.
\end{proof}

Now, let us study the probability that the family dies out.

Denote by $q_n=\Pr[Z_n=0]$ the probability that $Z$ is extinct by time $n$. Clearly, $q_n$ is monotone increasing in $n$. In the limiting case, we have the following.

\begin{definition}
    Let $(Z_n)_{n\in\mathbb{N}_0}$ be a branching process. We define its extinction probability by
    $$q=\lim_{n\to\infty}\Pr[Z_n=0].$$
\end{definition}

A natural question to ask is: under what condition does the family definitely die out, that is, $q=1$? We clearly have $q\geq p_0$ since $q_n$ is monotone increasing and $q=\lim_{n\to\infty}q_n$. If $p_0=0$, then $Z_n$ is monotone increasing in $n$ and as a result, $q=0$ as well.

\begin{theorem}[Extinction Probability of a Branching Process]
    Let $(Z_n)_{n\in\mathbb{N}_0}$ be a branching process with offspring distribution $(p_k)_{k\in\mathbb{N}_0}$ such that $p_1\neq 1$. Then
    \begin{enumerate}[(a)]
        \item $\{r\in[0,1]:\psi(r)=r\}=\{q,1\}$.
        \item The following holds:
        $$q<1\iff \lim_{z\to 1}\psi'(z)>1\iff \sum_{k=1}^\infty kp_k>1.$$
    \end{enumerate}
\end{theorem}
\begin{proof}
    ~
    \begin{enumerate}[(a)]
        \item Let $F=\{r\in[0,1]:\psi(r)=r\}$.
        %F for "Fixed"
        Clearly, $1\in F$ as $\psi(1)=1$. Note that $q_n=\psi_n(0)=\psi(q_{n-1})$ for all $n\in\mathbb{N}$. As $\psi$ is continuous,
        $$\psi(q)=\psi\left(\lim_{n\to\infty} q_n\right)=\lim_{n\to\infty}\psi(q_n)=\lim_{n\to\infty} q_{n+1}=q.$$
        Thus $q\in F$.
        
        We next claim that $q=\min F$. Let $r\in F$. Then $r\geq 0= q_0$. If $r\geq q_n$ for some $n\in\mathbb{N}$, then as $\psi$ is monotone increasing, $r=\psi(r)\geq\psi(q_n)=q_{n+1}$.
        
        Inductively, $r\geq q_n$ for every $n\in\mathbb{N}$. The claim follows. We complete the remainder of the proof in the second part.
        
        \item The second equivalence follows by \cref{pgfpowerseries}. For the first equivalence, consider the following two cases.
        \begin{itemize}
            \item $\lim_{z\to 1}\psi'(z)\leq 1$. Since $\psi$ is strictly convex, it follows that $\psi(z)>z$ for all $z\in[0,1)$ and so $F=\{1\}$. By the first part of the proof, $q=1$.
            
            \item $\lim_{z\to 1}\psi'(z)> 1$. Since $\psi$ is strictly convex and $\psi(0)\geq 0$, there is some unique $r\in(0,1)$ such that $\psi(r)=r$. Then $F=\{r,1\}$ and by the first part, $q=\min F=r<1$.
        \end{itemize}
    \end{enumerate}
    This completes the proof.
\end{proof}

\clearpage
\section{The Integral}

In the following, we assume $(\Omega,\mathcal{A},\mu)$ to be a measure space. We denote by $\mathcal{E}$ the vector space of simple functions on $(\Omega,\mathcal{A})$ and by $\mathbb{E}^+=\{f\in E:f\geq 0\}$ the cone of nonnegative simple functions. If
$$f=\sum_{i=1}^n \alpha_i\indic_{A_i}$$
for some $n\in\mathbb{N}$, mutually disjoint sets $A_1,\ldots,A_n\in\mathcal{A}$ and $\alpha_1,\ldots,\alpha_n\in(0,\infty)$ then this representation is said to be a \textit{normal representation of $f$}.

\subsection{Set Up to Define the Integral}

\begin{lemma}
    Let $f=\sum_{i=1}^m\alpha_i\indic_{A_i}$ and $f=\sum_{j=1}^n\beta_j\indic_{B_j}$ be two normal representations of $f\in\mathbb{E}^+$. Then
    $$\sum_{i=1}^m\alpha_i\mu(A_i)=\sum_{j=1}^n\beta_j\mu(B_j)$$
\end{lemma}
\begin{proof}
    Clearly, if $\alpha_i\neq 0$ for some $i$, then $A_i\subseteq\bigcup_{j=1}^n B_j$. A similar result holds for $B_j$. Thus,
    $$\sum_{i=1}^m\alpha_i\mu(A_i) = \sum_{i=1}^m\sum_{j=1}^n\alpha_i\mu(A_i\cap B_j).$$
    If $\mu(A_i\cap B_j)\neq 0$, then $f(\omega)=\alpha_i=\beta_j$ for any $\omega\in A_i\cap B_j$. Therefore,
    $$\sum_{i=1}^m\sum_{j=1}^n\alpha_i\mu(A_i\cap B_j) = \sum_{j=1}^n\sum_{i=1}^m\beta_j\mu(A_i\cap B_j) = \sum_{j=1}^n\beta_j\mu(B_j).$$
\end{proof}

\begin{definition}
    Define $I:\mathbb{E}^+\to[0,\infty]$ by
    $$I(f) = \sum_{i=1}^m\alpha_i\mu(A_i)$$
    if $f$ has the normal representation $f=\sum_{i=1}^m\alpha_i\indic_{A_i}$.
\end{definition}
The above definition makes sense due to the previous lemma.

\begin{lemma}
\label{properties Of I for simple}
    Let $f,g\in\mathbb{E}^+$ and $\alpha\geq 0$. Then
    \begin{enumerate}[(a)]
        \item $I(\alpha f)=\alpha I(f)$,
        \item $I(f+g)=I(f)+I(g)$, and
        \item If $f\leq g$, then $I(f)\leq I(g)$.
    \end{enumerate}
\end{lemma}
We leave the proof of this theorem as an exercise to the reader.

\begin{definition}[Integral]
    If $f:\Omega\to[0,\infty]$ is measurable, then we define the \textit{integral} of $f$ with respect to $\mu$ by
    $$\int f\d{\mu} = \sup\left\{I(g):g\in\mathbb{E}^+, g\leq f\right\}.$$
\end{definition}

Note that by \cref{properties Of I for simple}(iii), $I(f)=\int f\d{\mu}$ for any $f\in\mathbb{E}^+$. That is, the integral is an extension of $I$ from $\mathbb{E}^+$ to the set of non-negative measurable functions. We expand this to measurable functions in general in the next subsection.

\vspace{2mm}
Let $f,g:\Omega\to\overline{\mathbb{R}}$. Similar to how we write $f\leq g$ if $f(\omega)\leq g(\omega)$ for all $\omega\in\Omega$, we write $f\leq g$ almost everywhere if there exists some set $N\in\mathcal{A}$ such that $\mu(N)=0$ and $f(\omega)\leq g(\omega)$ for all $\omega\in\Omega\setminus N$.

\begin{theorem}
\label{properties of integral 2}
    Let $f,g,f_1,f_2,\ldots$ be measurable maps $\Omega\to[0,\infty]$. Then
    \begin{enumerate}[(a)]
        \item If $f\leq g$, then $\int f\d{\mu}\leq \int g\d{\mu}$.
        \item If $f_n \uparrow f$, then $\int f_n\d{\mu} \uparrow \int f\d{\mu}$.
        \item If $\alpha,\beta\in[0,\infty]$, then
        $$\int(\alpha f+\beta g)\d{\mu}=\alpha\int f\d{\mu} + \beta\int g\d{\mu}$$
        where we take $\infty\cdot 0=0$.
    \end{enumerate}
\end{theorem}
\begin{proof}
    ~
    \begin{enumerate}[(a)]
        \item This is obvious from the definition of the integral.
        
        \item By the definition of the integral,
        
        $$\lim_{n\to\infty}\int f_n\d{\mu}=\sup_{n\in\mathbb{N}}\int f_n\d{\mu}\leq \int f\d{\mu}$$
        
        We must now show that $\int f\d{\mu}\leq \sup_{n\in\mathbb{N}}\int f_n\d{\mu}$. Let $g\in\mathbb{E}^+$ with $g\leq f$. It is enough to show that $\sup_{n\in\mathbb{N}}\int f_n\d{\mu}\geq \int g\d{\mu}$.
        Fix some $t\in(0,1)$. For each $n\in\mathbb{N}$, define
        
        $$A_n = \{\omega\in\Omega:f_n(x)\geq tg(x)\}.$$
        
        Note that each $A_n$ is measurable (Why?) and that $A_i\subset A_{i+1}$ for each $i\in\mathbb{N}$.
        
        We first claim that $\bigcup_{i=1}^\infty A_i = \Omega$. We prove this as follows. For any $\omega\in\Omega$,
        \begin{itemize}
            \item If $f(\omega)\leq tg(\omega)$, then $f(\omega)=0$ and $\omega\in A_n$ for every $n\in\mathbb{N}$.
            
            \item If $f(\omega)> tg(\omega)$, then there exists some $n\in\mathbb{N}$ such that $f_n(\omega)>tg(\omega)$. It follows that $\omega\in A_n$.
        \end{itemize}
        Therefore, $\Omega\subseteq\bigcup_{i=1}^\infty A_i$. Since the reverse inclusion is obviously true, we have $\bigcup_{i=1}^n A_i = \Omega$.
        
        Now,
        $$\int f_n\d{\mu} \geq \int tg\indic_{A_n}\d{\mu}$$
        Taking the limit as $n\to\infty$ and $t\to 1$, we have
        $$\lim_{n\to\infty}\int f_n\d{\mu}\geq \int g\d{\mu}.$$
        This completes the proof.
        
        \item By \cref{measurable function simple sequence}, there exist sequences $(f_n)_{n\in\mathbb{N}}$ and $(g_n)_{n\in\mathbb{N}}$ in $\mathbb{E}^+$ such that $f_n\uparrow f$ and $g_n\uparrow g$. Then by \cref{properties Of I for simple} and (ii),
        \begin{align*}
            \int (\alpha f+\beta g)\d{\mu} &= \lim_{n\to\infty} \int(\alpha f+\beta g)\d{\mu} \\
            &= \alpha \lim_{n\to\infty} \int f\d{\mu} + \beta \lim_{n\to\infty} \int g\d{\mu} = \alpha\int f\d{\mu} + \beta\int g\d{\mu}.
        \end{align*}
    \end{enumerate}
\end{proof}


It is important to note that (b) in the above only works if the sequence of functions is non-decreasing. We do \textit{not} have a similar result for a non-increasing sequence. To see this, consider $f_1,f_2,\ldots:\mathcal{B}(\mathbb{R})\to\overline{\mathbb{R}}$ given by $f_n=\indic_{[n,\infty)}$ for each $n\in\mathbb{N}$.

% \begin{theorem}
% \label{ae integral equality}
%     Let $f,g$ be measurable maps $\Omega\to[0,\infty]$. If $f\leq g$ almost everywhere, Then $\int f\d\mu\leq \int g\d\mu$. In particular, if $f=g$ almost everywhere, then $\int f\d\mu=\int g\d\mu$.
% \end{theorem}
% \begin{proof}
    
% \end{proof}

\subsection{The Integral and some Properties}

We have now introduced enough to define the integral for measurable functions in general.

\begin{fdef}[Integrals of Measurable Functions]
\label{defIntegral}
    Let $f:\Omega\to\overline{\mathbb{R}}$ be measurable. We call $f$ \textit{$\mu$-integrable} if $\int |f|\d{\mu}<\infty$ and write
    $$\mathcal{L}^1(\mu)=\mathcal{L}^1(\Omega,\mathcal{A},\mu)=\left\{f:\Omega\to\overline{\mathbb{R}}:f\text{ is }\mu\text{-integrable}\right\}.$$
    For $f\in\mathcal{L}^1(\mu)$, we define the integral of $f$ with respect to $\mu$ by
    $$\int f(\omega)\mu(\d{\omega})=\int f\d{\mu}=\int f^+\d{\mu} - \int f^-\d{\mu}.$$
    For $A\in\mathcal{A}$, we define
    $$\int_A f\d{\mu} = \int (f\indic_A)\d{\mu}.$$
\end{fdef}

Elements of $\mathcal{L}^1(\textbf{P})$ are also sometimes called \textit{absolutely integrable} functions. 

\begin{theorem}
\label{properties of integral ae 1}
    Let $f:\Omega\to[0,\infty]$ be measurable.
    \begin{enumerate}[(a)]
        \item $f=0$ almost everywhere if and only if $\int f\d{\mu} = 0$.
        \item If $\int f\d{\mu}<\infty$, then $f<\infty$ almost everywhere.
    \end{enumerate}
\end{theorem}
\begin{proof}
    ~
    \begin{enumerate}[(a)]
        \item Let us first prove the forward implication. Let $P=\{\omega\in\Omega:f(\omega)>0\}$. Then $f\leq \infty\cdot\indic_{P}$. As $n\indic_{N}\uparrow\infty\indic_{N}$, by \cref{properties of integral 2},
        $$0\leq\int f\d{\mu}\leq\lim_{n\to\infty}\int n\indic_{N}\d{\mu}=0.$$
        
        For the backward implication, let $A_n=\{\omega\in\Omega:f(\omega)\geq 1/n\}$. Then $A_n\uparrow P$ and for any $n\in\mathbb{N}$,
        $$0=\int f\d{\mu} \geq \int \frac{1}{n}\indic_{A_n} = \frac{\mu(A_n)}{n}.$$
        This implies that $\mu(A_n)=0$ for any $n\in\mathbb{N}$ and therefore, $\mu(P)=0$.
        
        \item Let $A=\{\omega\in\Omega:f(\omega)=\infty\}$. Then for any $n\in\mathbb{N}$,
        $$\mu(A)=\int \indic_{A}\d{\mu}\leq \frac{1}{n}\int f\indic_{f\geq n}\leq \frac{1}{n} \int f\d{\mu}.$$
        Taking the limit as $n\to\infty$, we get $\mu(A)=0$.
    \end{enumerate}
\end{proof}

We now expand some of the properties that we proved earlier for non-negative measurable functions to measurable functions in general.

\begin{theorem}
\label{properties of integral final}
    Let $f,g\in\mathcal{L}^1(\mu)$.
    \begin{enumerate}[(a)]
        \item (Monotonicity). If $f\leq g$ almost everywhere, then $\int f\d\mu\leq \int g\d\mu$. In particular, if $f=g$ almost everywhere, then $\int f\d{\mu}= \int g\d{\mu}$.
        \item (Triangle Inequality). $|\int f\d\mu|\leq \int |f|\d\mu$.
        \item (Linearity). If $\alpha,\beta\in\mathbb{R}$, then $\alpha f+\beta g\in\mathcal{L}^1(\mu)$ and
        $$\int \alpha f+ \beta g\d\mu = \alpha\int f\d\mu + \beta\int g\d\mu.$$
    \end{enumerate}
\end{theorem}
\begin{proof}
    ~
    \begin{enumerate}[(a)]
        \item Since $f\leq g$ almost everywhere, $f^+\leq g^+$ and $f^-\geq g^-$ almost everywhere. It is enough to show that $\int f^+\d\mu\leq \int g^+\d\mu$ and $\int f^-\d\mu\geq \int g^-\d\mu$. Let us prove the former. The latter can similarly be shown.
        
        Let $h=g^+-f^+$. As $h^-=0$ almost everywhere, by \cref{properties of integral ae 1}, $\int h\d\mu=\int h^+\d\mu\geq 0$. The result follows.
        
        \item We have
        \begin{align*}
            \left|\int f\d\mu\right| &= \left|\int f^+\d\mu - \int f^-\d\mu\right| \\
            &\leq \int (f^+ + f^-)\d\mu \\ &= \int |f|\d\mu.
        \end{align*}
        
        \item To show linearity, it suffices to show that for any $\alpha\in[0,\infty)$,
        \begin{itemize}
            \item $\int (f+g)\d\mu = \int f\d\mu + \int g\d\mu$,
            \item $\int \alpha f\d\mu = \alpha\int f\d\mu$, and
            \item $\int (-f)\d\mu = -\int f\d\mu$.
        \end{itemize}
        These are easily shown by splitting each function $f$ into $f^+$ and $f^-$ and using \cref{properties of integral 2} wherever necessary.
    \end{enumerate}
\end{proof}

\begin{definition}
    Let $f:\Omega\to[0,\infty)$ be measurable. Define the measure $\nu$ by
    $$\nu(A)=\int (\indic_A f)\d\mu\text{ for }A\in\mathcal{A}.$$
    Then $\nu$ is said to have \textit{density $f$} with respect to $\mu$. We also denote $\nu$ by $f\mu$.
\end{definition}

In showing that $\nu$ is a measure using \cref{tripledoubleEquivalence}, finite additivity follows from finite additivity of the integral and lower semicontinuity follows from \cref{Monotone Convergence Theorem}.

\begin{theorem}
    Let $f:\Omega\to[0,\infty)$ and $g:\Omega\to\overline{\mathbb{R}}$ be measurable. Then $g\in\mathcal{L}^1(f\mu)$ if and only if $(gf)\in\mathcal{L}^1(\mu)$. In this case, $$\int g\d{(f\mu)} = \int (gf)\d\mu.$$
\end{theorem}
We omit the proof of the above. It may be shown by first assuming $g$ to be an indicator function, then extending this to simple functions, non-negative measurable functions and measurable functions.

\begin{definition}
    Let $f:\Omega\to\overline{\mathbb{R}}$ be measurable and $p\in[1,\infty)$. Define
    $$\norm{f}_p=\left(\int |f|^p\d\mu\right)^{1/p}\text{ and }$$
    $$\norm{f}_\infty=\inf \{k\geq 0:\mu(\{|f|\geq k\})=0\}.$$
    Further, for any $p\in[0,\infty]$, we define the vector space
    $$\mathcal{L}^p(\mu) = \{f:\Omega\to\overline{\mathbb{R}} : f\text{ is measurable and }\norm{f}_p<\infty\}.$$
\end{definition}

\begin{theorem}
    The map $\norm{\cdot}_1$ is a seminorm on $\mathcal{L}^1(\mu)$, that is, for any $f,g\in\mathcal{L}^1(\mu)$ and $\alpha\in\mathbb{R}$,
    \begin{align*}
        \norm{\alpha f}_1 &= |\alpha|\cdot\norm{f}_1 \\
        \norm{f+g}_1 &\leq \norm{f}_1 + \norm{g}_1 \\
        \norm{f}_1 &\geq 0\text{ with equality if and only if }f=0\text{ almost everywhere.}
    \end{align*}
\end{theorem}
\begin{proof}
    The first and third (in)equalities follow from \cref{properties of integral final}(c) and \cref{properties of integral ae 1}(a) respectively. The second inequality follows from the fact that $|f+g|\leq |f|+|g|$. We leave the details of the proof to the reader.
\end{proof}

In fact, $\norm{\cdot}_p$ is a seminorm on $\mathcal{L}^p(\mu)$ for any $p\in[1,\infty]$. The proofs of the first and third (in)equalities are similarly straightforward. The proof of the second however requires Minkowski's inequality. %MINKOWSKI INEQUALITY!!!!!!!!!!!!!!!!!!!!!!!!!!!!!!!!!!!!!!!!!!!!!!!!!!!!!!!!!!!!!!!!!!!!!!!!!!!!!!!!!!!!!!!!!!!!!!!!!!!!!!!!!!!!!!!!!!!!!!!!!!!!!!!!!!!!!!!!!!!!!!!!!!!!!!!!!!!!!!!!!!!!!!!!!!!!!!!!!!!!!!!!!!!!!!!!!!!!!!!!!!!!!!!!!!!!!!!!!!!!!!

\begin{theorem}
    Let $\mu(\Omega)<\infty$ and $1\leq p'\leq p\leq \infty$. Then $\mathcal{L}^{p}(\mu)\subseteq\mathcal{L}^{p'}(\mu)$ and further, the canonical inclusion $\mathcal{L}^{p}(\mu)\hookrightarrow\mathcal{L}^{p'}(\mu)$ given by $f\mapsto f$ is continuous.
\end{theorem}
\begin{proof}
    Let us first take the case where $p=\infty$. For any $f\in\mathcal{L}^\infty(\mu)$, since $|f|\leq \norm{f}_\infty$ almost everywhere,
    $$\int |f|^{p'}\d\mu \leq \int \norm{f}_\infty^{p'} \d\mu = \mu(\Omega)\norm{f}_\infty^{p'} < \infty.$$
    It follows that for any $f,g\in\mathcal{L}^\infty(\mu)$,
    $$\norm{f-g}_{p'} \leq (\mu(\Omega))^{1/p'}\norm{f-g}_\infty$$
    and so the inclusion map is continuous.
    
    Let us next take the case where $p$ is finite. Then for any $f\in\mathcal{L}^{p}(\mu)$ since $|f|^{p'}\leq 1+|f|^p$,
    $$\int |f|^{p'}\d\mu \leq \int 1+|f|^p \d\mu = \mu(\Omega) + \int |f|^p \d\mu < \infty.$$
    
    Now, for any $f,g\in\mathcal{L}^{p}(\mu)$, let $c=\norm{f-g}_{p}$. Then
    \begin{align*}
        |f-g|^{p'} &= |f-g|^{p'}\indic_{|f-g|\leq c} + |f-g|^{p'}\indic_{|f-g|> c} \\
        &\leq c^{p'} + c^{p'-p}|f-g|^p
    \end{align*}
    
    This implies that
    \begin{align*}
        \norm{f-g}_{p'} &\leq c\left(\int 1+c^{-p}|f-g|^p\d\mu\right)^{1/p'} \\
        &= \norm{f-g}_{p}(1+\mu(\Omega))^{1/p'}.
    \end{align*}
    This completes the proof.
\end{proof}

\subsection{Monotone Convergence and Fatou's Lemma}

Under what conditions can we exchange the limit and the integral? We answered this question in part in \cref{properties of integral 2}(b). Over the course of this subsection, we attempt to answer this.

\begin{ftheo}[Monotone Convergence, Beppo-Levi Theorem]
\label{Monotone Convergence Theorem}
Let $f_1,f_2,\ldots\in\mathcal{L}^1(\mu)$ and $f:\Omega\to\overline{\mathbb{R}}$ be measurable. Assume that $f_n\uparrow f$ almost everywhere. Then
$$\lim_{n\to\infty}\int f_n\d\mu = \int f\d\mu$$
where both sides can equal $\infty$.
\end{ftheo}
\begin{proof}
    Let $N$ be a set such that $\mu(N)=0$ and $f_n(\omega)\uparrow f(\omega)$ for all $\omega\in N^c$. For each $n\in\mathbb{N}$, define
    $$f_n'=(f_n-f_1)\indic_{N^c}\text{ and }f'=(f-f_1)\indic_{N^c}.$$
    Note that $f_n'\uparrow f'$ and each of these functions are non-negative. By \cref{properties of integral 2}(b), $\int f_n'\d\mu\uparrow\int f'\d\mu$. Then by \cref{properties of integral final}(a),
    \begin{align*}
        \lim_{n\to\infty} \int f_n\d\mu &= \lim_{n\to\infty}\left(\int f_n'\d\mu + \int f_1\d\mu\right) \\
        &= \int f'\d\mu + \int f_1\d\mu \\
        &= \int f\d\mu.
    \end{align*}
\end{proof}

\begin{ftheo}[Fatou's Lemma]
\label{fatous lemma}
    Let $f\in\mathcal{L}^1(\mu)$ and let $f_1,f_2,\ldots$ be measurable with $f_n\geq f$ almost everywhere for all $n\in\mathbb{N}$. Then
    $$\int \liminf_{n\to\infty} f_n\d\mu \leq \liminf_{n\to \infty} \int f_n\d\mu$$
\end{ftheo}
\begin{proof}
    Considering $f_n-f$ for each $n\in\mathbb{N}$, we may assume each $f_n$ to be non-negative almost everywhere. For each $m\in\mathbb{N}$, consider $g_m=\inf_{n\geq m}f_n$. Note that $g_m\uparrow \liminf_{n\to\infty} f_n$. Thus, using \cref{properties of integral final}(a) and \cref{Monotone Convergence Theorem},
    $$\int \liminf_{n\to\infty} f_n = \int \lim_{m\to\infty} g_m\d\mu = \lim_{m\to\infty}\int g_m\d\mu \leq \liminf_{n\to\infty}\int f_n\d\mu.$$
\end{proof}

In the above theorem, we require an integrable $f$ for the statement to hold. This $f$ is called a ``minorant".

\vspace{2mm}
An example to show this is the following, known as Peterburg's game.

Consider a gamble in a casino where you double your bet with probability $p\leq\frac{1}{2}$ and lose it with probability $1-p$. We gamble over and over. This can be modelled by the probability space $(\Omega,\mathcal{A},\textbf{P})$ where $\Omega=\{-1,1\}^\mathbb{N}$, $\mathcal{A}=(2^{\{-1,1\}})^{\otimes\mathbb{N}}$, and $\textbf{P}=((1-p)\delta_{-1} + p\delta_1)^\mathbb{N}$. Let us denote by $D_n:\Omega\to\{-1,1\}$ the output of the $n$th game

\vspace{1mm}
If the player bets $b_i$ dollars on the $i$th game, then his total profit after the $n$th game is $S_n\sum_{i=1}^n b_iD_i$. Now, let the gambler assume the following strategy. He bets $1$ dollar on the first game. If he wins, he stops playing ($H_n=0$ for all $n\geq 2$). If he loses, he doubles the bet in the subsequent round.That is,
$$
H_n=
\begin{cases}
    0, &\text{if $D_i=1$ for some $i<n$} \\
    1, &\text{otherwise.}
\end{cases}
$$
The probability of no win until the $n$th game is $(1-p)^n$.

Therefore, $\textbf{P}[S_n=1-2^n]=(1-p)^n$ and $\textbf{P}[S_n=1]=1-(1-p)^n$. The average expected gain is
$$\int S_n\d{\textbf{P}} = (1-2^n)(1-p)^n + (1-(1-p)^n)=1-(2(1-p))^n\leq 0.$$
Define
$$
S=
\begin{cases}
    -\infty, &\text{if $-1=D_1=D_2=\cdots$} \\
    1, &\text{otherwise.}
\end{cases}
$$

Then $\lim_{n\to\infty} S_n = S$ almost surely. However, $\lim_{n\to\infty}\int S_n\d{\textbf{P}} < \lim_{n\to\infty}\int S\d{\textbf{P}}$ since $S=1$ almost surely. 

By Fatou's Lemma, this is only possible if there is no integrable minorant $f$ for $(S_n)_{n\in\mathbb{N}}$.

Indeed, letting $\tilde S = \inf \{S_n:{n\in\mathbb{N}}\}$,
$$\textbf{P}[\tilde S = 1-2^{n-1}]=\textbf{P}[D_1=D_2=\cdots=D_{n-1}=-1\text{ and }D_n=1]=p(1-p)^{n-1}.$$
Therefore,
$$\int \tilde S\d{\textbf{P}}=\sum_{n=1}^\infty (1-2^{n-1})p(1-p)^{n-1}=-\infty.$$

\vspace{2mm}

Also, in Fatou's Lemma, we cannot replace the $\liminf$ with a $\limsup$, which can be seen from the following example.

Let $X$ be drawn uniformly randomly from $[0,1]$, and let $X_n$ be the $n$th binary digit of $X$ (we needn't worry about the case where there is ambiguity in the binary expression as this almost surely does not occur). We have $\expec
[X_n]=1/2$ for all $n$. Then $\limsup_{n\to\infty}X_n$ is almost surely $1$, so we cannot replace $\liminf$ with $\limsup$ in Fatou's lemma.

\subsection{Miscellaneous}

It may be shown that if $f:I\to\mathbb{R}$ is Riemann integrable on $I=[a,b]$, then $f$ is Lebesgue integrable on $I$ with integral
$$\int_I f\d\lambda = \int_a^b f(x)\d x$$

The converse is not true, that is, a function that is Lebesgue integrable need not be Riemann integrable. An example of this is $\indic_\mathbb{Q}$.

\begin{theorem}
\label{geq leq int leq g}
    Let $f:\Omega\to\mathbb{R}$ be measurable and $f\geq 0$ almost everywhere. Then
    $$\sum_{n=1}^\infty \mu(\{f\geq n\})\leq \int f\d\mu \leq \sum_{n=0}^\infty \mu(\{f>n\}).$$
\end{theorem}
\begin{proof}
    Define $f_1=\lfloor f\rfloor$ and $f_2=\lceil f\rceil$. Clearly, $f_1\leq f\leq f_2$ and so, $\int f_1\d\mu\leq \int f\d\mu\leq \int f_2\d\mu$. We then have
    \begin{align*}
        \int f_1\d\mu &= \sum_{k=1}^\infty k\mu(\{f_1=k\}) \\
        &= \sum_{k=1}^\infty \sum_{n=1}^k \mu(\{f_1=k\}) \\
        &= \sum_{n=1}^\infty \sum_{k=n}^\infty \mu(\{f_1=k\}) \\
        &= \sum_{n=1}^\infty \mu(\{f_1\geq n\}) \\
        &= \sum_{n=1}^\infty \mu(\{f\geq n\}).
    \end{align*}
    Similarly, for the second inequality,
    \begin{align*}
        \int f_2\d\mu &= \sum_{n=1}^\infty \mu(\{f_2\geq n\}) \\
        &= \sum_{n=1}^\infty \mu(\{f> n-1\}).
    \end{align*}
    This proves the result.
\end{proof}

\begin{theorem}
\label{int func eq int geq}
    Let $f:\Omega\to\mathbb{R}$ be measurable and $f\geq 0$ almost everywhere. Then
    $$\int f\d\mu = \int_0^\infty \mu(\{f\geq t\})\d t.$$
\end{theorem}
\begin{proof}
    Define $g$ by $g(t)=\mu(\{f\geq t\})$. We may assume that $g(t)<\infty$ for all $t>0$.
    
    For $\varepsilon>0$ and $k\in\mathbb{N}$, let $g_\varepsilon = \min(g, g(\varepsilon))$, $f_\varepsilon=f\indic_{\{f\geq\varepsilon\}}$, and $f_{\varepsilon,k}=2^kf_\varepsilon$. and
    $$\alpha_{\varepsilon,k}=2^{-k}\sum_{n=1}^\infty \mu(\{f_{\varepsilon}\geq 2^{-k} n\}).$$
    Note that $\alpha_{\varepsilon,k} \xrightarrow{k\to\infty} \int_0^\infty g_\varepsilon\d\mu$ (Why?). Using \cref{geq leq int leq g} on $f_{\varepsilon,k}$,
    \begin{align*}
        \alpha_{\varepsilon,k} &= 2^{-k}\sum_{n=1}^\infty \mu(\{f_{\varepsilon,k}\geq n\}) \leq \int f_\varepsilon\d\mu \\
        &\leq 2^{-k}\sum_{n=0}^\infty \mu(\{f_{\varepsilon,k}> n\}) \\
        &= 2^{-k}\sum_{n=0}^\infty \mu(\{f_{\varepsilon}> n2^{-k}\}) \leq \alpha_{\varepsilon,k} + 2^{-k}g(\varepsilon).
    \end{align*}
    Taking the limit as $k\to\infty$, equality must hold everywhere so we get
    $$\int_0^\infty g_\varepsilon \d\mu = \int f_\varepsilon\d\mu.$$
    Taking the limit as $\varepsilon\downarrow 0$ completes the proof.
\end{proof}

\clearpage
\section{Moments}

When describing random variables, quantities like the expectation, median and variance are often used. They describe the behaviour of the variable on average, and how much they deviate from the average.

\subsection{Parameters of Random Variables}

In the following, let $(\Omega,\mathcal{A},\textbf{P})$ be a probability space.

\begin{definition}
    Let $X$ be a real random variable.
    \begin{enumerate}[(i)]
        \item If $X\in\mathcal{L}^1(\textbf{P})$, then $X$ is called integrable and we call
        $$\expec[X]=\int X\d{\textbf{P}}$$
        the \textit{expectation} or \textit{mean} of $X$. If $\expec[X]=0$, we call $X$ \textit{centered}. The expectation of $X$ is given by the same expression even if only $X^+$ or $X^-$ is integrable.
        
        \item Let $n\in\mathbb{N}$ and $X\in\mathcal{L}^1(\textbf{P})$. Then for any $k\in [n]$,
        $$m_k=\expec[X^k]\text{ and }M_k=\expec[|X|^k]$$
        are called the \textit{$k$th moments} and \textit{$k$th absolute moments} of $X$ respectively.
        
        \item If $X\in\mathcal{L}^2(\textbf{P})$, $X$ is called \textit{square integrable} and
        $$\Var[X]=\expec[X^2]-\expec[X]^2$$
        is called the \textit{variance} of $X$. The number $\sigma=\sqrt{\Var[X]}$ is called the \textit{standard deviation} of $X$ (This makes sense as we prove in \cref{properties of variance}(a) that $\Var[X]\geq 0$). We sometimes write $\Var[X]=\infty$ if $\expec[X^2]=\infty$.
        
        \item If $X,Y\in\mathcal{L}^2(\textbf{P})$, we define the \textit{covariance of $X$ and $Y$} by
        $$\Cov[X,Y]=\expec[(X-\expec[X])(Y-\expec[Y])].$$
        $X$ and $Y$ are called \textit{uncorrelated} if $\Cov[X,Y]=0$ and \textit{correlated} otherwise.
    \end{enumerate}
\end{definition}

We now give some basic properties of expectation. All of them follow from the corresponding properties of the integral.

\vspace{2mm}
The mean represents the average value of the random variable, the variance gives a measure of the deviation or dispersion of the random variable from the mean, and the covariance gives a measure of how related two random variables are.

\begin{theorem}[Properties of Expectation]
\label{expectation rules}
    Let $X,Y,X_n,Z_n$ ($n\in\mathbb{N}$) be real integrable random variables on $(\Omega,\mathcal{A},\textbf{P})$. Then
    \begin{enumerate}[(a)]
        \item If $\textbf{P}_X=\textbf{P}_Y$, then $\expec[X]=\expec[Y]$.
        \item For any $c\in\mathbb{R}$, $cX\in\mathcal{L}^1(\textbf{P})$ and $X+Y\in\mathcal{L}^1(\textbf{P})$. Further,
        $$\expec[cX]=c\expec[X]\text{ and }\expec[X+Y]=\expec[X]+\expec[Y].$$
        \item If $X\geq 0$ almost surely, then $\expec[X]=0$ if and only if $X=0$ almost surely.
        \item If $X\leq Y$ almost surely, then $\expec[X]\leq\expec[Y]$. Further, $\expec[X]=\expec[Y]$ if and only if $X=Y$ almost surely.
        \item $|\expec[X]|\leq\expec[|X|]$.
        \item If $X_n\geq 0$ almost surely for each $n\in\mathbb{N}$, then $\expec[\sum_{i=0}^\infty X_i] = \sum_{i=0}^\infty \expec[X_i]$.
        \item If $Z_n\uparrow Z$ for some $Z$, then $\expec[Z]=\lim_{n\to\infty}Z_n$.
    \end{enumerate}
\end{theorem}

Note that as a consequence of part (b) of the above,
$$\Cov[X,Y]=\expec[XY]-\expec[X]\expec[Y].$$

Setting $X=Y$, we have $\Cov[X,X]=\Var[X]$.

\vspace{2mm}
In the above, we did not use anything other than the properties of the integral itself. When we involve probability, namely independence, we get the following result.


\begin{theorem}[Independent Variables are Uncorrelated]
\label{independent uncorrelated}
    Let $X,Y\in\mathcal{L}^1(\textbf{P})$ be independent. Then $XY\in\mathcal{L}^1(\textbf{P})$ and further, $\expec[XY]=\expec[X]\expec[Y]$. Note that this implies $\Cov[X,Y]=0$.
\end{theorem}
\begin{proof}
    Let us first take the case where $X$ and $Y$ each take finitely many values. Then $Z=XY$ takes finitely many values as well and so, $Z\in\mathcal{L}^1(\textbf{P})$. We now have
    \begin{align*}
        \expec[Z] &= \sum_{z\in\mathbb{R}\setminus\{0\}} z\Pr[Z=z] \\
        &= \sum_{x\in\mathbb{R}\setminus\{0\}}\sum_{y\in\mathbb{R}\setminus\{0\}} xy\Pr[X=x, Y=y] \\
        &= \left(\sum_{x\in\mathbb{R}\setminus\{0\}}x\Pr[X=x]\right)\left(\sum_{y\in\mathbb{R}\setminus\{0\}}y\Pr[Y=y]\right) = \expec[X]\expec[Y].
    \end{align*}
    
    Now, take the case where $X$ and $Y$ are each non-negative. (They may take an infinite number of values)
    Using \cref{measurable function simple sequence}(a), let $(X_n)_{n\in\mathbb{N}},(Y_n)_{n\in\mathbb{N}}$ be sequences of simple functions such that $X_n\uparrow X$ and $Y_n\uparrow Y$. By \cref{Monotone Convergence Theorem},
    \begin{align*}
        \expec[XY] &= \lim_{n\to\infty}\expec[X_nY_n] \\
        &= \lim_{n\to\infty} \expec[X_n]\expec[Y_n] \\
        &= \lim_{n\to\infty} \expec[X_n] \lim_{n\to\infty} \expec[Y_n] = \expec[X]\expec[Y].
    \end{align*}
    
    The general result where $X$ and $Y$ need not be non-negative is easily proved by splitting $X$ into $X^+-X^-$ and $Y$ as $Y^+-Y^-$.
\end{proof}

The converse however, is not true. That is, uncorrelated variables need not be independent. For example, let $X$ take $0$ and $1$ with probability $1/2$ each and $Y$ take $-1$ and $1$ with probability $1/2$ each. Then $XY$ and $X$ are uncorrelated but not independent.

\begin{theorem}[Wald's Identity]
\label{walds identity randvarsum expec}
    Let $T,X_1,X_2,\ldots\in\mathcal{L}^1(\textbf{P})$ such that $X_1,X_2,\ldots$ are identically distributed and let $T$ take values only in $\mathbb{N}_0$. Define $S_T=\sum_{i=1}^T X_i$. Then $S_T\in\mathcal{L}^1(\textbf{P})$ and $\expec[S_T]=\expec[T]\expec[X_1]$.
\end{theorem}
\begin{proof}
    For each $n\in\mathbb{N}$, let $S_n=\sum_{i=1}^n X_n$. Then we can write $S_T$ as
    $$S_T=\sum_{n=0}^\infty \indic_{\{T=n\}} S_n.$$
    We may prove that $S_T\in\mathcal{L}^1(\textbf{P})$ by performing the following calculation using $|S_T|$ instead of $S_T$.
    
    Since $\indic_{\{T=n\}}$ and $S_n$ are independent for each $n\in\mathbb{N}$, we now have
    \begin{align*}
        \expec[S_T] &= \expec\left[\sum_{n=0}^\infty \indic_{\{T=n\}} S_n\right] \\
        &= \sum_{n=0}^\infty \expec[\indic_{\{T=n\}}]\expec[S_n] \\
        &= \sum_{n=0}^\infty \Pr[T=n] \expec\left[\sum_{i=0}^n X_i\right] \\
        &= \sum_{n=0}^\infty n \Pr[T=n] \expec[X_1]) \\
        &= \expec[T]\expec[X_1].
    \end{align*}
\end{proof}

Let us now discuss some properties of the variance.

\begin{theorem}[Properties of Variance]
\label{properties of variance}
    Let $X\in\mathcal{L}^2(\textbf{P})$. Then
    \begin{enumerate}[(a)]
        \item $\Var[X]=\expec[(X-\expec[X])^2] \geq 0$,
        \item $\Var[X]=0$ if and only if $X=\expec[X]$ almost surely, and
        \item the map $f:\mathbb{R}\to\mathbb{R}$ given by $x\mapsto \expec[(X-x)^2]$ attains its minimum at $x_0=\expec[X]$ taking value $f(x_0)=\Var[X]$.
    \end{enumerate}
\end{theorem}
\begin{proof}
    ~
    \begin{enumerate}[(a)]
        \item This is clear from the fact that $\Cov[X,X]=\Var[X]$.
        \item This follows as $\expec[(X-\expec[X])^2]=0$ if and only if $(X-\expec[X])^2=0$ almost surely.
        \item Expanding the expression of $f$ as $\expec[X^2]+x^2-2x\expec[X]$, we see that $f(x)=\Var[X]+(x-\expec[X])^2$. The result is clear.
    \end{enumerate}
\end{proof}

\begin{theorem}
    Let $X_1,\ldots,X_m,Y_1,\ldots,Y_n\in\mathcal{L}^2(\textbf{P})$ and $\alpha_1,\ldots,\alpha_m,\beta_1,\ldots,\beta_n,d,e\in\mathbb{R}$. Then
    $$\Cov\left[d+\sum_{i=1}^m \alpha_iX_i, e+\sum_{j=1}^n \beta_jY_j\right]=\sum_{\substack{1\leq i\leq m \\ 1\leq j\leq n}} \alpha_i\beta_j \Cov[X_i, Y_j].$$
\end{theorem}

\begin{proof}
    By \cref{expectation rules},
    \begin{align*}
        \Cov\left[d+\sum_{i=1}^m \alpha_iX_i, e+\sum_{j=1}^n \beta_jY_j\right] &= \expec\left[\left(\sum_{i=1}^m \alpha_i(X_i-\expec[X_i])\right)\left(\sum_{j=1}^m \beta_i(Y_j-\expec[Y_j])\right)\right] \\
        &= \sum_{\substack{1\leq i\leq m \\1\leq j\leq n}} \alpha_i\beta_j \expec[(X_i-\expec[X_i])(Y_j-\expec[Y_j])] \\
        &= \sum_{\substack{1\leq i\leq m \\1\leq j\leq n}} \alpha_i\beta_j \Cov[X_i,Y_j].
    \end{align*}
\end{proof}

{
The above can be stated more concisely as:
\begin{quote}
The map $\Cov:\mathcal{L}^2(\textbf{P})\times\mathcal{L}^2(\textbf{P})\to\mathbb{R}$ is a positive semidefinite symmetric bilinear form and $\Cov[X,Y]=0$ if $Y$ is almost surely constant.
\end{quote}
}

Breaking up the individual terms,
\begin{itemize}
    \item $\Cov[X,Y]=0$ if $Y$ is almost surely constant means that the $d$ and $e$ in the equation do not contribute to the covariance.
    \item ``Symmetric bilinear form" means that $\Cov[X,Y]=\Cov[Y,X]$, $\Cov[X_1+X_2,Y]=\Cov[X_1,Y]+\Cov[X_2,Y]$, and $\Cov[\alpha X,Y]=\alpha\Cov[X,Y]$ for all $X,Y,X_1,X_2\in\mathcal{L}^2(\textbf{P})$ and $\alpha\in\mathbb{R}$.
    \item ``Positive semidefinite" means that $\Cov[X,X]\geq 0$ for all $X\in\mathcal{L}^2(\textbf{P})$. This is because $\Cov[X,X]=\Var[X]\geq 0$.
\end{itemize}

\begin{corollary}
\label{bienayme formula}
    For any $X,X_1,\ldots,X_m\in\mathcal{L}^2(\textbf{P})$ and $\alpha\in\mathbb{R}$, $\Var[\alpha X]=\alpha^2\Var[X]$ and
    $$\Var\left[\sum_{i=1}^m X_i\right]=\sum_{i=1}^m \Var[X_i] + \sum_{\substack{1\leq i,j\leq m \\ i\neq j}} \Cov[X_i, X_j].$$
    In particular, for uncorrelated $X_1,\ldots,X_m$,
    $$\Var\left[\sum_{i=1}^m X_i\right] = \sum_{i=1}^m \Var[X_i].$$
\end{corollary}

The above is known as the \textit{Bienaym\'{e} formula}.

\begin{ftheo}[Cauchy-Schwarz Inequality]
    Let $X,Y\in\mathcal{L}^2(\textbf{P})$. Then
    $$(\Cov[X,Y])^2\leq \Var[X]\Var[Y].$$
\end{ftheo}
\begin{proof}
    If $\Var[Y]=0$, the statement is trivial. If $\Var[Y]\neq 0$,
    \begin{align*}
        0 &\leq \Var\left[X - \frac{\Cov[X,Y]}{\Var[Y]}Y\right]\Var[Y] \\
        &= \left(\Var[X] + \left(\frac{\Cov[X,Y]}{\Var[Y]}\right)^2\Var[Y] - 2\frac{\Cov[X,Y]}{\Var[Y]}\Cov[X,Y]\right)\Var[Y] \\
        &= \Var[X]\Var[Y] - \Cov[X,Y]^2.
    \end{align*}
\end{proof}

Our choice of $\frac{\Cov[X,Y]}{\Var[Y]}$ in the above is not arbitrary. The term
$$\rho=\frac{\Cov[X,Y]}{\sqrt{\Var[X]\Var[Y]}}$$
gives a measure of the correlation of $X$ and $Y$, and is sometimes called the \textit{correlation coefficient}. It shows how linearly dependent $X-\expec[X]$ and $Y-\expec[Y]$ are.

The Cauchy-Schwarz Inequality applied to $X-\expec[X]$ and $Y-\expec[Y]$ implies that $-1\leq\rho\leq 1$. Now, let us try to estimate $Y-\expec[Y]$ by a linear combination $a(X-\expec[X])+d$. We shall try to minimize
\begin{align*}
    \expec[(Y-\expec[Y]-c(X-\expec[X])-d)^2] &= \Var[Y] + c^2\Var[X]-2c\Cov[X,Y]+d^2 \\
    &= \Var[Y] + c^2\Var[X]-2c\rho\sqrt{\Var[X]\Var[Y]}+d^2.
\end{align*}
We must clearly take $d=0$. The minimum of the resulting expression is attained at $c=\rho\sqrt{\frac{\Var[Y]}{\Var[X]}}$, and the minimum expectation is $\Var[Y](1-\rho^2)$. The closer $\rho$ is to $1$, the more linearly dependent $Y-\expec[Y]$ and $X-\expec[X]$ are. Here, by linearly dependent, we mean that there exist $a_1,a_2$ not both $0$ such that $a_1(X-\expec[X])+a_2(Y-\expec[Y])=0$ almost surely.

Further, $|\rho|=1$ if and only if $X-\expec[X]$ and $Y-\expec[Y]$ are linearly dependent.

\vspace{1mm}
It follows that equality holds in the Cauchy-Schwarz inequality if and only if there exist $a,b,c\in\mathbb{R}$ not all $0$ such that $aX+bY+c=0$ almost surely.

\begin{theorem}[Blackwell-Girshick Equation]
\label{blackwell girshick randvarsum variance}
    Let $T,X_1,X_2,\ldots\in\mathcal{L}^1(\textbf{P})$ such that $X_1,X_2,\ldots$ are identically distributed and let $T$ take values only in $\mathbb{N}_0$. Define $S_T=\sum_{i=1}^T X_i$. Then
    $$\Var[S_T]=\expec[X_1]^2\Var[T] + \expec[T]\Var[X_1].$$
\end{theorem}
\begin{proof}
    For each $n\in\mathbb{N}$, let $S_n=\sum_{i=1}^n X_n$. Then writing $S_T$ as we did in the proof of \hyperref[walds identity randvarsum expec]{Wald's Identity},
    \begin{align*}
        \expec[S_T^2] &= \expec\left[\sum_{n=0}^\infty \indic_{\{T=n\}} S_n\right] \\
        &= \sum_{n=0}^\infty \expec[\indic_{\{T=n\}}]\cdot\expec[S_n^2] \\
        &= \sum_{n=0}^\infty \Pr[T=n]\left(\Var[S_n]+\expec[S_n]^2\right) \\
        &= \sum_{n=0}^\infty \Pr[T=n]\left(n\Var[X_1]+n^2\expec[X_1]^2\right) \\
        &= \expec[T]\Var[X_1] + \expec[T^2]\expec[X_1]^2.
    \end{align*}
    Now, \hyperref[walds identity randvarsum expec]{Wald's identity} implies
    \begin{align*}
        \Var[S_T] &= \expec[T]\Var[X_1] + \expec[T^2]\expec[X_1]^2 - (\expec[T]\expec[X_1])^2 \\
        &= \expec[T]\Var[X_1] + \Var[T]\expec[X_1]^2.
    \end{align*}
\end{proof}

We now state the expectations and variances for some of the probability distributions that we mentioned earlier in \cref{examples of random variables}.

\begin{enumerate}
    \item Let $p\in[0,1]$ and $X\sim\Ber_p$. Then $\expec[X]=p$ and $\Var[X]=p(1-p)$.
    
    \item Let $p\in[0,1]$, $n\in\mathbb{N}$ and $X\sim b_{n,p}$. Then $\expec[X]=np$ and $\Var[X]=np(1-p)$. This can easily be shown by noting that $X$ is equal to the sum of $n$ i.i.d. random variables, each of which has distribution $\Ber_p$.
    
    \item Let $\mu\in\mathbb{R}$, $\sigma_2>0$ and $X\sim\mathcal{N}_{\mu,\sigma^2}$. Then $\expec[X]=\mu$ and $\Var[X]=\sigma^2$.
    
    \item Let $\theta>0$ and $X\sim\exp_\theta$. Then $\expec[X]=\theta^{-1}$ and $\Var[X]=\theta^{-2}$.
\end{enumerate}

\subsection{The Weak Law of Large Numbers}

We earlier mentioned that variance gives a measure of how much a random variable deviates from the expectation. This is made rigorous by the Chebyshev inequality.

\begin{ftheo}[Markov Inequality]
\label{markov inequality}
    Let $X$ be a real random variable and $f:[0,\infty)\to[0,\infty)$ be monotone increasing. Then for any $\varepsilon>0$ with $f(\varepsilon)>0$,
    $$\Pr[|X|\geq\varepsilon]\leq\frac{\expec[f(|X|)]}{f(\varepsilon)}.$$
\end{ftheo}
\begin{proof}
    We have
    \begin{align*}
        \expec[f(|X|)]&\geq \expec[f(|X|)\indic_{f(|X|)\geq f(\varepsilon)}] \\
        &\geq \expec[f(\varepsilon)\indic_{f(|X|)\geq f(\varepsilon)}] \\
        &\geq f(\varepsilon)\Pr[|X|\geq \varepsilon].
    \end{align*}
\end{proof}

The Markov inequality forms a basis for most probability related inequalities we encounter.

\begin{corollary}[Chebyshev Inequality]
\label{chebyshev inequality}
    Let $X\in\mathcal{L}^2(\textbf{P})$ and $\varepsilon>0$. Then
    $$\Pr[|X-\expec[X]|\geq\varepsilon]\leq \varepsilon^{-2}\Var[X].$$
\end{corollary}
\begin{proof}
    Using the Markov inequality on $f:x\mapsto x^2$ and $X-\expec[X]$ gives the required result.
\end{proof}

This shows that the variance quantifies how much a random variable deviates from it's expectation.

\begin{fdef}[Convergence of Random Variables]
    Let $(X_n)_{n\in\mathbb{N}}$ be a sequence of random variables.
    \begin{enumerate}[(i)]
        \item $(X_n)_{n\in\mathbb{N}}$ is said to \textit{converge in probability} towards a random variable $X$ if for all $\varepsilon>0$,
        $$\lim_{n\to\infty}\Pr[|X_n-X|>\varepsilon]=0.$$
        \item $(X_n)_{n\in\mathbb{N}}$ is said to \textit{converge almost surely} towards a random variable $X$ if
        $$\Pr\left[\lim_{n\to\infty}X_n=X\right]=1.$$
    \end{enumerate}
\end{fdef}

To see that these two are not the same, consider the sequence of random variables $(X_n)_{n\in\mathbb{N}}$ where $X_n$ takes $1$ with probability $1/n$ and $0$ otherwise. Then $(X_n)_{n\in\mathbb{N}}$ converges in probability to the $0$ random variable, but does not converge almost surely.

\vspace{2mm}
Given a random variable $X$, we expect the arithmetic mean of a large number of independent observations of $X$ to be close to $\expec[X]$ (assuming, of course, that its variance is finite.).

\vspace{1mm}
Making this property more general, we define the following.

\begin{fdef}[Laws of Large Numbers]
    Let $(X_n)_{n\in\mathbb{N}}$ be a sequence of real random variables in $\mathcal{L}^1(\textbf{P})$ and let $\tilde{S}_n=\sum_{i=1}^n (X_n-\expec[X_n])$.
    \begin{enumerate}[(i)]
        \item We say that $(X_n)_{n\in\mathbb{N}}$ satisfies the \textit{weak law of large numbers} if for any $\varepsilon>0$,
        $$\lim_{n\to\infty}\left(\Pr\left[\left|\frac{1}{n}\tilde{S}_n\right| < \varepsilon\right]\right)=1.$$
        
        \item We say that $(X_n)_{n\in\mathbb{N}}$ satisfies the \textit{strong law of large numbers} if
        $$\Pr\left[\lim_{n\to\infty}\left|\frac{1}{n}\tilde{S}_n\right|=0\right]=1.$$
    \end{enumerate}
\end{fdef}

We sometimes abbreviate ``law of large numbers" to LLN.

The weak law is equivalent to saying that $(\tilde{S}_n/n)_{n\in\mathbb{N}}$ converges in probability to the $0$ random variable and the strong law is equivalent to saying that $(\tilde{S}_n/n)_{n\in\mathbb{N}}$ converges almost surely to the $0$ random variable.

\vspace{2mm}
%The weak law says that for a specified large $n$, $|\tilde{S}_n/n|$ is likely to be near $0$. That is, it allows for $|\tilde{S_n}/n|>\varepsilon$ happening an infinite number of times, although at infrequent intervals.
The weak law says that for any nonzero (specified) margin, the average of a sufficiently large number of observations will be within the margin of the expectation with high probability.

\vspace{1mm}
%On the other hand, the strong law says that this will almost surely not occur. It implies that with probability $1$, $|\tilde{S}_n/n|<\varepsilon$ holds for all large enough $n$.
On the other hand, the strong law says that almost surely, the sequence of \textit{sample} means converges to the expectation.

\vspace{2mm}
Note that as a consequence, if $(X_n)_{n\in\mathbb{N}}$ satisfies the strong law of large numbers, it must also satisfy the weak law of large numbers.

\begin{ftheo}
\label{uncorr finite var wLLN}
    Let $X_1,\ldots,X_n\in\mathcal{L}^2(\textbf{P})$ be uncorrelated random variables with $V=\sup_{n\in\mathbb{N}}\Var[X_n]<\infty$. Then $(X_n)_{n\in\mathbb{N}}$ satisfies the weak law of large numbers. Further,
    $$\Pr\left[\left|\frac{1}{n}\tilde{S}_n\right|\geq\varepsilon\right]\leq \frac{V}{\varepsilon^2n}\text{ for all $n\in\mathbb{N}$}.$$
\end{ftheo}
\begin{proof}
    Without loss of generality, assume $\expec[X_i]=0$ for all $i\in\mathbb{N}$. Then the \hyperref[bienayme formula]{Bienaym\'{e} formula} implies that
    $$\Var\left[\frac{1}{n}\tilde{S}_n\right]=\frac{1}{n^2}\sum_{i=1}^n \Var[X_i] \leq \frac{V}{n}.$$
    This shows how the more random variables we add, the more it ``concentrates" around the mean $0$.
    Now, by \hyperref[chebyshev inequality]{Chebyshev's inequality},
    $$\Pr\left[\left|\frac{1}{n}\tilde{S}_n\right|\geq\varepsilon\right]\leq \frac{V}{\varepsilon^2n}.$$
    Taking the limit as $n\to\infty$ gives that $(X_n)_{n\in\mathbb{N}}$ satisfies the weak LLN.
\end{proof}

In general, if we wish to show the existence of an object that has some properties, we give a constructive proof. Equipped with probability now, we have a non-constructive method to prove the existence of something, commonly known as the probabilistic method of proof. We do this by considering some distribution over all objects, and showing that the probability of a randomly selected object having the required properties is non-zero, thus implying the existence of the required object.

An example of this is the following.

\begin{corollary}[Weirstrass' Approximation Theorem]
    Let $f:[0,1]\to\mathbb{R}$ be a continuous real map. Then for $n\in\mathbb{N}$ there exist polynomials $f_n$ of degree at most $n$ such that
    $$\norm{f_n(x)-f(x)}_\infty \xrightarrow{n\to\infty} 0.$$
\end{corollary}
\begin{proof}
    For $n\in\mathbb{N}$, define $f_n:[0,1]\to\mathbb{R}$, known as the Bernstein polynomial of order $n$, by
    $$f_n(x) = \sum_{k=0}^n f(k/n)\binom{n}{k}x^k(1-x)^{n-k}$$
    As $f$ is continuous on the compact interval $[0,1]$, it is uniformly continuous. Fixing some $\varepsilon>0$, there exists $\delta>0$ such that
    $$|f(x)-f(y)|<\varepsilon\text{ for all $x,y$ such that } |x-y|<\delta.$$
    Now, let $p\in[0,1]$, $X_1,X_2,\ldots$ be i.i.d. random variables with distribution $\Ber_p$, and $S_n=\sum_{i=1}^n X_n\sim b_{n,p}$. Note that
    $$\expec[f(S_n/n)]=f(p).$$
    Uniform continuity implies that
    $$|f(S_n/n) - f(p)| \leq \varepsilon + 2\norm{f}_\infty\indic_{|S_n/n - p| \geq \delta}$$
    Therefore,
    \begin{align*}
        |f_n(p) - f(p)| &\leq \expec[|f(S_n/n) - f(p)|] \\
        &\leq \varepsilon + 2\norm{f}_\infty\Pr\left[\left|\frac{S_n}{n} - p\right|\geq\delta\right] \\
        &\leq \varepsilon + 2\norm{f}_\infty\cdot \frac{p(1-p)}{n\delta^2} \\
        &\leq \varepsilon + \frac{\norm{f}_\infty}{2n\delta^2}.
    \end{align*}
    The result follows.
\end{proof}

\begin{corollary}
    Let $n\in\mathbb{N}$ and $p_1,\ldots,p_n\in[0,1]$. Let $X_1,\ldots,X_n$ be independent random variables such that $X_i\sim\Ber_{p_i}$ for each $i\in[n]$. Define $S_n=\sum_{i=1}^n X_i$ and $m=\expec[S_n]$. Then for any $\delta>0$,
    $$\Pr\left[S_n\geq(1+\delta)m\right]\leq \left(\frac{e^\delta}{(1+\delta)^{1+\delta}}\right)^m$$
    and
    $$\Pr\left[S_n\leq(1-\delta)m\right]\leq \exp\left(-\frac{\delta^2m}{2}\right).$$
\end{corollary}
\begin{proof}
    For some $\lambda>0$, consider $f:[0,\infty)\to[0,\infty)$ given by $f(x)=e^{\lambda x}$. By the \hyperref[markov inequality]{Markov Inequality}, for any $\varepsilon>0$,
    \begin{align*}
        \Pr[S_n\geq (1+\delta)m] &\leq \frac{\expec[e^{\lambda S_n}]}{e^{\lambda m(1+\delta)}} \\
        &= e^{-\lambda m(1+\delta)}\prod_{i=1}^n \expec[e^{\lambda X_i}]
    \end{align*}
    Now, for each $i\in[n]$,
    $$\expec[e^{\lambda X_i}] = 1+p_i(e^\lambda-1)\leq e^{p_i(e^{\lambda}-1)}.$$
    Therefore, since $m=p_1+\cdots+p_n$,
    \begin{align*}
        \Pr[S_n\geq (1+\delta)m] &\leq e^{-\lambda m(1+\delta)}\prod_{i=1}^n e^{p(e^\lambda - 1)} \\
        &= e^{-\lambda m(1+\delta)} e^{m(e^\lambda - 1)}.
    \end{align*}
    We can now optimize over $\lambda$ to find the minimum of the expression on the right. Setting $\lambda=\ln(1+\delta)$, we obtain
    $$\Pr[S_n\geq (1+\delta)m]\leq \left(\frac{\delta}{(1+\delta)^{1+\delta}}\right)^m.$$
    
    The second bound can be obtained similarly as we can optimize the following inequality over $\lambda$:
    \begin{align*}
        \Pr[X\leq (1-\delta)m] &= \Pr[e^{-\lambda X}\geq e^{-\lambda m(1-\delta)}] \\
        &\leq \frac{\expec[e^{S_n}]}{e^{-\lambda m(1-\delta)}}.
    \end{align*}
\end{proof}

More generally, if $X$ is the sum of $n$ independent random variables $X_1,\ldots,X_n$, then
$$\Pr[X\geq a]\leq \min_{t>0} e^{-ta}\prod_i\expec[e^{tX_i}]$$
and
$$\Pr[X\leq a]\leq \min_{t>0} e^{ta}\prod_i\expec[e^{-tX_i}].$$
The above inequalities are known as the \textit{Chernoff Bounds}.

\subsection{The Strong Law of Large Numbers}

There are many versions of the Strong Law of Large Numbers, the one we present here is that given by Etemadi.

Namely, if $X_1,X_2,\ldots\in\mathcal{L}^2(\textbf{P})$ are pairwise independent and identically distributed, then $(X_n)_{n\in\mathbb{N}}$ follows the strong LLN.

Before we prove this, we start with some lemmas.

Define $\mu=\expec[X_1]$ and $S_n=X_1+\cdots+X_n$.

\begin{lemma}
\label{prlem1etem}
    For $n\in\mathbb{N}$, let $Y_n=X_n\indic_{\{|X_n|\leq n\}}$ and $T_n=Y_1+\cdots+Y_n$. Then $(X_n)_{n\in\mathbb{N}}$ follows the strong law of large numbers if $T_n/n\xrightarrow{n\to\infty}\mu$ almost surely.
\end{lemma}
\begin{proof}
    By \cref{geq leq int leq g}, $$\sum_{n=1}^\infty \Pr[|X_n|> n]\leq\sum_{n=1}^\infty \Pr[|X_n|\geq n]\leq\expec[X_1]<\infty.$$ Now due to to the \hyperref[borelCantelliLemma]{Borel-Cantelli Lemma},
    $$\Pr[X_n\neq Y_n\text{ for infinitely many $n$}]=0.$$
    There then almost surely exists some $n_0$ such that $X_n=Y_n$ for all $n\geq n_0$. Therefore, for all $n\geq n_0$,
    $$\frac{T_n-S_n}{n} = \frac{T_{n_0}-S_{n_0}}{n}\xrightarrow{n\to\infty}0.$$
    Since $T_n/n\xrightarrow{n\to\infty}\mu$ almost surely, $S_n/n\xrightarrow{n\to\infty}\mu$ almost surely and our proof is complete.
\end{proof}

\begin{lemma}
    For all $x>0$, $2x\sum_{n>x}n^{-2}\leq 4$.
\end{lemma}
\begin{proof}
    For any $m\in\mathbb{N}$,
    $$\sum_{n=m}^\infty n^{-2} \leq m^{-2} + \int_m^\infty t^{-2}\d t = m^{-2} + m^{-1} \leq \frac{2}{m}.$$
    The result follows.
\end{proof}

\begin{lemma}
\label{prlem2etem}
    With the above notation,
    $$\sum_{n=1}^\infty \frac{\expec[Y_n^2]}{n^2}\leq 4\expec[|X_1|].$$
\end{lemma}
\begin{proof}
    By \cref{int func eq int geq},
    $$\expec[Y_n^2]=\int_0^\infty \Pr(Y_n^2\geq t)\d t.$$
    Simplifying by substituting $t=x^2$,
    $$\expec[Y_n^2] = \int_0^\infty 2x\Pr[|Y_n|\geq x]\d x\leq \int_0^n 2x\Pr[|X_1|\geq x]\d x.$$
    For $m\in\mathbb{N}$, define the function $f_m$ by
    
    $$f_m(x) = \left(\sum_{n=1}^m n^{-2}\indic_{\{x<n\}}\right)2x\Pr[|X_1|>x]\cdot.$$
    Let $f_m\uparrow f$. Note that $f\leq 4\Pr[|X_1|>x]$.
    
    Finally, by the \hyperref[Monotone Convergence Theorem]{Monotone Convergence Theorem},
    \begin{align*}
        \sum_{n=1}^\infty \frac{\expec[Y_n^2]}{n^2} &\leq \sum_{n=1}^\infty n^{-2} \int_0^n 2x\Pr[|X_1|\geq x]\d x \\
        &= \int_0^\infty \left(\sum_{n=1}^\infty n^{-2}\indic_{\{x<n\}}\right) 2x\Pr[|X_1|>x] \\
        &\leq 4\int_0^\infty \Pr[|X_1|>x] = 4\expec[|X_1|].
    \end{align*}
\end{proof}

We may now prove the main result of this subsection.

\begin{ftheo}[Etemadi's Strong Law of Large Numbers\cite{etemadi}]
\label{etemadi sLLN}
    Let $X_1,X_2,\ldots\in\mathcal{L}^1(\textbf{P})$ be pairwise independent and identically distributed random variables. If $\expec[|X_1|]<\infty$, then $(X_n)_{n\in\mathbb{N}}$ satisfies the strong law of large numbers.
\end{ftheo}
\begin{proof}
    Since $(X_n^+)_{n\in\mathbb{N}}$ and $(X_n^-)_{n\in\mathbb{N}}$ satisfy the conditions of the theorem, it suffices to consider only $(X_n^+)_{n\in\mathbb{N}}$, that is, $X_n\geq 0$ for all $n\in\mathbb{N}$.
    
    Fix some $\varepsilon>0$ and $\alpha>1$ and let $k_n=\lfloor\alpha^n\rfloor$ for each $n$. Then by \hyperref[chebyshev inequality]{Chebyshev's inequality},
    \begin{align*}
        \sum_{n=1}^\infty \Pr\left[\left|\frac{T_{k_n}-\expec[T_{k_n}]}{k_n}\right|>\varepsilon\right] &\leq \varepsilon^{-2} \sum_{n=1}^\infty \frac{\Var[T_{k_n}]}{k_n^2} \\
        &= \varepsilon^{-2} \sum_{n=1}^\infty \frac{1}{k_n^2}\sum_{i=1}^{k_n} \Var[Y_i] \\
        &= \varepsilon^{-2} \sum_{i=1}^\infty \Var[Y_i] \sum_{n:k_n\geq m} \frac{1}{k_n^2} \\
        &\leq \frac{1}{\varepsilon^2(1-\alpha^{-2})} \sum_{i=1}^\infty \frac{\Var[Y_i]}{m^2} = \frac{1}{\varepsilon^2(1-\alpha^{-2})} \sum_{i=1}^\infty \frac{\expec[Y_i^2]}{m^2}
    \end{align*}
    By \cref{prlem2etem}, this is finite. Letting $\varepsilon\downarrow 0$, the \hyperref[borelCantelliLemma]{Borel-Cantelli Lemma} implies that
    $$\lim_{n\to\infty} \frac{T_{k_n} - \expec[T_{k_n}]}{k_n}=0\text{ almost surely.}$$
    By the \hyperref[Monotone Convergence Theorem]{Monotone Convergence Theorem}, $\expec[Y_n]\xrightarrow{n\to\infty}\expec[X_1]$.
    
    This in turn gives that $\expec[T_{k_n}]/k_n\xrightarrow{n\to\infty}\expec[X_1]$ and therefore, 
    $$\lim_{n\to\infty}\frac{T_{n}}{n}=\lim_{n\to\infty}\frac{T_{k_n}}{k_n}=\expec[X_1]\text{ almost surely.}$$
    \Cref{prlem1etem} completes the proof.
\end{proof}

\vspace{2mm}
We now give an example of the applications of the above.

Let $f\in\mathcal{L}^1([0,1])$ be a function  $[0,1]\to\mathbb{R}$ and say we want to evaluate $I=\int_0^1 f(x)\d x$. Let $X_1,X_2,\ldots$ be independent random variables uniformly distributed in $[0,1]$ and define
$$\Hat{I}_n = \frac{1}{n}\sum_{i=1}^n f(X_i).$$
The \hyperref[etemadi sLLN]{strong law of large numbers} implies that $\Hat{I}_n\xrightarrow{n\to\infty} I$ almost surely. This method of integration is known as \textit{Monte Carlo integration}.

If $f$ is a sufficiently nice function, usual numerical methods tend to give better estimates. However, If $[0,1]$ is replaced with $G\subseteq\mathbb{R}^d$ for very large $d$, Monte Carlo integration tends to be useful. 

\vspace{1mm}
Note that although $\Hat{I}_n$ converges to $I$, we do not have an estimate of how fast it converges.

\subsection{Speed of Convergence in the Strong Law of Large Numbers}

We had discussed the speed of convergence in the weak LLN (\cref{uncorr finite var wLLN}).

In the strong LLN, we get good estimates if we assume the existence of higher moments, such as the following result that can be seen as an analogue of the \hyperref[chebyshev inequality]{Chebyshev Inequality}.

\begin{ftheo}[Kolmogorov's Inequality]
\label{kolmogorov's inequality}
    Let $n\in\mathbb{N}$ and $X_1,\ldots,X_n$ be independent random variables with $\expec[X_i]=0$ and $\Var[X_i]<\infty$ for all valid $i$. Let $S_k=X_1+\cdots+X_k$ for $k\in[n]$. Then for any $t>0$,
    $$\Pr[\max\{S_k:k\in[n]\}\geq t] \leq \frac{\Var[S_n]}{t^2+\Var[S_n]}$$
    and
    $$\Pr[\max\{|S_k|:k\in[n]\}\geq t] \leq t^{-2}\Var[S_n].$$
\end{ftheo}
The latter equation in particular is referred to as Kolmogorov's Inequality.
\begin{proof}
    Let $\tau$ be the first time at which the partial sum exceeds $t$, that is,
    $$\tau = \min\{k\in[n]:S_k\geq t\}.$$
    and $A_k=\{\tau=k\}$ for each $k$. Further, let
    $$A=\left\{\max\{S_k:k\in[n]\geq t\right\}=\biguplus_{i=1}^n A_k.$$
    Fix some $c\geq 0$. Note that $(S_k+c)\indic_{A_k}$ is $\sigma(X_1,\ldots,X_k)$-measurable and $(S_n-S_k)$ is $\sigma(X_{k+1},\ldots,X_n)$-measurable. \Cref{independent mutually disjoint set union} implies that these two random variables are independent and thus,
    $$\expec[(S_k+c)\indic_{A_k}(S_n-S_k)]=\expec[(S_k+c)\indic_{A_k}]\expec[S_n-S_k]=0.$$
    We now have
    \begin{align*}
        \Var[S_n] + c^2 &= \expec[(S_n+c)^2] \\
        &\geq \sum_{k\in[n]}\expec[(S_n+c)^2\indic_{A_k}] \\
        &= \sum_{k\in[n]}\expec[\left((S_k+c)^2 + 2(S_k+c)(S_n-S_k)+(S_n-S_k)^2\right)\indic_{A_k}] \\
        &= \sum_{k\in[n]} \expec[\left((S_k+c)^2+(S_n-S_k)^2\right)\indic_{A_k}] \\
        &\geq \sum_{k\in[n]} \expec[(S_k+c)^2\indic_{A_k}] \\
        &\geq \expec[(t+c)^2\indic_{A}] = (t+c)^2\Pr[A].
    \end{align*}
    
    That is,
    $$\Pr[A]\leq \frac{\Var[S_n]+c^2}{(t+c)^2}.$$
    Setting $c=\Var[S_n]/t$ gives the first inequality.
    
    \vspace{2mm}
    For the second inequality, let
    $$\tilde{\tau}=\min\{k\in[n]:|S_k|\geq t\},$$
    $\tilde{A}_k=\{\tilde{\tau}=k\}$ and $\tilde{A}=\{\tilde{\tau}\leq n\}$. Repeating the steps similar to the first inequality but with $c=0$ gives the required result.
\end{proof}

Now that we have Kolmogorov's inequality, we may establish some more results.

\begin{theorem}
    Let $X_1,X_2,\ldots$ be independent random variables with $\expec[X_n]=0$ for any $n\in\mathbb{N}$ and $V=\sup_{n\in\mathbb{N}}\Var[X_n]<\infty$. Then for any $\varepsilon>0$,
    $$\limsup_{n\to\infty} \frac{|S_n|}{n^{1/2}(\log(n))^{\varepsilon+(1/2)}}=0\text{ almost surely.}$$
\end{theorem}
\begin{proof}
    Let $l(n)=n^{1/2}(\log(n))^{\varepsilon+(1/2)}$ and $k_n=2^n$ for $n\in\mathbb{N}$. For sufficiently large $n$ and $k\in\mathbb{N}$ such that $k_{n-1}\leq k\leq k_n$, $|S_k|/l(k)\leq 2|S_k|/l(k_n)$  (Why?). It then suffices to show that for any $\delta>0$,
    $$\limsup_{n\to\infty} \frac{\max\{|S_k|:k\leq k_n\}}{l(k_n)} \leq \delta\text{ almost surely.}$$
    For each $n\in\mathbb{N}$, define $A_{n,\delta}=\{\max\{|S_k|:k\leq k_n\} > \delta l(k_n)\}$
    
    Then by \hyperref[kolmogorov's inequality]{Kolmogorov's inequality},
    
    \begin{align*}
        \sum_{n=1}^\infty \Pr[A_{n,\delta}] &\leq \sum_{n=1}^\infty Vk_n(\delta l(k_n))^{-2} \\
        &= \frac{V}{\delta^2(\log 2)^{1+2\varepsilon}}\sum_{n=1}^\infty \frac{1}{n^{1+2\varepsilon}} < \infty.
    \end{align*}
    Applying the \hyperref[borelCantelliLemma]{Borel-Cantelli Lemma} on $(A_{n,\delta})_{n\in\mathbb{N}}$ gives that $\Pr\left[\limsup_{n\to\infty}A_{n,\delta}\right]=0$, which leads to the result. 
\end{proof}

Readers who are familiar with complexity theory can think of this as just saying that $|S_n|=o\left(n^{1/2}(\log (n))^{\varepsilon+(1/2)}\right)$.

\vspace{2mm}

A stronger result that we do not prove here states that for iid square integrable centered random variables $X_1,X_2,\ldots$,
$$\limsup_{n\to\infty} \frac{|S_n|}{\sqrt{2n\Var[X_1]\log(\log(n))}}=1\text{ almost surely.}$$
This result is significantly stronger as it says that $|S_n|=\Theta\left(\sqrt{2n\Var[X_1]\log(\log(n))}\right)$.



\clearpage

\bibliographystyle{plainnat}   
\bibliography{references}

\end{document}